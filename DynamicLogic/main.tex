\documentclass[a4paper]{article}

\usepackage{color}
\usepackage[fleqn]{amsmath}
\usepackage{amsthm}
\usepackage{amssymb}
\usepackage{stmaryrd}
\usepackage{mathrsfs,cmll}
\usepackage[sans]{dsfont}
\usepackage{graphicx}

\usepackage{lingmacros} % linguistic macros for enumerating sentences
\usepackage{bussproofs} 
\EnableBpAbbreviations

% Dynamic connectives and quantifiers

\newcommand{\dynexists}{\mathord{\reflectbox{$\mathds{E}$}}}

\newcommand{\dynall}{\mathord{%
\raisebox{\depth}{\rotatebox{180}{$\mathds{A}$}}}}

\newcommand{\dynor}{\mathbin{\mathds{V}}}

\newcommand{\dynand}{\mathbin{%
\reflectbox{\raisebox{\depth}{\rotatebox{180}{${\dynor}$}}}}}

\newcommand{\dynimplies}{\mathbin{%
\mbox{\raisebox{.28ex}{\rule{1.7ex}{.15ex}}%
\hspace{-1.7ex}\raisebox{.73ex}{\rule{1.7ex}{.15ex}}%
\hspace{-.4ex}\raisebox{-.01ex}{\rotatebox{45}{\rule{.83ex}{.15ex}}}%
\hspace{-.69ex}\raisebox{.58ex}{\rotatebox{135}{\rule{.83ex}{.15ex}}}}
}}

\newcommand{\dynneg}{\mathop{%
\mbox{\raisebox{.68ex}{\rule{1.2ex}{.15ex}}%
\hspace{-.15ex}\raisebox{.2ex}{\rule{.15ex}{.63ex}}%
\hspace{.25ex}\raisebox{.2ex}{\rule{.15ex}{.63ex}}}
}}

% lambda-terms

\newif\ifinnerapp \newif\ifinnerabs

\def\abs#1#2%
{\ifinnerabs%
(\lambda #1 . \,%
\innerappfalse\innerabsfalse #2)%
\else%
\lambda #1 . \,%
\innerappfalse\innerabsfalse #2%
\fi\innerappfalse\innerabsfalse}

\def\app#1#2% 
{\ifinnerapp%
(\innerappfalse\innerabstrue #1 \, \innerapptrue\innerabstrue #2)%
\else%
\innerappfalse\innerabstrue #1 \, \innerapptrue\innerabstrue #2%
\fi\innerappfalse\innerabsfalse}



%
%       Italized theorem-like environments
%
\newtheorem{theorem}{Theorem}[section]
\newtheorem{conjecture}[theorem]{Conjecture}
\newtheorem{lemma}[theorem]{Lemma}
\newtheorem{proposition}[theorem]{Proposition}
\newtheorem{corollary}[theorem]{Corollary}
\newtheorem{principle}[theorem]{Principle}
\newtheorem{fact}[theorem]{Fact}
\newtheorem{definition}[theorem]{Definition}
\newtheorem{remark}[theorem]{Remark}
\newtheorem{notation}[theorem]{Notation}

%
%       Non-italized theorem-like environments:
%
%\newtheorem{@definition}[theorem]{Definition}
%\newenvironment{definition}{\begin{@definition}\upshape}{\end{@definition}}
%\newtheorem{@remark}[theorem]{Remark}
%\newenvironment{remark}{\begin{@remark}\upshape}{\end{@remark}}
%\newtheorem{@notation}[theorem]{Notation}
%\newenvironment{notation}{\begin{@notation}\upshape}{\end{@notation}}
\newtheorem{@example}[theorem]{Example}
\newenvironment{example}{\begin{@example}\upshape }{\end{@example}}
\newtheorem{@convention}[theorem]{Convention}
\newenvironment{convention}{\begin{@convention}\upshape}{\end{@convention}}
\newtheorem{@note}[theorem]{Note}
\newenvironment{note}{\begin{@note}\upshape}{\end{@note}}


\title{The title of the paper}
\author{Philippe de~Groote
\and
Ekaterina Lebedeva}
\date{}

\begin{document}
\maketitle

\begin{abstract}
Xxx xxx xxx xxx xxx
\end{abstract}


\input{intro}
\section{Mathematical preliminaries} \label{tex:math_prelim}

\cite{Church:1940:A-formulation-of-the-simple-theory-of-types}

In this section the basic definitions and theorems of type-free (\ref{sec:A1}) and simply-typed (\ref{sec:A2}) lambda calculus, based on~\cite{Barendregt:1981:The-Lambda-Calculus:-Its-Syntax-and-Semantics},~\cite{Barendregt:1992:Lambda-Calculi-with-Types} and~\cite{HindleySeldin:2008:Lambda-Calculus-and-Combinators-an-Introduction}, are presented. The last subsection overviews the basic principles of the continuation-passing technique and illustrates them on simple programming examples.

\subsection{Type-free Lambda Calculus} \label{sec:A1}

\begin{definition}[$\lambda$-terms] The set of \textbf{$\lambda$-terms} $\Lambda$ constructed from an enumerable set of variables $V = \{ v, v_1 ,v_2, \dots\}$ is defined inductively as follows:
\begin{center}
$
\begin{array}{rcll}
x \in V & \Longrightarrow & x \in \Lambda &\\
M,N \in \Lambda &  \Longrightarrow & (MN) \in \Lambda & \text{(\textbf{application})}\\ 
 x \in V, M \in \Lambda&  \Longrightarrow  & (\lambda x.M) \in \Lambda & \text{(\textbf{abstraction})}
\end{array} 
$
\end{center}
\end{definition}

In the term $ (\lambda x.M)$, called abstraction, the variable $x$ is the \textbf{argument} of the function and $M$ is the \textbf{body} of the function. %The variable $x$ is \textbf{bound} by $\lambda$ in  $ (\lambda x.M)$.

\begin{example} \label{app:ex1} The following are $\lambda$-terms:
\begin{align*}
& x \\
& (x_1x_2) \\
& (\lambda x. (x_1 x_2)) \\
& (\lambda x_1. (x_1 x_2)) \\
& ((\lambda x. (x_1 x_2)) x_3 ) 
\end{align*}
\end{example}


\begin{remark}[Parenthesis conventions] \label{rem:parcon}
\begin{itemize}
\item application is left-associative
$$MN_1 N_2 \dots N_n \seq (\dots ( (MN_1) N_2 ) \dots N_n)$$
\item  a sequence of $\lambda$-abstractions $\lambda x_1. (\lambda x_2. ( \dots (\lambda x_n. M) )  ) $ is abbreviated as \\ $\lambda x_1 x_2 \dots x_n.M$
$$\lambda x_1 x_2 \dots x_n.M \seq \lambda x_1. (\lambda x_2. ( \dots (\lambda x_n. M) )  )  $$
\item parentheses surrounding the body of an abstraction can be dropped
$$\lambda x_1 x_2 \dots x_n.(M) \seq \lambda x_1 x_2 \dots x_n.M$$
\item outermost parentheses can be dropped 
$$(M) \seq M$$
\end{itemize}
\end{remark}
%application has a higher precedence than abstraction. 
Note that according to the conventions on parentheses, term $\lambda x. MN$ is a more concise way of writing $\lambda x. (MN)$ and is not equivalent to $(\lambda x. M)N$. 

\begin{example} According to Remark~\ref{rem:parcon}, the $\lambda$-terms in Example~\ref{app:ex1} can be written as follows:
\begin{align*}
& x \\
& x_1x_2 \\
& \lambda x. x_1 x_2 \\
& \lambda x_1. x_1 x_2 \\
& (\lambda x. x_1 x_2) x_3 
\end{align*}
\end{example}



\begin{definition}[Free and bound variables] A variable $x$ is \textbf{free} in a $\lambda$-term $M$ if $x$ is not in the scope of $\lambda x$. If $x$ is in the scope of $\lambda x$, it is \textbf{bound}.
\end{definition}

\begin{example} In the term $ (\lambda x. x_1 x_2)$, the variables $x_1$ and $x_2$ are free. In the term $(\lambda x_1. x_1 x_2)$, the variable $x_1$ is bound and the variable $x_2$ is free. In the term $x(\lambda x.x)$, the variable occurs free in the subterm $x$ and bound in the subterm $\lambda x.x$. 
\end{example}

\begin{definition}[Closed $\lambda$-terms] \ 
\begin{enumerate}
\item The set of \textbf{free variables} of $M$, $FV(M)$ is defined inductively as follows:
\begin{align*}
 FV(x)  & = \{ x \} \\
 FV(\lambda x.M)  & = FV(M) - \{ x \} \\
 FV(MN)  & = FV(M) \cup FV(N)
\end{align*}
\item M is \textbf{closed} or a \textbf{combinator}, if $FV(M) = \emptyset$
%\item The result of \textbf{substitution} of $N$ for (the free occurrences of) $x$ in $M$, denoted $x[x:=N]$, is defined as follows ($x \neq y$):
%\begin{center}
%$
%\begin{array}{rcl}
%x[x:=N] & \defeq & N \\
%y[x:=N] &  \defeq & y\\ 
%(PQ)[x:=N]&  \defeq  & (P[x:=N])(Q[x:=N])\\
%(\lambda y.P)[x:=N] & \defeq & \lambda y.(P[x:=N]) \\
%(\lambda x.P)[x:=N] & \defeq & \lambda x.P
%\end{array} 
%$
%\end{center}
\end{enumerate}
\end{definition}

If an equation $M=N$ is provable in the lambda calculus, the provability is denoted by $\lambda \vdash M = N$
or sometimes just by $M=N$.

\begin{definition}[Axioms and rules] For all $M,N, L,Z \in \Lambda$ the following axioms and rules hold:
\begin{prooftree}
\AXC{} \RightLabel{$\beta$-conversion}
\UIC{$(\lambda x.M)N = M[x:=N]$}
\end{prooftree}

\begin{prooftree}
\AXC{}
\UIC{$M = M$}
\end{prooftree}

\begin{prooftree}
\AXC{$M = N$}
\UIC{$N = M$}
\end{prooftree}

\begin{prooftree}
\AXC{$M=N$}
\AXC{$N=L$}
\BIC{$M=L$}
\end{prooftree}

\begin{prooftree}
\AXC{$M=N$}
\UIC{$MZ=NZ$}
\end{prooftree}

\begin{prooftree}
\AXC{$M=N$}
\UIC{$ZM=ZN$}
\end{prooftree}

\begin{prooftree}
\AXC{$M=N$} \RightLabel{rule $\xi$}
\UIC{$\lambda x.M=\lambda x.N$}
\end{prooftree}

\end{definition}

Importantly, substitution $[x:=N]$ in $M$, denoted $M[x:=N]$, is only applicable to the free occurrences of $x$ in $M$. For example,
\begin{align*}
(xy(\lambda x.x))[x:=N] = Ny(\lambda x.x)
\end{align*}

\begin{definition}[Substitution] The result of \textbf{substitution} of $N$ for the free occurences of $x$ in $M$, i.e. $M[x:=N]$, is defined inductively on the structure of $M$ as follows:
\begin{align*}
 x[x:=N] \defeq  \ & N \\
 y[x:=N] \defeq  \ &  y \ \ \text{provided} \ x \nseq y \\
 (\lambda y.M_1)[x:=N] \defeq   \ & \lambda y. (M_1 [x:=N]) \\
 (M_1 M_2)[x:=N] \defeq  \  & (M_1[x:=N])(M_2[x:=N]) \\
 (\lambda x.M_1)[x:=N]  \defeq \ & \lambda x.M_1
\end{align*}

\end{definition}

\begin{lemma}[Substitution lemma] If  $x \nseq y$ and $x \notin FV(L)$, then
\begin{align*}
 M[x:=N][y:=L] \seq M[y:=L][x:=N[y:=L]]
\end{align*}
\end{lemma}
\begin{proof} The proof is by induction on the structure of $M$.
\end{proof}


When performing a substitution $M[x:=N]$, it is necessary to rename those bound variables in $M$ that are free in N. Otherwise, the substitution 
may lead to a false result. For example, without renaming the bound variable $x$ in $\lambda x.xy$, the substitution  $(\lambda x.xy)[y:=x]$ leads to the term $\lambda x.xx$ that acts differently from the desired term, because the free variable $x$ became bound. However, changing $x$ to $z$, for example, before making the substitution, leads to the desired term $\lambda z.zx$. 

\begin{definition}[$\alpha$-conversion] A \textbf{change of bound variable} $x$ in $M$, or an \textbf{$\alpha$-conversion} in $M$, is the replacement of an occurrence of $\lambda x.N$ in $M$ by $\lambda y.(N[x:=y])$, where $y$ does not occur in $N$.
\end{definition}

\begin{definition}[$\alpha$-congruency] $M$ and $N$ are \textbf{$\alpha$-congruent}, denoted $M \congr N$, if one can result from the other by a finite series of changes of bound variables.
\end{definition}

\begin{example}
\begin{align*}
& \lambda x.xy \congr \lambda z.zy \ncongr \lambda x.xx \\
& \lambda xy.yx(\lambda x.x) \congr \lambda xy. yx(\lambda z.z)  \congr \lambda zy. yz(\lambda x.x) \\
& \lambda xy.yx\congr  \lambda zy. yz \congr  \lambda zx. xz \congr \lambda yx. xy 
\end{align*}
\end{example}

It is natural to identify the terms that are $\alpha$-congruent, as they represent the same processes. Moreover, the same processes can be represented by different terms. For example, $\lambda x.Mx$ and $M$ both lead to $MN$ when applied to $N$. Hence, the following rule can be introduced:
\begin{definition}[Extensionality] \textbf{Extensionality} is the following derivation rule, provided $x \notin FV(MN)$:
\begin{prooftree}
\AXC{$Mx = Nx$}
\UIC{$M = N$}
\end{prooftree}
\end{definition}
The extensionality rule allows to prove $\lambda x. Mx = M$. Alternatively,  $\lambda x. Mx = M$ can be considered to be an axiom:
\begin{definition}[$\eta$-conversion] Let $x \notin FV(M)$. Then
\begin{prooftree}
\AXC{} \RightLabel{$\eta$-conversion}
\UIC{$\lambda x. Mx = M$}
\end{prooftree}
\end{definition}

\begin{definition}[$\beta$-normal form]
\begin{enumerate}
\item $M$ is a \textbf{$\beta$-normal form}, if $M$ has no subterm of the form $(\lambda x.L)K$
\item $M$ \textbf{has a $\beta$-normal form}, if there exists an $N$ such that $N$ is a $\beta$-normal form and $N = M$.
\end{enumerate}
\end{definition}

When $M$ is a $\beta$-normal form, it is often said that $M$ is in normal form.

\begin{example} \
\begin{enumerate}
\item $\lambda x.x$ is in normal form.
\item $(\lambda x.x)y$ has a normal form, namely $y$.
\item $(\lambda xy.y)z$ has a normal form, namely $(\lambda y.y)$.
\item $(\lambda xy.x)z$ has a normal form, namely $z$.
\item $(\lambda x.xx)(\lambda y.yy)$ has no normal form.
\end{enumerate}
\end{example}

A notion of reduction on $\Lambda$ is a binary relation on $\Lambda$. The classical notion of reduction $\beta$ is defined as follows:
\begin{definition} $\beta = \{ ( (\lambda x. M)N , M[x:=N] ) | M, N \in \Lambda  \} $
\end{definition}

\begin{definition}[$\beta$-redex, $\beta$-contractum] 
A \textbf{$\beta$-redex} is a term $M$ such that $(M,N) \in \beta$ for some term $N$. In this case $N$ is called \textbf{$\beta$-contractum} of $M$.
\end{definition}

\begin{definition} The notion of reduction $\beta$ induces the following binary relations:
\begin{align*}
\bconv \  & \textbf{one-step } \beta \textbf{-reduction} \\
\bred \  &  \beta \textbf{-reduction} \\
=_{\beta} \  & \beta \textbf{-equality} \text{ (also called } \beta \textbf{-convertibility}\text{)} 
\end{align*}
These relations are inductively defined as follows:
\begin{center}
$
\begin{array}{rcl}
(M,N) \in \beta & \Longrightarrow & M \bconv N \\
M \bconv N &  \Longrightarrow & ZM \bconv ZN \\ 
M \bconv N &  \Longrightarrow & MZ \bconv NZ \\ 
M \bconv N & \Longrightarrow & \lambda x.M \bconv \lambda x.N \\
& & \\
M \bconv N &  \Longrightarrow & M \bred N \\ 
M \bred M & \\ 
M \bred N, N \bred Z & \Longrightarrow & M \bred Z \\
&& \\
M \bred N &  \Longrightarrow & M =_{\beta} N \\ 
M =_{\beta} N & \Longrightarrow & N =_{\beta} M\\ 
M =_{\beta} N, N =_{\beta} Z & \Longrightarrow & M =_{\beta} Z 
\end{array} 
$
\end{center}


%\begin{enumerate}
%\item $(M,N) \in \beta$
%\end{enumerate}

\end{definition}

The notions of $\beta$-redex and $\beta$-equality allow to give an alternative, more formal, definition of $\beta$-normal form:
\begin{definition}[$\beta$-normal form]  
\begin{enumerate}
\item A term $N$ is called a $\beta$-normal form, if $N$ does not contain (as subterm) any $\beta$-redex.
\item A term $N$ is a $\beta$-normal form of $M$, if $N$ is a $\beta$-normal form and $M =_{\beta} N$.
\end{enumerate}
\end{definition}

%\begin{definition}[$\beta$-contractum, $\beta$-redex] Any term of the form 
%\begin{align*} (\lambda x.M)N
%\end{align*}
%is called a \textbf{$\beta$-redex} and the corresponding term
%\begin{align*} M[x:=N]
%\end{align*}
%is called its \textbf{contractum}.
%\end{definition}

%
%\begin{definition}[$\beta$-contracting, $\beta$-reducing]ff
%\end{definition}

The notion of $\beta$-reduction is very important, because it characterizes provability in $\lambda$ and it is Church-Rosser:

\begin{proposition} $M =_{\beta} N$ iff $\lambda \vdash M = N$
\end{proposition}
\begin{proof} See~\cite[p.59]{Barendregt:1981:The-Lambda-Calculus:-Its-Syntax-and-Semantics}.
%\begin{itemize}
%\item Assume $\lambda \vdash M = N$. Use the induction on the length of the proof of $M=N$.
%\item Assume $M =_{\beta} N$. Show by induction on the definition of the relations involved that
%\end{itemize}
\end{proof}


\begin{theorem}[Church-Rosser theorem for $\bred$] If $L \bred M$ and $L \bred N$, then there exists a term $Z$ such that $M \bred Z$ and $N \bred Z$.
\end{theorem}
\begin{proof} See~\cite[p.62]{Barendregt:1981:The-Lambda-Calculus:-Its-Syntax-and-Semantics} or~\cite[p.289]{HindleySeldin:2008:Lambda-Calculus-and-Combinators-an-Introduction}.
\end{proof}
\begin{figure}[h!]
  \centering
    \includegraphics[width=0.3\textwidth]{images/Diamond.pdf}
      \caption{Diamond property.} \label{fig:diamond}
\end{figure}

The Church-Rosser theorem says that for two $\beta$-convertible terms, there is a term to which they both $\beta$-reduce, as illustrated in Figure~\ref{fig:diamond}. The property described in the theorem, that if a term can be reduced to two different terms, then these two terms can be further reduced to one term, is called the \textbf{diamond property} or \textbf{confluence}. The theorem states that $\beta$-reduction is confluent.

\subsection{Simply-typed Lambda Calculus}  \label{sec:A2}

Lambda terms can be assigned expressions, called ``types'', to denote their intended input and output sets. There exists two typing paradigms: 
\`{a} la Curry, sometimes called \textbf{implicit}, and \`{a} la Church, sometimes called \textbf{explicit}. This section first recalls the basics of Curry-style approach and then briefly compares it with the Church-style.

\begin{definition}[Simple types] Given a set $A$ of \textbf{atomic types}, the set of \textbf{types} $T$ is inductively defined as follows:
\begin{center}
$
\begin{array}{rcll}
\alpha \in A & \Longrightarrow & \alpha \in T &\\
\alpha, \beta \in A &  \Longrightarrow & (\alpha \rightarrow \beta) \in T & \text{(\textbf{function types})}
\end{array} 
$
\end{center}
\end{definition}
An atomic type is intended to denote some particular set. A function type $(\alpha \rightarrow \beta)$ is intended to denote some set of functions from $\alpha$ to $\beta$, i.e. the functions that take as the argument a member of the set denoted by $\alpha$ and return as an output a member of the set denoted by $\beta$.


\begin{remark}[Parenthesis convention] A complex functional type $(\alpha_1 \rightarrow (\alpha_2 \rightarrow \dots \rightarrow (\alpha_{n-1} \rightarrow \alpha_n)  \dots ))$ is abbreviated as $\alpha_1 \rightarrow \alpha_2 \rightarrow \dots \rightarrow \alpha_n$ (i.e. parentheses are associated to the right):
\begin{align*}
(\alpha_1 \rightarrow (\alpha_2 \rightarrow \dots \rightarrow (\alpha_{n-1} \rightarrow \alpha_n)  \dots )) & \seq \alpha_1 \rightarrow \alpha_2 \rightarrow \dots \rightarrow \alpha_n
\end{align*}
\end{remark}


\begin{definition}[$\lambda\!\rightarrow$-Curry]
\begin{enumerate}
\item A \textbf{statement} is of the form $M:\sigma$ with $M \in \Lambda$ and $\sigma \in T$. The type $\sigma$ is the \textbf{predicate} and the term $M$ is the \textbf{subject} of the statement.
\item A \textbf{declaration} is a statement with a variable as a subject.
\item A \textbf{basis} is a set of declarations with distinct variables as subjects.
\end{enumerate}
\end{definition}


\begin{definition}[Derivation rules in $\lambda\!\rightarrow$-Curry] A statement $M:\sigma$ is \textbf{derivable} from a basis $\Gamma$, denoted $\Gamma \vdash_{\lambda\rightarrow \text{-Curry}} M: \sigma$ , $\Gamma \vdash_{\lambda\rightarrow} M: \sigma$ or simply $\Gamma \vdash M: \sigma$, if $\Gamma \vdash M: \sigma$ can be produced by the following rules:
 
\begin{prooftree}
\AXC{} \RightLabel{axiom}
\UIC{$\Gamma, x: \alpha \vdash x: \alpha $}
\end{prooftree} 
 
 \begin{prooftree}
\AXC{$\Gamma \vdash M: \alpha \rightarrow \beta$}
\AXC{$\Gamma \vdash N: \alpha$} \RightLabel{app}
\BIC{$\Gamma \vdash MN: \beta$}
\end{prooftree}

\begin{prooftree}
\AXC{$\Gamma, x: \alpha \vdash M:\beta$} \RightLabel{abs}
\UIC{$\Gamma \vdash \lambda x.M: \alpha \rightarrow \beta$}
\end{prooftree}

\end{definition}


%\begin{definition}
%
%\medskip \noindent \textbf{Compatibility rules}:
%
%\begin{prooftree}
%\AXC{} \RightLabel{if $x$ is a variable}
%\UIC{$x \evalto x$}
%\end{prooftree}
%
%\begin{prooftree}
%\AXC{} \RightLabel{if $c$ is a constant}
%\UIC{$c \evalto c$}
%\end{prooftree}
%
%\begin{prooftree}
%\AXC{$M \evalto N$}
%\UIC{$\lambda x. M \evalto \lambda x.N $}
%\end{prooftree}
%
%\begin{prooftree}
%\AXC{$M \evalto \lambda x. K$}
%\AXC{$Q \evalto N$}
%\AXC{$K[x:=N] \evalto O$}
%\TIC{$ MQ \evalto O$}
%\end{prooftree}
%
%\begin{prooftree}
%\AXC{$M \evalto x$}
%\AXC{$Q \evalto N$}
%\BIC{$ MQ \evalto xN$}
%\end{prooftree}
%
%\begin{prooftree}
%\AXC{$M \evalto OK$}
%\AXC{$Q \evalto N$}
%\BIC{$ MQ \evalto OKN$}
%\end{prooftree}
%\end{definition}

\begin{lemma}[Substitution lemma for $\lambda\!\rightarrow$-Curry] \label{lem:substitution-Curry} \
\begin{enumerate}
\item If $\Gamma \vdash M:\sigma$, then $\Gamma[\alpha :=\tau] \vdash M:\sigma[\alpha :=\tau]$.
\item Suppose $\Gamma, x: \sigma \vdash M: \tau$ and $\Gamma \vdash N: \sigma$. Then $\Gamma \vdash M[x:=N]: \tau$.
\end{enumerate}
\end{lemma}
\begin{proof} 
\begin{enumerate}
\item The proof is by induction on the derivation of $M:\sigma$.
\item The proof is by induction on the generation of $\Gamma, x: \sigma \vdash M:\tau$
\end{enumerate}
\end{proof}

The following theorem states that the set of terms having a certain type is closed under reduction:
\begin{theorem}[Subject reduction theorem for $\lambda\!\rightarrow$-Curry] Suppose $M \bred N$. Then 
\begin{center}
$
\begin{array}{rcl}
\Gamma \vdash M: \sigma &\Longrightarrow & \Gamma \vdash N: \sigma
\end{array}
$
\end{center}
\end{theorem}
\begin{proof} See~\cite[p.41]{Barendregt:1992:Lambda-Calculi-with-Types}.
\end{proof}

While in Curry's approach each term is assigned a type after the term has been built, in Church's approach, the type of a term is integrated in the term itself. For example, the term $\lambda x.x$ can be assigned a type according to the Curry and Church styles respectively as follows:
\begin{align*}
& \vdash_{Curry} \lambda x.x: (\sigma \rightarrow \sigma) \\
& \vdash_{Church} \lambda x^\sigma.x: (\sigma \rightarrow \sigma) 
\end{align*}
The term $\lambda x^\sigma.x$ itself is annotated in a Church system by $\sigma$. This means that $\lambda x^\sigma.x$ takes the argument $x$ from the particular set denoted by $\sigma$. In contrast, a Curry system allows each term to have a polymorphic type. For example, the term $\lambda x.x:  (\sigma \rightarrow \sigma)$ denotes the operation of doing nothing regardless how $\sigma$ is instantiated: it can stand, for example, for integers or for booleans. 

\begin{definition}[$T$-annotated $\lambda$-terms] Let $V$ be a set of variables, $T$ be a set of types. The set $\Lambda_T$ of $T$\textbf{-annotated} $\lambda$\textbf{-terms} is defined as follows:
\begin{center}
$
\begin{array}{rcl}
x \in V & \Longrightarrow & x \in \Lambda_T \\
M, N \in  \Lambda_T& \Longrightarrow & MN \in \Lambda_T \\
x \in V, M \in \Lambda_T, \sigma \in T & \Longrightarrow & \lambda x^\sigma.M \in \Lambda_T
\end{array}
$
\end{center}
\end{definition}

The typed lambda calculus  \`{a} la Church is defined similarly to the typed lambda calculus  \`{a} la Curry: an important difference is in the derivation rule corresponding to the abstraction: the abstracted variable is explicitly annotated with a type in the Church-style system.
The explicit annotation of types in Church-style system makes it possible to decide whether a term has a certain type. This is an undecidable question for some Curry systems. On the other hand, a Curry-style system has more power and more flexibility than a Church-style system. For example, the easiest way to answer the question whether an untyped term $M$ has any typed analogues is to re-state the question in Curry's notation. Furthermore, Curry-style systems can be generalized in ways Church-style systems cannot. 

Terms \`{a} la Church can be easily mapped into terms \`{a} la Curry. This is done simply by ``erasing'' all type annotations within the term \`{a} la Church:
\begin{definition} $| \cdot |: \Lambda_T \rightarrow \Lambda$ is defined as follows:
\begin{center}
$
\begin{array}{rcl}
|x|& \defeq & x \\
|MN|& \defeq & |M||N| \\
|\lambda x^{\sigma}.M|& \defeq & \lambda x.|M|
\end{array}
$
\end{center}
\end{definition}
The following proposition states that terms in the Church version project to terms in the Curry version of $\lambda\!\rightarrow$; and that terms in the Curry style can be ``lifted'' to terms in the Church style:
\begin{proposition}  
\ 
\begin{enumerate}
\item Let $M \in \Lambda_T$. Then
\begin{align*}
\Gamma \vdash_{Church} M:\sigma  \ \ \Longrightarrow \ \ & \Gamma \vdash_{Curry} |M|:\sigma
\end{align*}
\item Let $N\in \Lambda$. Then
\begin{align*}
\Gamma \vdash_{Curry} N:\sigma  \ \  \Longrightarrow \ \ & \text{exists } M  \in \Lambda_T  \\
& \text{such that }  \Gamma \vdash_{Church}  M:\sigma \text{ and } |M| \congr N
\end{align*}
\end{enumerate}
\end{proposition}
\begin{proof} Both (1) and (2) are proved by induction on the given derivation.
\end{proof}
See~\cite{Barendregt:1992:Lambda-Calculi-with-Types}  and~\cite{HindleySeldin:2008:Lambda-Calculus-and-Combinators-an-Introduction} for profound introductions to the two typing styles and their detailed comparisons.


\subsection{Continuation} \label{subsec:Continuation}

Montague's program requires the meaning of an expression to be computed compositionally from meanings of its lexical constituents. Dynamics of natural language makes, however, this task non-trivial. A similar kind of problem occurred in the mathematical semantics of programming languages, particularly in formalizing full jumps (``goto'' statements) in a compositional way. To solve this problem, the method of continuations was introduced in~\cite{StracheyWadsworth:1974:Continuations:-A-Mathematical-Semantics-for-Handling-Full-Jumps} to extend the mathematical semantics of programming languages~\cite{ScottStrachey:1971:Toward-a-Mathematical-Semantics-for-Computer-Languages} with a general formalized notion of control of full jumps. According to this method, if the evaluation of a program is in a state $s$, and the following command is $c$, continuation represents the state transition which would be produced up to the end of the program (i.e. the rest of the computation) after performing the state transformation specified by $c$. Every function of a program written in continuation-passing style (CPS) is given as an argument to the continuation of the program with respect to this function. 

\subsubsection{CPS Transformation}

A program can be transformed from direct style to continuation-passing style (CPS).
A possible way of CPS-transforming terms of typed lambda calculus was introduced in~\cite{Plotkin:1975:Call-by-Name-Call-by-Value-and-the-lambda-Calculus}. Plotkin's call-by-value transformation is shown in Definition~\ref{def:CPS-transformation}: 
%Moreover, they showed that for typed $\lambda$-calculus the original terms can be recovered from corresponding CPS-terms.
%~\cite{MeyerWand:1985:Continuation-Semantics-in-Typed-Lambda-Calculi-Summary}
\begin{definition}[CPS-transformation]\label{def:CPS-transformation} Let $o$ be a distinguished type. For each term $t$ of type $\rho$, it is possible to construct its CPS-transformation $\overline{t}$ of type $\overline{\rho}$ in the following way
%
\begin{align*}
\overline{x} & = \lambda \phi^{\rho \rightarrow o}. \phi x^{\rho}  \\
\overline{\lambda x^{\alpha}.N^{\beta}} & = \lambda \phi^{(\alpha \rightarrow \beta)' \rightarrow o}. \phi  (\lambda x^{\alpha'}. \overline{N}^{\overline{\beta \rightarrow o}}) \\
\overline{M^{\alpha \rightarrow \beta}N^\alpha} & = \lambda \phi^{\beta' \rightarrow o}. \overline{M}^{\overline{\alpha \rightarrow \beta} }(\lambda m^{(\alpha \rightarrow \beta)'}. \overline{N}^{\overline{\alpha}}(\lambda n^{\alpha'}.mn \phi)) 
\end{align*}
%
where $\phi$ of type $(\rho' \rightarrow o)$ is a \textbf{continuation} of $t$.

Each type  $\overline{\rho}$ is defined as $(\rho' \rightarrow o) \rightarrow o$, where $\rho'$ is as follows
%
\begin{align*}
\rho' = \left\{
\begin{array}{rl} 
\rho & \text{if } \rho \text{ is basic}\\
{\alpha'} \rightarrow ({\beta}' \rightarrow o) \rightarrow o & \text{if } \rho = \alpha \rightarrow \beta \end{array} \right.
\end{align*}
\end{definition}

For a better understanding of CPS-transformation, it is important to keep the type assignments in mind. That is why terms in Definition~\ref{def:CPS-transformation} are presented together with their types as superscripts. Equations~\eqref{eqs:cps} repeat equations in Definition~\ref{def:CPS-transformation} with types omitted for better readability:
%
\begin{subequations}
\begin{align}
\overline{x} & = \lambda \phi. \phi x \label{eq:var}  \\
\overline{\lambda x.N} & = \lambda \phi. \phi (\lambda x. \overline{N}) \label{eq:abst}\\
\overline{MN} & = \lambda \phi. \overline{M}(\lambda m. \overline{N}(\lambda n.mn \phi)) \label{eq:app}
\end{align} \label{eqs:cps}
\end{subequations}

It is possible to make a CPS-transformation of an application in a way that the argument is evaluated before the function, \eqref{eq:app-nws} shows the corresponding transformation rule:
%
\begin{align}
\overline{MN} & = \lambda \phi. \overline{N}(\lambda n. \overline{M}(\lambda m.mn \phi)) \label{eq:app-nws}
\end{align}

The transformation presented above imposes call-by-value evaluation strategy in the resulting term. According to call-by-value strategy, the argument expression is evaluated before being passed to the function.\footnote{Call-by-value is usually contrasted with call-by-name evaluation strategy. According to call-by-name, the argument of a function is not evaluated before the evaluation of the function. Call-by-value evaluation is chosen here as it is closer to Montague's technique of type raising, as discussed in Section~\ref{sec:ContinuationInNLSemantics}.}

\begin{example} \label{ex:CPS1} Consider a $\lambda$-term $t$ computing  $\sqrt{x^3}$:
% 
\begin{align*}
t = \lambda x. \mathsf{sqrt} (\mathsf{cube} \ x)
\end{align*}
%
where $\mathsf{sqrt}$ is a function computing the square root of its argument and $\mathsf{cube}$ is a function computing the cubic power of its argument. Term $t$ can be CPS-transformed (to call-by-value evaluation style) in the following way:
%
 { \small
 \begin{align}
&\overline{\lambda x. \mathsf{sqrt} (\mathsf{cube} \ x )} \notag \\
 = \ &\lambda \phi. \phi (\lambda x. \overline{\mathsf{sqrt} (\mathsf{cube} \ x})) \tag{by~\eqref{eq:abst}} \\
  = \  & \lambda \phi. \phi (\lambda x. (\lambda \phi'. \overline{\mathsf{sqrt}} (\lambda m.\overline{\mathsf{cube} \ x} (\lambda n. m n \phi') ) ) )\tag{by~\eqref{eq:app}} \\
  = \ & \lambda \phi. \phi (\lambda x. (\lambda \phi'. \overline{\mathsf{sqrt}} (\lambda m. ( \lambda \phi''. \overline{\mathsf{cube}} (\lambda m'. \overline{x} (\lambda n'. m'n' \phi'') ) )   (\lambda n. m n \phi') ) ) )\tag{by~\eqref{eq:app}} \\
  \bred & \lambda \phi. \phi (\lambda x. (\lambda \phi'. \overline{\mathsf{sqrt}} (\lambda m. ( \overline{\mathsf{cube}} (\lambda m'. \overline{x} (\lambda n'. m'n'  (\lambda n. m n \phi') ) ) )   ) ) )\notag \\
  = \ & \lambda \phi. \phi (\lambda x. (\lambda \phi'. \overline{\mathsf{sqrt}} (\lambda m. ( \overline{\mathsf{cube}} (\lambda m'. (\lambda \phi''.\phi'' x) (\lambda n'. m'n'  (\lambda n. m n \phi') ) ) )   ) ) )\tag{by~\eqref{eq:var}} \\
  \bred  &\lambda \phi. \phi (\lambda x. (\lambda \phi'. \overline{\mathsf{sqrt}} (\lambda m. ( \overline{\mathsf{cube}} (\lambda m'. m' x  (\lambda n. m n \phi') ) )  ) ) ) \label{eq:ex1nf}
 \end{align} }
 
The CPS versions of $\mathsf{sqrt}$ and $\mathsf{cube}$ are shown in Equations~\eqref{sqrt} and~\eqref{cube} respectively:
%
\begin{subequations}
\begin{align}
\overline{\mathsf{sqrt}} & = \lambda \phi. \phi ( \lambda x \phi'. \phi ' (\mathsf{sqrt}  x))  \label{sqrt} \\
\overline{\mathsf{cube}} & = \lambda \phi. \phi  (  \lambda x \phi'. \phi ' (\mathsf{cube}  x)) \label{cube}
\end{align} 
\end{subequations}

After substituting $\overline{\mathsf{sqrt}}$ and $\overline{\mathsf{cube}}$ in~\eqref{eq:ex1nf} for~\eqref{sqrt} and~\eqref{cube} and $\beta$-reducing,  the normalized CPS-transformed term shown below is obtained:
%
\begin{align*}  
\overline{t} =\lambda \phi. \phi ( \lambda x \phi'. \phi' (\mathsf{sqrt} (\mathsf{cube} x) ) )
\end{align*} 
\end{example} 

Definition~\ref{def:CPS-transformation} transforms a program in direct style into an equivalent program in CPS. For example, $t$ and $\overline{t}$ return the same result for the same argument $4$, as Equations~\eqref{eq:t4} and~\eqref{eq:CPSt4} in the next example show: 
\begin{example}
\begin{align} 
t(4) & = (\lambda x. \mathsf{sqrt} (\mathsf{cube} \ x))(4) \notag\\
& \bconv \mathsf{sqrt} (\mathsf{cube} \ 4) \notag \\
& =  \mathsf{sqrt} \ 64 \notag \\
& = 8 \label{eq:t4}
\end{align}
%
\begin{align}
\lambda \phi. \overline{t} (\lambda \phi'. \phi' 4 \phi) = \ & \lambda \phi. (\lambda \phi. \phi ( \lambda x \phi'. \phi' (\mathsf{sqrt} (\mathsf{cube} x) ) )) (\lambda \phi'. \phi' 4 \phi) \notag \\
\bconv \ & \lambda \phi.  (\lambda \phi'. \phi' 4 \phi)  ( \lambda x \phi'. \phi' (\mathsf{sqrt} (\mathsf{cube} x) ) ) \notag \\
\bconv \ & \lambda \phi.   ( \lambda x \phi'. \phi' (\mathsf{sqrt} (\mathsf{cube} x) ) ) 4 \phi \notag \\
\bconv \ & \lambda \phi. \phi (\mathsf{sqrt} (\mathsf{cube} 4) )   \notag \\
= \ & \lambda \phi. \phi (\mathsf{sqrt} 64 )   \notag \\
= \ & \lambda \phi. \phi 8   \label{eq:CPSt4}  
\end{align}
\end{example}

\subsubsection{CPS Control} \label{subsubsec:CPSControl}

As demonstrated in the previous section, by simply applying Definition~\ref{def:CPS-transformation} one transforms a program into an equivalent CPS program. However, the advantage of a program written in continuation passing style is that it can be expanded with unusual expressions managing its execution, like non-local transfers of control. For example, one of the possible control expressions of a function can be to  discard all the future of the computation (i.e. its continuation) and return the error as a result of the whole program, in case an error happens within this function. Examples~\ref{ex:divbyzero}--~\ref{ex:CPScontrolfinal} show how it can be done in the case of division by zero.
%
\begin{example} \label{ex:divbyzero}
 Consider a $\lambda$-term $t$ computing  $\sqrt{ \dfrac{x}{y}}$: 
 %
\begin{align}
t = \lambda xy. \mathsf{sqrt} (\mathsf{div} \ x \ y) \label{eq:tstat}
\end{align}
%
During the evaluation of $t$, if the execution of $\mathsf{div}$ leads to an error, the error message is passed to the function $\mathsf{sqrt}$ as the argument. Assuming that $\mathsf{sqrt}$ is defined simply as a square root of its argument, this leads to $\mathsf{sqrt}$ being applied to a value for which it is not defined.  For example, this happens if the second of the arguments given to $t$ is zero:
%
\begin{align}
t(9)(0) & =  (\lambda xy. \mathsf{sqrt} (\mathsf{div} \ x \ y))(9)(0)  \notag \\
& \bconv   (\lambda y. \mathsf{sqrt} (\mathsf{div} \ 9 \ y) ) (0)  \notag \\
& \bconv \mathsf{sqrt} (\mathsf{div} \ 9 \ 0)  \notag \\
& = \mathsf{sqrt} (\mathtt{error} ) \label{sqrterrow}\\
& =  \ ??? \notag
\end{align}

\end{example}

This situation could be handled in the original direct program by adding at the beginning of $\mathsf{sqrt}$ a conditional expression that checks whether the argument is an $\mathtt{error}$ (and returning an  $\mathtt{error}$ in this case) or a number (and calculating the square root of the number in this case). However, this means that, following this style, such a condition would have to be added to all other possible functions of the program. In contrast, the CPS technique allows to provide control to the division function itself in a way that it terminates the whole program when an error occurs during its execution.

It is possible to CPS-transform $t$ according to  Definition~\ref{def:CPS-transformation} and the next example demonstrates this. The transformations of functions $\mathsf{sqrt} $ and $\mathsf{div} $, shown below, are necessary for intermediary steps:
%
\begin{subequations}
\begin{align}
\overline{\mathsf{sqrt}} & = \lambda \phi . \phi ( \lambda x \phi'  . \phi'(  \mathsf{sqrt}  x))  \label{sqrt-3} \\
\overline{\mathsf{div}} & = \lambda \phi . \phi  (  \lambda x \phi' . \phi' ( \lambda y \phi''. \phi '' ( \mathsf{div} \ x \ y))) \label{div-3} 
\end{align} \label{eq:sqrtdiv-3}
\end{subequations}

\begin{example}
 %After substituting $ \overline{\mathsf{sqrt}}$ and $ \overline{\mathsf{div}}$ in~\eqref{eq:exdivbyzero0} with their corresponding terms shown in~\eqref{eq:sqrtdiv-3} and $\beta$-reducing, the normalized CPS-transformed $t$, shown in~\eqref{eq:exdivbyzero}, is obtained:
\begin{small}
\begin{align}
& \overline{\lambda xy. \mathsf{sqrt} (\mathsf{div} \ x \ y)}  \notag \\
= \ & \lambda \phi. \phi ( \lambda x. (\overline{\lambda y. \mathsf{sqrt} (\mathsf{div} \ x \ y)}  ))) \tag{by~\eqref{eq:abst}} \\
= \ & \lambda \phi. \phi ( \lambda x. ( \lambda \phi'. \phi' ( \lambda y. \overline{ \mathsf{sqrt} (\mathsf{div} \ x \ y)}  ))) \tag{by~\eqref{eq:abst}} \\
= \ & \lambda \phi. \phi ( \lambda x. ( \lambda \phi'. \phi' ( \lambda y. ( \lambda \phi''. \overline{ \mathsf{sqrt}} ( \lambda m. \overline{\mathsf{div} \ x \ y}  (\lambda n. mn \phi'')  ))) ))\tag{by~\eqref{eq:app}} \\
 = & \dots  \tag{by~\eqref{eq:app}} \\
 \bred \ &  \lambda \phi. \phi ( \lambda x. ( \lambda \phi'. \phi' ( \lambda y. ( \lambda \phi''. \overline{\mathsf{sqrt}}(\lambda m. \overline{\mathsf{div}}  (\lambda m''. m'' x ( \lambda m'. m' y (\lambda n. mn \phi ''))) ) )) ) ) \label{eq:exdivbyzero0}\\
= & \dots \tag{by~\eqref{sqrt-3} and~\eqref{div-3}} \\
 \bred \ &  \lambda \phi. \phi ( \lambda x \phi'. \phi' ( \lambda y \phi ''. \phi''  (\mathsf{sqrt} ( \mathsf{div} \ x\ y ))))\label{eq:exdivbyzero} 
\end{align}
\end{small}
\end{example}

%\overline{t} & =  \lambda \phi. \phi (\lambda x \phi'. \phi' ( \lambda y \phi''. \phi''(\mathsf{sqrt} (\mathsf{div} \ x \ y) ))) \label{eq:exdivbyzeroCPS}

Since normalized term $\overline{t}$ in~\eqref{eq:exdivbyzero} is obtained from $t$ in~\eqref{eq:tstat} only by applying CPS-transformation definitions, it is equivalent to $t$  and behaves analogously when an error happens inside the division function. To demonstrate it, $\overline{t}$ is ``fed'' in the next example with the same (continuized) arguments 9 and 0 as $t$ in Example~\ref{ex:divbyzero}. The fact that the result is equally undesirable can be seen by comparing Equations~\eqref{sqrterrow} and~\eqref{sqrterrowCPS}:

\begin{example}
\begin{align}
& \lambda \phi'''. \overline{t} (\lambda \phi''''. \phi'''' 9 (\lambda \phi'''''. \phi''''' 0 \phi''')) \notag \\
 = \ & \lambda \phi'''. (\lambda \phi. \phi ( \lambda x \phi'. \phi' ( \lambda y \phi ''. \phi''  (\mathsf{sqrt} ( \mathsf{div} \ x\ y ))))) (\lambda \phi''''. \phi'''' 9 (\lambda \phi'''''. \phi''''' 0 \phi''')) \notag \\
  \bconv  \ & \lambda \phi'''.  (\lambda \phi''''. \phi'''' 9 (\lambda \phi'''''. \phi''''' 0 \phi''')) ( \lambda x \phi'. \phi' ( \lambda y \phi ''. \phi''  (\mathsf{sqrt} ( \mathsf{div} \ x\ y ))))\notag \\  
    \bconv  \ & \lambda \phi'''.  ( \lambda x \phi'. \phi' ( \lambda y \phi ''. \phi''  (\mathsf{sqrt} ( \mathsf{div} \ x\ y )))) 9 (\lambda \phi'''''. \phi''''' 0 \phi''')\notag \\
     \bred  \ & \lambda \phi'''.   (\lambda \phi'''''. \phi''''' 0 \phi''') ( \lambda y \phi ''. \phi''  (\mathsf{sqrt} ( \mathsf{div} \ 9\ y ))) \notag \\ 
 \bconv  \ & \lambda \phi'''.    ( \lambda y \phi ''. \phi''  (\mathsf{sqrt} ( \mathsf{div} \ 9\ y )))  0 \phi'''\notag \\
\bred \ & \lambda \phi'''. \phi''' \mathsf{sqrt} (\mathsf{div} \ 9 \ 0)  \notag \\
 =  \ & \lambda \phi'''. \phi''' (\mathsf{sqrt} (\mathtt{error} )) \label{sqrterrowCPS}\\
 =   \ & ??? \notag
\end{align}
\end{example}

However, an advantage of the CPS-written program~\eqref{eq:exdivbyzero} is that it can be extended with an unusual control due to the fact that it, as well as its functions, have continuations as arguments. Particularly, division can be refined in a way so that when the divisor is zero, it disregards the continuation and immediately returns an error message as a result of the whole program; and otherwise provides the result of the division to its continuation. 
Thus, CPS-transformation of $\mathsf{div}$ can be done not directly by Equation~\eqref{div-3}, but by using its modified version~\eqref{div-4} with a control on whether $y$ is equal to zero or not:
%
\begin{align}
\overline{\mathsf{div}} = \ & \lambda \phi . \phi ( \lambda x \phi'. \phi' ( \lambda y .  \mathsf{if} \ (y =0)  \ \mathsf{then} \ \mathtt{error}  \ \mathsf{else}  \  (\lambda \phi''. \phi'' (\mathsf{div} \ x \ y) )) ) \label{div-4}
\end{align} 

Then, using Equation~\eqref{div-4} in~\eqref{eq:exdivbyzero0}, and following Definition~\ref{def:CPS-transformation} in the rest of the transformation, a new CPS program that has a special treatment for division by zero is obtained, as shown below: 
%
\begin{example}
%\begin{footnotesize}
\begin{align}
& \overline{\lambda xy. \mathsf{sqrt} (\mathsf{div} \ x \ y)}  \notag \\
 = & \dots  \tag{by~\eqref{eq:abst} and~\eqref{eq:app}} \\
 \bred \ &  \lambda \phi. \phi ( \lambda x. ( \lambda \phi'. \phi' ( \lambda y. ( \lambda \phi''. \overline{\mathsf{sqrt}}(\lambda m. \overline{\mathsf{div}}  (\lambda m''. m'' x ( \lambda m'. m' y (\lambda n. mn \phi ''))) ) )) ) )\notag\\
= &  \lambda \phi. \phi ( \lambda x. ( \lambda \phi'. \phi' ( \lambda y. ( \lambda \phi''. ( \lambda \phi . \phi ( \lambda x \phi'  . \phi'(  \mathsf{sqrt}  x)) ) \tag{by~\eqref{sqrt-3}} \\
& \phantom{ \lambda \phi. \phi ( \lambda x. ( \lambda \phi'. \phi' ( \lambda y. ( \lambda \phi''. } (\lambda m. \overline{\mathsf{div}}  (\lambda m''. m'' x ( \lambda m'. m' y (\lambda n. mn \phi ''))) ) )) ) ) \notag \\
\bred \ &   \lambda \phi. \phi ( \lambda x. ( \lambda \phi'. \phi' ( \lambda y. ( \lambda \phi''.  \overline{\mathsf{div}}  (\lambda m''. m'' x ( \lambda m'. m' y (\lambda n.  \phi''(  \mathsf{sqrt}  n)))) ) ))  ) \notag \\
= \ & \lambda \phi. \phi ( \lambda x. ( \lambda \phi'. \phi' ( \lambda y. ( \lambda \phi''.  (\lambda \phi . \phi ( \lambda x' \phi'. \phi' ( \lambda y'.  \mathsf{if} \ (y' =0)  \ \mathsf{then} \ \mathtt{error}  \notag  \\
& \phantom{ \lambda \phi. \phi ( \lambda x. ( \lambda \phi'. \phi' ( \lambda y. ( \lambda \phi''.  (\lambda \phi . \phi ( \lambda x' \phi'. \phi' ( \lambda y'.} 
\mathsf{else}  \  (\lambda \phi'''. \phi''' (\mathsf{div} \ x' \ y') )) ) ) \notag \\
& \phantom{\lambda \phi. \phi ( \lambda x. ( \lambda \phi'. \phi' ( \lambda y. ( \lambda \phi''.}
(\lambda m''. m'' x ( \lambda m'. m' y (\lambda n.  \phi''(  \mathsf{sqrt}  n)))) ) ))  )  \tag{by~\eqref{div-4}} \\
\bconv \ & \lambda \phi. \phi ( \lambda x. ( \lambda \phi'. \phi' ( \lambda y. ( \lambda \phi''. (\lambda m''. m'' x ( \lambda m'. m' y (\lambda n.  \phi''(  \mathsf{sqrt}  n)))) \notag \\
& \phantom{\lambda \phi. \phi ( \lambda x. ( \lambda \phi'. \phi' ( \lambda y. ( \lambda \phi''. } 
( \lambda x' \phi'. \phi' ( \lambda y'.  \mathsf{if} \ (y' =0)  \ \mathsf{then} \ \mathtt{error}  \notag\\
& \phantom{\lambda \phi. \phi ( \lambda x. ( \lambda \phi'. \phi' ( \lambda y. ( \lambda \phi''.( \lambda x' \phi'. \phi' ( \lambda y'.   } 
 \mathsf{else}  \  (\lambda \phi'''. \phi''' (\mathsf{div} \ x' \ y') )) ) ) )))\notag \\
\bconv \ & \lambda \phi. \phi ( \lambda x. ( \lambda \phi'. \phi' ( \lambda y. ( \lambda \phi''. ( \lambda x' \phi'. \phi' ( \lambda y'.  \mathsf{if} \ (y' = 0)  \ \mathsf{then} \ \mathtt{error}  \notag \\
&  \phantom{\lambda \phi. \phi ( \lambda x. ( \lambda \phi'. \phi' ( \lambda y. ( \lambda \phi''. ( \lambda x' \phi'. \phi' ( \lambda y'. }
 \mathsf{else}  \  (\lambda \phi'''. \phi''' (\mathsf{div} \ x' \ y') )) ) \notag \\
& \phantom{\lambda \phi. \phi ( \lambda x. ( \lambda \phi'. \phi' ( \lambda y. ( \lambda \phi''. ( }
 x ( \lambda m'. m' y (\lambda n.  \phi''(  \mathsf{sqrt}  n))) ) )))\notag \\
 \bred \ & \lambda \phi. \phi ( \lambda x. ( \lambda \phi'. \phi' ( \lambda y. ( \lambda \phi''.   ( \lambda m'. m' y (\lambda n.  \phi''(  \mathsf{sqrt}  n))) \notag \\
 & \phantom{\lambda \phi. \phi ( \lambda x. ( \lambda \phi'. \phi' ( \lambda y. ( \lambda \phi''.  }
 ( \lambda y'.  \mathsf{if} \ (y' = 0)  \ \mathsf{then} \ \mathtt{error}  \notag \\
& \phantom{\lambda \phi. \phi ( \lambda x. ( \lambda \phi'. \phi' ( \lambda y. ( \lambda \phi''.   ( \lambda y'.  }  
  \mathsf{else}  \  (\lambda \phi'''. \phi''' (\mathsf{div} \ x \ y') )) ) )))\notag \\
 \bconv \ & \lambda \phi. \phi ( \lambda x. ( \lambda \phi'. \phi' ( \lambda y. ( \lambda \phi''.    ( \lambda y'.  \mathsf{if} \ (y' = 0)  \ \mathsf{then} \ \mathtt{error}  \notag \\
 & \phantom{\lambda \phi. \phi ( \lambda x. ( \lambda \phi'. \phi' ( \lambda y. ( \lambda \phi''.    ( \lambda y'. }
  \mathsf{else}  \  (\lambda \phi'''. \phi''' (\mathsf{div} \ x \ y') )) y (\lambda n.  \phi''(  \mathsf{sqrt}  n)) ) )))\notag \\
  \bconv \ & \lambda \phi. \phi ( \lambda x. ( \lambda \phi'. \phi' ( \lambda y. ( \lambda \phi''.    \mathsf{if} \ (y = 0)  \ \mathsf{then} \ \mathtt{error}  \notag \\
  & \phantom{\lambda \phi. \phi ( \lambda x. ( \lambda \phi'. \phi' ( \lambda y. ( \lambda \phi''.  }
   \mathsf{else}  \  (\lambda \phi'''. \phi''' (\mathsf{div} \ x \ y) )  (\lambda n.  \phi''(  \mathsf{sqrt}  n)) ) ))) \label{eq:exdivbyzeroCPScontrol} 
\end{align}
%\end{footnotesize}
\end{example}

%
Term~\eqref{eq:exdivbyzeroCPScontrol}, abbreviated below as $\overline{t}^{control}$, computes a square root of a result of a division, as programs $t$ and $\overline{t}$ do. However,  $\overline{t}^{control}$ additionally has the control over division by zero. Particularly, when the divisor is equal to zero, $\overline{t}^{control}$, unlike $t$ and $\overline{t}$, terminates with an error. This is demonstrated in Example~\ref{ex:CPScontrolfinal}:
%
\begin{example} \label{ex:CPScontrolfinal}
%\begin{footnotesize}
\begin{align}
& \lambda \phi. \overline{t}^{control} (\lambda \phi'. \phi' 9 (\lambda \phi''. \phi'' 0 \phi)) \notag \\
= \ & \lambda \phi. (\lambda \phi. \phi ( \lambda x. ( \lambda \phi'. \phi' ( \lambda y. ( \lambda \phi''.    \mathsf{if} \ (y = 0)  \ \mathsf{then} \ \mathtt{error}  \notag \\
& \phantom{\lambda \phi. (\lambda \phi. \phi ( \lambda x. ( \lambda \phi'. \phi' ( \lambda y. ( \lambda \phi''.} 
 \mathsf{else}  \  (\lambda \phi'''. \phi''' (\mathsf{div} \ x \ y) )  (\lambda n.  \phi''(  \mathsf{sqrt}  n)) ) )))) \notag \\
 & \phantom{\lambda \phi.} (\lambda \phi'. \phi' 9 (\lambda \phi''. \phi'' 0 \phi)) \notag \\
 \bconv \ & \lambda \phi.  (\lambda \phi'. \phi' 9 (\lambda \phi''. \phi'' 0 \phi)) \notag \\
 & \phantom{\lambda \phi.} ( \lambda x. ( \lambda \phi'. \phi' ( \lambda y. ( \lambda \phi''.    \mathsf{if} \ (y = 0)  \ \mathsf{then} \ \mathtt{error} \notag \\
& \phantom{\lambda \phi.  ( \lambda x. ( \lambda \phi'. \phi' ( \lambda y. ( \lambda \phi''.    } 
 \mathsf{else}  \  (\lambda \phi'''. \phi''' (\mathsf{div} \ x \ y) )  (\lambda n.  \phi''(  \mathsf{sqrt}  n)) ) ))) \notag \\
 \bconv \ & \lambda \phi.   ( \lambda x. ( \lambda \phi'. \phi' ( \lambda y. ( \lambda \phi''.    \mathsf{if} \ (y = 0)  \ \mathsf{then} \ \mathtt{error} \notag \\
 & \phantom{\lambda \phi.   ( \lambda x. ( \lambda \phi'. \phi' ( \lambda y. ( \lambda \phi''.   }
  \mathsf{else}  \  (\lambda \phi'''. \phi''' (\mathsf{div} \ x \ y) )  (\lambda n.  \phi''(  \mathsf{sqrt}  n)) ) ))) 9 (\lambda \phi''. \phi'' 0 \phi) \notag \\
 \bconv \ & \lambda \phi.   ( \lambda \phi'. \phi' ( \lambda y. ( \lambda \phi''.    \mathsf{if} \ (y = 0)  \ \mathsf{then} \ \mathtt{error} \notag \\
 & \phantom{ \lambda \phi.   ( \lambda \phi'. \phi' ( \lambda y. ( \lambda \phi''. }  \mathsf{else}  \  (\lambda \phi'''. \phi''' (\mathsf{div} \ 9 \ y) )  (\lambda n.  \phi''(  \mathsf{sqrt}  n)) ) ))  (\lambda \phi''. \phi'' 0 \phi) \notag \\
  \bconv \ & \lambda \phi.   (\lambda \phi''. \phi'' 0 \phi) ( \lambda y. ( \lambda \phi''.    \mathsf{if} \ (y = 0)  \ \mathsf{then} \ \mathtt{error} \notag \\
  & \phantom{ \lambda \phi.   (\lambda \phi''. \phi'' 0 \phi) ( \lambda y. ( \lambda \phi''.   }  \mathsf{else}  \  (\lambda \phi'''. \phi''' (\mathsf{div} \ 9 \ y) )  (\lambda n.  \phi''(  \mathsf{sqrt}  n)) ) )  \notag \\
  \bconv \ & \lambda \phi.   ( \lambda y. ( \lambda \phi''.    \mathsf{if} \ (y = 0)  \ \mathsf{then} \ \mathtt{error} \ \mathsf{else}  \  (\lambda \phi'''. \phi''' (\mathsf{div} \ 9 \ y) )  (\lambda n.  \phi''(  \mathsf{sqrt}  n)) ) ) 0 \phi \notag \\
    \bred \ & \lambda \phi. \mathsf{if} \ (0 = 0)  \ \mathsf{then} \ \mathtt{error} \ \mathsf{else}  \  (\lambda \phi'''. \phi''' (\mathsf{div} \ 9 \ 0) )  (\lambda n.  \phi(  \mathsf{sqrt}  n))  \notag \\
=  \ & \ \mathtt{error} \notag %\label{errormsg}
\end{align}
%\end{footnotesize}
\end{example}

As already mentioned above, the method of continuation was introduced to give compositional semantics to full jumps in programming languages. Next example illustrates on the example of a simple language that continuation-based interpretation of $\abort$ operation leads to desired evaluation of a program, while a direct interpretation does not:

\begin{example}
Consider a small programming language $L_1$ consisting of variables from set $\vvar$, expressions from set $\expr$, symbols $:=$ and $;$. The language has the following simple grammar, where $S$ stands for a statement:
\begin{align}
S  \defeq \ & \vvar := \expr \ | \notag \\
&  S;S \notag
\end{align}
Let $\state$ be a function mapping variables to natural numbers:
\begin{align}
\state = \ \vvar \rightarrow \mathbb{N} \notag
\end{align}
The statements $S$ in $L_1$ are interpreted with the function $I$ that maps expressions to natural numbers and statements to transformation on states:
\begin{align}
I_{\expr}: \ & \expr \rightarrow \mathbb{N} \notag \\
I_{S}: \ & S \rightarrow (\state \rightarrow \state) \notag 
\end{align}
Then, compositional semantics of this language can be defined as below, where $\xi$ is a state:
\begin{align}
I_{S}(x:=e)\xi = \ & \xi[x := I_{\expr}(e)] \notag \\
I_{S}(S_1;S_2)\xi = \ & I_S(S_2)(I_S(S_1)\xi) \notag 
\end{align}
Consider, for example, the following program written in $L_1$:
\begin{align*}
& x:= 3; \\
& x:= 5
\end{align*}
After the program is evaluated, the final state stores the value $5$ for $x$:
\begin{align}
I_S(x:=3; x:=5)\xi  = \ & I_S(x:=5)(I_S(x:=3)\xi ) \notag \\
= \ & I_S(x:=5) \xi [ x:=3 ]  \notag \\
= \ & \xi [ x:=5] \notag
\end{align}
Assume now that the language $L_1$ is extended with an $\abort$ operation, resulting in new language $L_2$: 
\begin{align}
S  \defeq \ & \vvar := \expr \ | \notag \\
& S;S \ | \notag \\
& \abort \notag
\end{align}
Intuitively, the interpretation of $\abort$ in a state $\xi$ should return the same state:
\begin{align}
I_{S}(\abort)\xi = \xi \notag
\end{align}
It is desirable that the following program, after its termination (due to $\abort$) should store $3$ as the value of $x$:
\begin{align*}
& x:= 3; \notag \\
& \abort; \notag \\
& x:= 5 \notag
\end{align*}
However, evaluating the program according to the rules defined above, 
results in $x$ having the value $5$:
\begin{align}
I_S(x:=3;\abort; x:=5)\xi  = \ & I_S(x:=5)(I_S(x:=3; \abort)\xi ) \notag \\
= \ & I_S(x:=5)(I_S(\abort)(I_S(x:=3) \xi)) \notag \\
= \ & I_S(x:=5)(I_S(\abort) \xi [ x:=3 ] )  \notag \\
= \ & I_S(x:=5) \xi [  x:=3 ] \notag \\
= \ & \xi [ x:=5] \notag
\end{align}
This can be solved by defining compositional semantics using continuations. Then interpretation $I_S$ is of the type $(S \rightarrow \state \rightarrow (\state \rightarrow \state))$ and the semantic rules are as follows:
\begin{align}
I_{S}(x:=e) \xi \phi = \ & \phi(\xi [ x:=I_{\expr}(e)]) \notag \\
I_{S}(S_1;S_2)\xi \phi = \ &  I_S(S_1)\xi(\lambda s.I_{S}(S_2)s \phi) \notag \\
I_{S}(\abort) \xi \phi = \ &\xi \notag
\end{align}
Evaluation of the program according to continuation-based interpetation rules results in $x$ being assigned the desired value $3$:
\begin{align}
I_{S}(x:=3; \abort; x:=5) \xi \phi  = \ & I_{S}(x:=3) \xi ( \lambda s. I_{S}(\abort) s ( \lambda s'. I_{S}(x:=5) \xi \phi )) \notag \\
= \ & I_{S}(x:=3) \xi (\lambda s.s) \notag \\
= \ & \lambda s.s (\xi [x:=3] )  \notag \\
= \ & \xi [x:=3] \notag
\end{align}
When $\abort$ is absent, continuation-based evaluation of the program  correctly assigns the value $5$ to $x$:
\begin{align}
I_{S}(x:=3; x:=5) \xi \phi = \ & \phi(\xi[x:=3][x:=5]) \notag \\
= \ & \phi (\xi [x:=5]) \notag
\end{align}
\end{example}

\subsubsection{Continuation Technique in Natural Language Semantics} \label{subsubsec:ContinuationInNLSemantics}

%Recent research~\cite{deGroote:2001:Type-raising-continuations-and-classical-logic,Barker:2002:Continuations-and-the-Nature-of-Quantification,Barker:2004:Continuations-in-Natural-Language,Shan:2005:Linguistic-Side-Effects,deGroote:2006:Towards-a-Montagovian-Account-of-Dynamics} shows that CPS-transform is a very useful method for formalizing natural language (discourse) semantics.

Continuation-like approaches can already be found in earlier work on formal compositional semantics of natural language. Montague's~\cite{Montague:1973:The-Proper-Treatment-of-Quantification-in-Ordinary-English} technique of type-raising can be seen as using continuation technique for quantifiers to take scope over complete sentences. In this type-raised setting, scope ambiguities (subject wide scope versus object wide scope) correspond to the change of evaluation order of the arguments. This is one of the features that continuation-style allows. Indeed, compare Montague's account of scope ambiguity shown in~\eqref{eq:MS:looooves}, with respectively equations~\eqref{eq:app} and~\eqref{eq:app-nws} of call-by-value continuation-passing-style transformation, repeated below in~\eqref{eq:apppp}, that impose different evaluation order of the arguments:
%
\begin{subequations}
\begin{align}
\I{loves}_{sws} = \ & \lambda OS. S (\lambda x. O (\lambda y. \textbf{love} x y) ) \\
\I{loves}_{ows} = \ & \lambda OS. O (\lambda y. S (\lambda x. \textbf{love} x y) )
\end{align}
\label{eq:MS:looooves}
\end{subequations}
%
\begin{subequations}
\begin{align}
\overline{MN} & = \lambda \phi. \overline{M}(\lambda m. \overline{N}(\lambda n.mn \phi)) \\
\overline{MN} & = \lambda \phi. \overline{N}(\lambda n. \overline{M}(\lambda m.mn \phi)) 
\end{align}
\label{eq:apppp}
\end{subequations}


More recent research confirms that continuation-passing technique is very useful for formalizing natural language (discourse) semantics. For example Barker~\cite{Barker:2002:Continuations-and-the-Nature-of-Quantification} investigates in more details how continuized $\lambda$-terms expressing natural language meanings provide a way of dealing with phenomena related to quantification, such as scope displacement and scope ambiguity, and give a unified account of quantificational and non-quantificational noun phrases. Additionaly, Barker~\cite{Barker:2004:Continuations-in-Natural-Language} shows that CPS can be used to deal with focus, coordination and misplaced quantifiers.\footnote{Other interesting continuation-based analyses of different natural language phenomena include~\cite{Shan:2002:A-continuation-semantics-of-interrogatives-that-accounts-for-Bakers-ambiguity,Shan:2004:Delimited-continuations-in-natural-language,BarkerShan:2008:Donkey-Anaphora-is-In-Scope-Binding}.}

%Although a continuation-passing style allows many apparently non-compositional phenomena to be handled compositionally, it requires type-raising (e.g. an atomic type $\alpha$ becomes $((\alpha \rightarrow o) \rightarrow o) $) and thus result in more complex and less natural $\lambda$-terms and types. However, due to the similarity between type-raising and the double-negation translation (e.g. an atomic proposition $\P1$ becomes $((\P1 \rightarrow \bot) \rightarrow \bot)$), it is possible to obtain more concise and more intuitive lambda terms and types in natural languages semantics by using extended lambda calculi, such as the $\lambda \mu$-calculus~\cite{Parigot:1992:Lambda-My-Calculus:-An-Algorithmic-Interpretation-of-Classical-Natural-Deduction}, originally developed to exploit the double negation translation and provide a Curry-Howard isomorphism for classical logic. This was done in~\cite{deGroote:2001:Type-raising-continuations-and-classical-logic}, where ambiguities in the scope of natural language quantifiers were nicely related to the non-confluence of the $\lambda \mu$-calculus (with $\beta$-, $\mu$- and $\mu'$-reductions), which corresponds to the non-confluence of cut-elimination in classical logic.

Next section proposes an approach that uses continuations in order to implement the influential pragmatic ideas of Stalnaker~\cite{Stalnaker:1978:Assertion} in Montague's settings. This is done by providing two additional arguments to each $\lambda$-term interpreting a natural language proposition. One argument stands for the continuation of the term. Another argument represents a (previously computed) context. Moreover, within the body of the $\lambda$-term, the continuation is applied to the (possibly updated) context (therefore, the continuation is dependent on the context). This is inspired by Stalnaker's idea that each assertion is made in context, is dependent upon the context and updates the context.\footnote{\cite{Heim:1982:The-Semantics-of-Definite-and-Indefinite-Noun-Phrases} and \cite{Kamp:1981:A-Theory-of-Truth-and-Semantic-Representation} also elaborate the work of Stalnaker~\cite{Stalnaker:1978:Assertion}. They do it, however, in a fundamentally different, non-compositional, way. }
\section{A continuation-based account of dynamics} \label{sec:cont_based_dyn}

%This section reviews a continuation-based approach introduced in~\cite{deGroote:2006:Towards-a-Montagovian-Account-of-Dynamics} and shows how it handles cross-sentential and donkey anaphora. The approach uses standard tools of mathematical logic, such as simply-typed lambda calculus, and is, therefore, loyal to the Montague's program. 
 
This chapter provides an intuition for the continuation-based framework based on dynamic logic formally defined in Chapter~\ref{}. We begin with discussing the dynamic type of a sentence and its dynamic interpretation in simply-typed lambda calculus.
 
The meaning of a sentence is a function of the context. This can be expressed in lambda calculus by defining the interpretation of a sentence as an abstraction over a variable standing for the context. The notion of context is a complex notion and its formalization is not trivial. Indeed, there exist various proposals of its formalization. \TODO{Put references. Briefly mention these approaches.} 

The framework we propose is not restricted to any specific formalization of a context. In contrast, it provides room for integrating a desired context's representation and the context can be elaborated without affecting the core of the framework. This is achieved by defining the context as a term of a special type $\gamma$ viewed as a parameter. Hence, $\gamma$ can define any complex type.


\begin{definition}[Context, Environment] A \textbf{context} or \textbf{environment} is a term of type $\gamma$ that stores the essential information from what has already been processed in the computation of the meaning of the whole discourse.
\end{definition}

As mentioned above, a sentence is defined as an abstraction over a variable of type $\gamma$. In other words, the sentence awaits for a context. Moreover, the sentence may have a potential to change (update) the context. The update of the context can easily be implemented within the body of the lambda term interpreting the sentence. Furthermore, the updated context has to become available for a \emph{subsequent} sentence. The challenge is to implement this requirement within the interpretation of the current sentence in order to satisfy the compositionality principle. Here is where the notion of continuation is useful. The interpretation of a sentence takes the second argument standing for a continuation. In the body of the sentence, the continuation is provided the updated context as an argument and, therefore, this context becomes available for the future of the computation (and, particularly, for the subsequent sentence). We consider the continuation fed with the context to be an additional conjunct in the body of the term interpreting the sentence.


 We shall clarify the type of continuation. As discussed above, it serves for providing current (possibly updated) context to the future of the computation. Therefore, it is a function that has an argument of type $\gamma$. Since continuation applied to the context is an additional conjunct in the logical interpretation of the sentence, the continuation should return a proposition when given a context. Consequently, the continuation is defined as a term of type $(\gamma \rightarrow o)$.


\begin{definition}[Continuation] A \textbf{continuation} is a term of type $(\gamma \rightarrow o)$ that denotes what is still to be processed in the computation of the meaning of the whole discourse. 
\end{definition}

 Thereupon, we define the meaning of a sentence as a function of two arguments, a context $e$ of type $\gamma$ and a continuation $\phi$ of type $(\gamma \rightarrow o)$, that returns a proposition:
\begin{align}
\I{s} = \underbrace{\gamma}_{\begin{smallmatrix}
\text{type of}\\
\text{context}
\end{smallmatrix}} \rightarrow \underbrace{(\gamma \rightarrow o)}_{\begin{smallmatrix}
\text{type of}\\
\text{continuation}
\end{smallmatrix}} \rightarrow \underbrace{o}_{
\begin{smallmatrix}
\text{type of}\\
\text{proposition}
\end{smallmatrix}} \notag
\end{align}

Consequently, we define $(\gamma \rightarrow (\gamma \rightarrow o) \rightarrow o)$ to be the type of a dynamic proposition:
\begin{definition}[Type of a dynamic proposition] Every dynamic proposition is a term of type $(\gamma \rightarrow (\gamma \rightarrow o) \rightarrow o)$.
\end{definition}


Example~\ref{ex:2006-jlm} shows the dynamic meaning~\eqref{eq:ex:lovejm} of the simple sentence~\eqref{sent:JlovesM-2006}. Note the presence of the conjunct $\phi e^*$ in~\eqref{eq:ex:lovejm} that conveys that an updated context is passed as an argument to the continuation of a proposition, and is, therefore, accessible in the rest of the computation. As mentioned above, this kind of conjunct is a subterm of every proposition in the dynamic approach.
 \begin{example} \label{ex:2006-jlm} The meaning of the sentence~\eqref{sent:JlovesM-2006} is the lambda-term~\eqref{eq:ex:lovejm}:\footnote{We interpret objects as variables, i.e. as terms of type $\iota$.}
\enumsentence{ \txt{John loves Mary.} \label{sent:JlovesM-2006}}
\begin{align}
\underbrace{\lambda \overbrace{\underbrace{e^{\gamma}}_{\text{context}} \underbrace{\phi^{\gamma \rightarrow o}}_{\text{continuation}}.  \overbrace{\overbrace{ \overbrace{\textbf{love}^{\iota \rightarrow \iota \rightarrow o}  \ \textbf{j}^{\iota}}^{\iota \rightarrow o} \ \textbf{m}^{\iota}}^{o} \land \overbrace{\phi e^*}^{o}}^{o}}^{\gamma \rightarrow (\gamma \rightarrow o) \rightarrow o} }_{\text{dynamic proposition}} \label{eq:ex:lovejm}
%\underbrace{\underbrace{a+b}_\textrm{brace1} + c + d}_\textrm{brace2}
\end{align}
\indent where  $e^*$ is the context obtained by updating $e$.
\end{example}

 Context $e^*$ is just an abbreviation in~\eqref{eq:ex:lovejm}. To have the complete representation of the context, we should first decide what the parameter $\gamma$ is standing for. We will consider here a simple definition of $\gamma$ as a list of individuals:
\begin{align} \label{def:gammaislistofiota}
\gamma \defeq \texttt{ list of } [ \ \iota \ ]  
\end{align}
 
 
 This choice of the simple representation of the context is motivated  by our preference to focus on the mechanism of our compositional framework and avoid being distracted from this goal with the discussion what the ideal interpretation of the context should be. Indeed, the definition~\eqref{def:gammaislistofiota} of $\gamma$ is already sufficient for illustrating the basic idea of the approach on handling cross-sentential and donkey anaphora.
 

The context defined in~\eqref{def:gammaislistofiota}  stores interpretations of objects that previously occurred in the discourse. When a new object is interpreted as an individual $x$, the current context $e$ is updated with $x$, resulting in $(x::e)$, where $::$ is a list constructor of type $(\iota \rightarrow \gamma \rightarrow \gamma)$:

\begin{notation}[List constructor] The list constructor $::$ is a function that takes an individual and a context and returns an (updated) context:
\begin{align} 
 \I{::}  =  \iota \rightarrow \gamma \rightarrow \gamma
\end{align}
\end{notation}

\begin{remark}
Operation $::$ is right associative. For example, $(x::y::e)$ is equivalent to $(x::(y::e))$.
\end{remark}

Having defined the list constructor, we can substitute $e^*$ with the more precise $(\textbf{m} :: \textbf{j} ::{e})$ in Example~\ref{ex:2006-jlm}. Consequently, Sentence~\eqref{sent:JlovesM-2006} is more accuratly interpreted as follows:
\begin{align}
\lambda e \phi. \textbf{love}  \ \textbf{j} \ \textbf{m} \land \phi (\textbf{m} :: \textbf{j} ::{e}) \label{eq:ex:lovejm-2}
\end{align}

Term~\eqref{eq:ex:lovejm-2} has to be computed compositionally from lexical meanings $\I{John}$, $\I{Mary}$ and $\I{loves}$. More precisely, it has to be the result of normalizing  $\I{loves} \I{Mary} \I{John}$, as can be seen from the syntactic tree in Figure~\ref{fig:JohnLovesMary}.
\begin{figure}[h!]
 \centering
    \includegraphics[width=0.4\textwidth]{images/JohnLovesMary.pdf}
\caption{Syntactic parse tree of sentence \txt{John loves Mary}.} \label{fig:JohnLovesMary}
\end{figure}

A noun phrase in Montague semantics is a term taking a property as an argument and returning a proposition.  As motivated above, there should be two additional arguments (one standing for context and one standing for continuation) for a term to return a proposition in our dynamic framework. Therefore, everywhere where a term of type $o$ occurs in Montague's interpretation, there has to be a term of type $(\gamma \rightarrow (\gamma \rightarrow o) \rightarrow o)$ in the dynamic framework. This can be easily seen comparing~\eqref{eq:np:M:1} and~\eqref{eq:np:dG:1}, where $\Omega$ is an abbreviation for $(\gamma \rightarrow (\gamma \rightarrow o) \rightarrow o)$. Thus, a noun phrase is interpreted as a function of three arguments (a property, a context and a continuation) that returns a proposition, as can be more easily seen in~\eqref{eq:np:dG:2}, where no abbreviation is used:
\begin{subequations}
\begin{align}
\I{np} =_{Montague} \ & \underbrace{(\iota \rightarrow   o)}_{
\begin{smallmatrix}
\text{static}\\
\text{property}
\end{smallmatrix}} \rightarrow \underbrace{o}_{\begin{smallmatrix} 
\text{static}\\
\text{proposition}
\end{smallmatrix}} \label{eq:np:M:1} \\
\I{np} =_{de~Groote} \ & \underbrace{(\iota \rightarrow  \Omega )}_{\begin{smallmatrix}
\text{dynamic}\\
\text{property}
\end{smallmatrix}} \rightarrow \underbrace{\Omega}_{
\begin{smallmatrix}
\text{dynamic}\\
\text{proposition}
\end{smallmatrix}} \label{eq:np:dG:1} \\
\I{np} =_{de~Groote} \ & \underbrace{(\iota \rightarrow \gamma \rightarrow (\gamma \rightarrow o) \rightarrow o)}_{
\begin{smallmatrix}
\text{dynamic}\\
\text{property}
\end{smallmatrix}} \rightarrow \underbrace{\underbrace{\gamma}_{\text{context}} \rightarrow \underbrace{(\gamma \rightarrow o)}_{\text{continuation}} \rightarrow \underbrace{o}_{\text{proposition}}}_{\begin{smallmatrix}
\text{dynamic}\\
\text{proposition}
\end{smallmatrix}} \label{eq:np:dG:2}
\end{align}
\label{eq:np:MdG}
\end{subequations}

The interpretation of \txt{Mary}, for example, is as follows: 
\begin{align}
\I{Mary} =  \overbrace{\lambda \underbrace{{\P2}^{\iota \rightarrow \gamma \rightarrow (\gamma \rightarrow o) \rightarrow o}}_{\begin{smallmatrix}
\text{dynamic}\\
\text{property}
\end{smallmatrix}}. \underbrace{\lambda \underbrace{{e}^{\gamma}}_{\text{context}} \underbrace{{\phi}^{\gamma \rightarrow o}}_{\text{continuation}}. \overbrace{\overbrace{\overbrace{\P2 {\textbf{m}}^{\iota}}^{\gamma \rightarrow (\gamma \rightarrow o) \rightarrow o} e}^{(\gamma \rightarrow o) \rightarrow o} \ \ (\overbrace{\lambda e'^{\gamma}. \overbrace{\phi ( \overbrace{\upii{\textbf{m}}{e'}}^{\gamma})}^{o}}^{\gamma \rightarrow o})}^{o}}_{
\begin{smallmatrix}
\text{dynamic}\\
\text{proposition}
\end{smallmatrix}}}^{(\iota \rightarrow \gamma \rightarrow (\gamma \rightarrow o) \rightarrow o) \rightarrow \gamma \rightarrow (\gamma \rightarrow o) \rightarrow o}\label{eq:dG:Mary}
\end{align}

The interpretation of \txt{John} is analogous:
\begin{align}
\I{John} = \lambda \P2. \lambda e \phi. \P2 \textbf{j} e (\lambda e' .\phi (\textbf{j}::{e'})) \label{eq:dG:John}
\end{align}

A transitive verb is interpreted in Montague semantics as a term taking two type-raised individuals and returning a proposition.  Since in the dynamic framework there has to be an abstraction over a context and a continuation to get a proposition,  everywhere where a term of type $o$ occurs in Montague's interpretation, there has to be a term of type $(\gamma \rightarrow (\gamma \rightarrow o) \rightarrow o)$ in de Groote's interpretation. This can be seen comparing types in~\eqref{eq:tv:MdG}:
\begin{subequations}
\begin{align}
\I{tv} =_{Montague} \ & (\underbrace{(\iota \rightarrow   o)}_{\text{property}} \rightarrow \underbrace{o}_{\text{proposition}}) \rightarrow (\underbrace{(\iota \rightarrow   o)}_{\text{property}} \rightarrow \underbrace{o}_{\text{proposition}})  \rightarrow \underbrace{o}_{\text{proposition}} \\
\I{tv} =_{de~Groote} \ &  (\underbrace{(\iota \rightarrow   \Omega)}_{\text{property}} \rightarrow \underbrace{\Omega}_{\text{proposition}}) \rightarrow (\underbrace{(\iota \rightarrow   \Omega)}_{\text{property}} \rightarrow \underbrace{\Omega}_{\text{proposition}})  \rightarrow \underbrace{\Omega}_{\text{proposition}} 
\end{align} \label{eq:tv:MdG}
\end{subequations}

Then the interpretation of \txt{loves} is as follows:
\begin{align}
\I{loves} = \overbrace{\lambda \Y2^{(\iota \rightarrow \Omega) \rightarrow \Omega} \X2^{(\iota \rightarrow \Omega) \rightarrow \Omega}.  \overbrace{\X2 ( \overbrace{\lambda \x1.  \overbrace{\Y2 ( \overbrace{\lambda \y1. (  \overbrace{\lambda {e'}^{\gamma} \phi^{\gamma \rightarrow o}. \overbrace{\overbrace{\overbrace{{\textbf{love}}^{\iota \rightarrow \iota \rightarrow o} {\x1}^{\iota}}^{\iota \rightarrow o} {\y1}^{\iota}}^{o} \land \overbrace{\phi e'}^{o}}^{o} )}^{\Omega}}^{\iota \rightarrow \Omega} )}^{\Omega}}^{\iota \rightarrow \Omega} )}^{\Omega} }^{((\iota \rightarrow \Omega) \rightarrow \Omega) \rightarrow ((\iota \rightarrow \Omega) \rightarrow \Omega) \rightarrow \Omega} \label{eq:dG:love}
\end{align}



\begin{example}[$\S_1$, \txt{John loves Mary}] Now, given lexical interpretations~\eqref{eq:dG:love},~\eqref{eq:dG:Mary} and~\eqref{eq:dG:John} of \txt{loves}, \txt{Mary} and \txt{John} respectively, the meaning~\eqref{eq:ex:lovejm-2} (\eqref{eq:2006:JohnLovesMary} below) of Sentence~\eqref{sent:JlovesM-2006}  can be computed compositionally:
\begin{align}
\S_1 = \ & \I{loves} \I{Mary} \I{John}  \notag \\
= \ & (\lambda \Y2 \X2. \X2( \lambda \x1. \Y2 (\lambda \y1. ( \lambda e' \phi. \textbf{love} \x1 \y1 \land \phi e' ))) )  \I{Mary} \I{John}  \notag \\
\bconv \ & (\lambda  \X2. \X2( \lambda \x1. \I{Mary}  (\lambda \y1. ( \lambda e' \phi. \textbf{love} \x1 \y1 \land \phi e' ))) )  \I{John}  \notag \\
\bconv \ &    \I{John}( \lambda \x1. \I{Mary}  (\lambda \y1. ( \lambda e' \phi. \textbf{love} \x1 \y1 \land \phi e' )))   \notag \\
= \ &    \I{John}( \lambda \x1. (\lambda \P2. \lambda e \phi. \P2 \textbf{m} e (\lambda e. \phi (\upii{\textbf{m}}{e'})))  (\lambda \y1. ( \lambda e' \phi. \textbf{love} \x1 \y1 \land \phi e' )))   \notag \\
\bconv \ &    \I{John}( \lambda \x1.  \lambda e \phi. (\lambda \y1. ( \lambda e' \phi. \textbf{love} \x1 \y1 \land \phi e' )) \textbf{m} e (\lambda e'. \phi (\upii{\textbf{m}}{e'})))   \notag \\
\bconv \ &    \I{John}( \lambda \x1.  \lambda e \phi. ( \lambda e' \phi. \textbf{love} \x1 \textbf{m} \land \phi e' ) e (\lambda e' . \phi (\upii{\textbf{m}}{e'})))   \notag \\
\bred \ &  \I{John}( \lambda \x1.  \lambda e \phi.  \textbf{love} \x1 \textbf{m} \land  (\lambda e' .\phi (\upii{\textbf{m}}{e'}) e))   \notag \\
\bconv \ &  \I{John}( \lambda \x1.  \lambda e \phi.  \textbf{love} \x1 \textbf{m} \land  \phi (\upii{\textbf{m}}{e} ))   \notag \\
= \ &    ( \lambda \P2. \lambda e \phi. \P2 \textbf{j} e (\lambda e' .\phi (\upii{\textbf{j}}{e'})))( \lambda \x1.  \lambda e \phi.  \textbf{love} \x1 \textbf{m} \land  \phi (\upii{\textbf{m}}{e} ))   \notag \\
\bconv \ &    \lambda e \phi. ( \lambda \x1.  \lambda e \phi.  \textbf{love} \x1 \textbf{m} \land  \phi (\upii{\textbf{m}}{e} ))   \textbf{j} e (\lambda e' . \phi (\upii{\textbf{j}}{e'})) \notag \\
\bred \ &    \lambda e \phi.  \textbf{love}  \textbf{j} \textbf{m} \land   (\lambda e' .\phi (\upii{\textbf{j}}{e'})) (\upii{\textbf{m}}{e} )  \notag \\
\bconv \ &    \lambda e \phi.  \textbf{love}  \textbf{j} \textbf{m} \land   \phi (\upii{\textbf{j}}{\upii{\textbf{m}}{e} })   \label{eq:2006:JohnLovesMary} 
\end{align}
\end{example}

To cope with anaphora, the context has to be accessed. This can be accomplished by defining a special function $\selK$ of type $(\gamma \rightarrow \iota)$ that takes a context and returns an individual. Assuming that $\selK$ implements an anaphora resolution algorithm and  works as an oracle always retrieving the correct antecedent makes it possible to interpret pronouns as shown, for example, for \txt{he} below:
\begin{align}
\I{he} =  \overbrace{\lambda \P2^{^\iota \rightarrow \gamma \rightarrow (\gamma \rightarrow o) \rightarrow o}. \lambda e^{\gamma} \phi^{\gamma \rightarrow o}. \overbrace{\overbrace{\overbrace{\P2 ( \overbrace{\selK_{he} e}^{\iota} )}^{\gamma \rightarrow (\gamma \rightarrow o) \rightarrow o} e}^{ (\gamma \rightarrow o) \rightarrow o} \ \phi }^{o}}^{(\iota \rightarrow \gamma \rightarrow (\gamma \rightarrow o) \rightarrow o) \rightarrow \gamma \rightarrow (\gamma \rightarrow o) \rightarrow o}  \label{eq:he:2006}
%\I{he} = \lambda \P2. \lambda e \phi. \P2 (\selK_{he}e)e\phi
\end{align}

\begin{example}[$\S_2$, \txt{He smiles at her}] \label{ex:2006:HeSmilesAtHer} The meaning of the sentence~\eqref{HeSmilesAtHer-2006} computed in accordance with the parse-tree in Figure~\ref{fig:ptS3-2006}  is as follows:
\enumsentence{\txt{He smiles at her.} \label{HeSmilesAtHer-2006}}
\begin{align}
\S_2 = \ & \I{smiles\_at} \I{her} \I{he} \notag \\
 = \ & (\lambda \Y2 \X2. \X2( \lambda \x1. \Y2 (\lambda \y1. ( \lambda e' \phi. \textbf{smile} \x1 \y1 \land \phi e' ))) )  \I{her} \I{he} \notag \\
\bconv \ & (\lambda  \X2. \X2( \lambda \x1.  \I{her}  (\lambda \y1. ( \lambda e' \phi. \textbf{smile} \x1 \y1 \land \phi e' ))) ) \I{he} \notag \\
\bconv \ &   \I{he} ( \lambda \x1.  \I{her}  (\lambda \y1. ( \lambda e' \phi. \textbf{smile} \x1 \y1 \land \phi e' ))) \notag  \\
= \ &   \I{he} ( \lambda \x1.  (\lambda \P2. \lambda e \phi. \P2 (\selK_{her}e)e\phi)  (\lambda \y1. ( \lambda e' \phi. \textbf{smile} \x1 \y1 \land \phi e' ))) \notag \\
 \bconv \ &   \I{he} ( \lambda \x1.  (\lambda e \phi.  (\lambda \y1. ( \lambda e' \phi. \textbf{smile} \x1 \y1 \land \phi e' )) (\selK_{her}e)e\phi) ) \notag \\
  \bconv \ &   \I{he} ( \lambda \x1.  (\lambda e \phi.   ( \lambda e' \phi. \textbf{smile} \x1  (\selK_{her}e) \land \phi e' ) e\phi) ) \notag \\
 \bred \ &   \I{he} ( \lambda \x1.  (\lambda e \phi.  \textbf{smile} \x1  (\selK_{her}e) \land \phi e) ) \notag \\
  = \ &  (\lambda \P2. \lambda e \phi. \P2 (\selK_{he}e)e\phi) ( \lambda \x1.  (\lambda e \phi.  \textbf{smile} \x1  (\selK_{her}e) \land \phi e) ) \notag \\
\bconv \ & \lambda e \phi. ( \lambda \x1.  (\lambda e \phi.  \textbf{smile} \x1  (\selK_{her}e) \land \phi e) )  (\selK_{he}e)e\phi \notag \\
\bconv \ & \lambda e \phi.   (\lambda e \phi.  \textbf{smile}  (\selK_{he}e)  (\selK_{her}e) \land \phi e) e\phi \notag \\
\bred \ & \lambda e \phi.   \textbf{smile}  (\selK_{he}e)  (\selK_{her}e) \land \phi e \label{eq:2006:HeSmilesAtHer}
\end{align}
\end{example}

\begin{figure}[h]
 \centering
    \includegraphics[width=0.4\textwidth]{images/HeSmilesatHer.pdf}
\caption{Syntactic parse tree of sentence \txt{He smiles at her}.} \label{fig:ptS3-2006}
\end{figure}

As Example~\ref{ex:2006:HeSmilesAtHer} shows, Sentence~\eqref{HeSmilesAtHer-2006} is meaningful in the sense that it has an interpretation~\eqref{eq:2006:HeSmilesAtHer}. However, the function $\selK$ can return individuals for \txt{he} and \txt{her} only when the sentence is evaluated over some context containing the corresponding antecedents. This happens when the sentence is uttered in a discourse. The pronominal anaphora can then be resolved during the update of the meaning of the discourse with the meaning of the sentence. 

Discourses can, like sentences, be interpreted as terms of type $(\gamma \rightarrow (\gamma \rightarrow o) \rightarrow o)$.  The update of a discourse interpreted as $\D$ with a sentence interpreted as $\S$ results in interpretation $\updt \ \D \ \S$ of a new discourse. This interpretation is defined by the following equation:
\begin{align}
\updt \ \D \ \S \defeq \overbrace{\lambda e^{\gamma} \phi^{\gamma \rightarrow o}. \overbrace{ \overbrace{\D^{\gamma \rightarrow (\gamma \rightarrow o) \rightarrow o}  e}^{(\gamma \rightarrow o) \rightarrow o} (\overbrace{\lambda e'^{\gamma}. \overbrace{\overbrace{\S^{\gamma \rightarrow (\gamma \rightarrow o) \rightarrow o}  e'}^{ (\gamma \rightarrow o) \rightarrow o} \phi}^{o}}^{\gamma \rightarrow o})}^{o}}^{\gamma \rightarrow (\gamma \rightarrow o) \rightarrow o} \label{eq:updtDS:2006}
\end{align}

\TODO{Take an empty discourse. Feed it with Sentence 1. Feed the result with Sentence 2.}

\begin{example}[$\updt \ \D \ \S$]
Now interpretations~\eqref{eq:2006:JohnLovesMary} and~\eqref{eq:2006:HeSmilesAtHer} can be composed through equation~\eqref{eq:updtDS:2006}, regarding~\eqref{sent:JlovesM-2006} as the discourse updated with the sentence~\eqref{HeSmilesAtHer-2006}. This leads to the interpretation of the piece of discourse~\eqref{sen:JLMHeSH:2006}:
\enumsentence{\txt{John loves Mary. He smiles at her.} \label{sen:JLMHeSH:2006}}
\begin{align}
& \lambda e \phi. (\overbrace{\lambda e \phi.  \textbf{love}  \textbf{j} \textbf{m} \land   \phi (\upii{\textbf{j}}{\upii{\textbf{m}}{e} }) }^{\D}  ) e (\lambda e'. (\overbrace{\lambda e \phi.   \textbf{smile}  (\selK_{he}e)  (\selK_{her}e) \land \phi e}^{\S}) e' \phi ) \notag \\
\bred \ & \lambda e \phi. (\lambda e \phi.  \textbf{love}  \textbf{j} \textbf{m} \land   \phi (\upii{\textbf{j}}{\upii{\textbf{m}}{e} }) ) e (\lambda e'.  \textbf{smile}  (\selK_{he}e')  (\selK_{her}e') \land \phi e' ) \notag \\
\bred \ & \lambda e \phi.\textbf{love}  \textbf{j} \textbf{m} \land    (\lambda e'.  \textbf{smile}  (\selK_{he}e')  (\selK_{her}e') \land \phi e' )  (\upii{\textbf{j}}{\upii{\textbf{m}}{e} }) \notag \\
\bconv \ & \lambda e \phi.\textbf{love}  \textbf{j} \textbf{m} \land    \textbf{smile}  (\selK_{he}(\upii{\textbf{j}}{\upii{\textbf{m}}{e} }))  (\selK_{her}(\upii{\textbf{j}}{\upii{\textbf{m}}{e} })) \land \phi  (\upii{\textbf{j}}{\upii{\textbf{m}}{e} }) \label{int:JLMHeSH:2006}
\end{align}
\end{example}

%Interpretation~\eqref{int:JLMHeSH:2006} of the discourse consisting of the utterance of~\eqref{sen:JLMHeSH:2006} is computed in a compositional manner. 
Note that in the resulting interpretation~\eqref{int:JLMHeSH:2006}, the context of the interpretation of the first sentence is passed to $\selK$ operators of the interpretation of the second sentence. Assuming that an anaphora resolution mechanism is implemented in $\selK$, the following semantic representation of~\eqref{sen:JLMHeSH:2006} is obtained: 
\begin{align}
\lambda e \phi.\textbf{love} \  \textbf{j} \ \textbf{m} \land    \textbf{smile}  \ \textbf{j} \ \textbf{m} \land \phi  (\upii{\textbf{j}}{\upii{\textbf{m}}{e} }) \label{int:JLMHeSH:2006-2}
\end{align}
The context $(\upii{\textbf{j}}{\upii{\textbf{m}}{e} })$ in~\eqref{int:JLMHeSH:2006} (and hence in~\eqref{int:JLMHeSH:2006-2}) being the argument of the continuation $\phi$ is accessible for future computation. This means that the individuals $\textbf{j}$ and $\textbf{m}$ can serve as ancestors for anaphoric pronouns in the following sentences. 

\section{The underlying dynamic logic} \label{sec:dyn_log}

It is possible to construct dynamic meanings as those discussed in Section~\ref{sec:cont_based_dyn} in a concise and systematic way. This is achieved by defining a continuation-based dynamic logic and this section presents this logic.

Terms and types are given by Definitions~\ref{def:terms-FO} and~\ref{def:types-FO}:
\begin{definition}[$\lambda$-terms]\label{def:FO-lambda-terms}
 The set of \textbf{$\lambda$-terms} $\Lambda$ is constructed from an enumerable set of variables $V = \{ v, v_1, v_2, \dots \}$, logical constants $\land, \exists, \neg$, two special constants $::$ and $\selK$, an enumerable set of predicate symbols $R = \{ R_1, R_2, \dots  \}$ and an enumerable set of constants $K = \{ c, c_1, c_2, \dots \}$  using application and (function) abstraction:
\begin{center}
$
\begin{array}{rcl}
x \in V & \Longrightarrow & x \in \Lambda\\
c \in K & \Longrightarrow & c \in \Lambda\\
M,N \in \Lambda &  \Longrightarrow & (MN) \in \Lambda\\ 
 x \in V, M \in \Lambda&  \Longrightarrow  & (\lambda x.M) \in \Lambda \\
  M \in \Lambda&  \Longrightarrow  & (\exists M) \in \Lambda\\ 
M,N \in \Lambda &  \Longrightarrow  & (M \land N) \in \Lambda \\
M \in \Lambda &  \Longrightarrow  & (\neg M)  \in \Lambda \\
M,x \in \Lambda &  \Longrightarrow  & (M::x) \in \Lambda\\
M \in \Lambda &  \Longrightarrow  & (\selK(M))  \in \Lambda
\end{array} 
$
\end{center}
\label{def:terms-FO}
\end{definition}

%M,x \in \Lambda &  \Longrightarrow  & (\updtfo(M,x))  \in \Lambda\\
%M \in \Lambda &  \Longrightarrow  & (\selK(M))  \in \Lambda\\


\begin{definition}[Free variables] The set of \textbf{free variables} of $t$, $FV(t)$, is defined inductively as follows:
\begin{align*}
FV(x) = \ &  \{ x \} \\
FV(c) = \ & \{ \} \\
%FV(\exists M) =  \ & FV(M) \\
%FV(M \land N) = \ & FV(M) \cup FV(N)  \\
%FV(\neg M) =  \ & FV(M) \\
%FV( \updtfo (M,x)) = \ &  FV(M) \cup FV(x)  \\
%FV(\selK(M)) = \ & FV(M) \\
FV(M N) = \ & FV(M) \cup FV(N)  \\
FV( \lambda x. M) = \ & FV(M) - \{ x \}
\end{align*}
\end{definition}

\begin{definition}[Types]
The set $\Types$ of types is defined inductively as follows:
\begin{center} $
\begin{array}{rll}
\text{Atomic types:} & \iota \in \Types & \text{(static individuals)} \\
 & o \in \Types &  \text{(static propositions)} \\
&  \gamma \in \Types &  \text{(context)} \\\\
 \text{Complex types:} & \alpha, \beta \in \Types \Longrightarrow & (\alpha \rightarrow \beta) \in \Types \\
\end{array} 
$ \end{center}
\label{def:types-FO}
\end{definition}

Typing rules define typing relations between terms and types:
\begin{definition}[Typing rules] A statement $t:\delta$, meaning $t$ has type $\delta$, is \textbf{derivable} from the basis $\Delta$, i.e. $\Delta \vdash t: \delta$, if $\Delta \vdash t: \delta$ can be produced using the following rules:

\begin{prooftree}
\AXC{} \RightLabel{axiom}
\UIC{$\Gamma, x: \alpha \vdash x: \alpha $}
\end{prooftree}

\begin{prooftree}
\AXC{} \RightLabel{axiom}
\UIC{$\Gamma \vdash \top: o $}
\end{prooftree}

\begin{prooftree}
\AXC{} %\RightLabel{axiom}
\UIC{$\Gamma, M: o, N:o \vdash M \land N: o $}
\end{prooftree}

\begin{prooftree}
\AXC{} %\RightLabel{axiom}
\UIC{$\Gamma, M: \iota \rightarrow o \vdash \exists M: o $}
\end{prooftree}

\begin{prooftree}
\AXC{} %\RightLabel{axiom}
\UIC{$\Gamma, M: o \vdash \neg M: o $}
\end{prooftree}

\begin{prooftree}
\AXC{} %\RightLabel{axiom}
\UIC{$\Gamma, c: \gamma \vdash \selK \ c:  \iota $}
\end{prooftree}

\begin{prooftree}
\AXC{} %\RightLabel{axiom}
\UIC{$\Gamma, i: \iota, c: \gamma \vdash (i::c) :  \gamma$}
\end{prooftree}


\begin{prooftree}
\AXC{} \RightLabel{axiom}
\UIC{$\Gamma \vdash c_{iv}:   \iota \rightarrow o $}
\end{prooftree}


\begin{prooftree}
\AXC{} \RightLabel{axiom}
\UIC{$\Gamma \vdash c_{tv}:   \iota \rightarrow \iota \rightarrow o $}
\end{prooftree}

\begin{prooftree}
\AXC{} \RightLabel{axiom}
\UIC{$\Gamma \vdash c_{n}:   \iota \rightarrow o $}
\end{prooftree}

\begin{prooftree}
\AXC{} \RightLabel{axiom}
\UIC{$\Gamma \vdash c_{np}:   (\iota \rightarrow o) \rightarrow o $}
\end{prooftree}

%\begin{prooftree}
%\AXC{} \RightLabel{axiom}
%\UIC{$\Gamma \vdash c_{det}:   (\iota \rightarrow o) \rightarrow ( ( \iota \rightarrow o ) \rightarrow o  )$}
%\end{prooftree}

\begin{prooftree}
\AXC{} \RightLabel{axiom}
\UIC{$\Gamma \vdash c_{rp}:  (  ((\iota \rightarrow o) \rightarrow o)  \rightarrow o)  \rightarrow   ( \iota \rightarrow o ) \rightarrow ( \iota \rightarrow o )$}
\end{prooftree}


\begin{prooftree}
\AXC{$\Gamma \vdash v: \alpha \rightarrow \beta$}
\AXC{$\Gamma \vdash u: \alpha$} \RightLabel{app}
\BIC{$\Gamma \vdash vu: \beta$}
\end{prooftree}

\begin{prooftree}
\AXC{$\Gamma, x: \alpha \vdash v:\beta$} \RightLabel{abs}
\UIC{$\Gamma \vdash \lambda x.v: \alpha \rightarrow \beta$}
\end{prooftree}
where $c_{tv}$,  $c_{iv}$, $c_{n}$, $c_{np}$ and $c_{rp}$ are constants standing for transitive and intransitive  verbs, nouns, noun phrases and relative pronouns respectively. Typing rules for other syntactic categories can be added analogously.

\end{definition}

The first axiom and the two rules (application and abstraction) are standard typing relations in simply-typed lambda calculus. The second axiom determines the type of the logical $\top$ symbol. The constant $\selK$ has type $(\gamma \rightarrow \iota)$ and the constant $::$ has type $(\iota \rightarrow \gamma \rightarrow \gamma)$. 

For each logical constant ($\neg$, $\land$ and $\exists$), its dynamic counterpart is specified by the following definition: 
\begin{definition}[Dynamic logical constants] \label{def:DLCnst-FO} Let $\A2$ and $\B2$ be terms of type $(\gamma \rightarrow (\gamma \rightarrow o) \rightarrow o)$, $\P2$ be the term of type $(\iota \rightarrow \gamma \rightarrow (\gamma \rightarrow o) \rightarrow o)$, $e$ and  $e'$ be terms of type $\gamma$, $\phi$ be a term of type $(\gamma \rightarrow o)$, $\x1$ be the term of type $\iota$. \textbf{Dynamic negation, conjunction and existential quantification} are defined respectively as follows:
\begin{subequations}
\begin{align}
\n2 \A2 \defeq \ & \lambda e \phi. \neg (\A2 e (\lambda e'. \top) ) \land \phi e  \label{def:DLCnst-FO-neg} \\
\A2 \cnj2 \B2 \defeq \ & \lambda e\phi. \A2 e (\lambda e'. \B2 e' \phi) \label{def:DLCnst-FO-cnj} \\
\ex2 ( \lambda \x1. \P2[\x1]) \defeq \ & \lambda e \phi. \exists ( \lambda \x1. \P2 [\x1] \ (x::e) \ \phi )\label{def:DLCnst-FO-ex} 
\end{align}
\end{subequations}
\end{definition}
Dynamic negation of a dynamic proposition $\A2$, $\n2 \A2$, is an abbreviation used for the term shown in~\eqref{def:DLCnst-FO-neg}. Within the body of this term, the continuation and context of $\A2$ are ``erased'' (by giving the term $(\lambda e'. \top)$ as the second argument of $\x2$)\footnote{This can be more clearly seen from Corollary~\ref{cor:DL-FO-handtop}.}, the resulting static proposition is negated, and the conjunct $\phi e$ is added. Note that both $\phi$ and $e$ are variables bound by $\lambda$. Therefore, the context of $\A2$ is not available to the continuation of the resulting term $\n2 \A2$. 

Dynamic existentially quantified term $\ex2 ( \lambda \x1. \P2[\x1]) $ is an abbreviation for the term shown in~\eqref{def:DLCnst-FO-ex}. It has a $\lambda$-abstraction over variables $e$ and $\phi$ and an existentially quantified variable $\x1$. Its body contains $\P2[\x1]$, which is given $e$ updated with $\x1$, $(x::e)$, and $\phi$ as arguments.  Note that $\n2$ and $\ex2$ are defined respectively via $\neg$ and $\exists$.
Dynamic conjunction is defined as a composition of two dynamic terms. The logical conjunction of static propositions is provided by the fact that each dynamic proposition has a conjunct $\phi e $ in its body.


 Static negation, conjunction and existential quantification can now be translated into dynamic ones: 
\begin{definition}[Embedding of first order logic into dynamic logic]\label{def:DTerms-FO} Let $\P1$ be a term of type $(\iota_1 \rightarrow \dots \rightarrow \iota_n \rightarrow o)$, $\A1$ and $\B1$ be terms of type $o$, $\t1_1, \dots, \t1_n$ and $\x1$ be terms of type $\iota$. Then propositions, negated propositions, conjunctions of two propositions and existentially quantified propositions are dynamized in the following way:
\begin{subequations}
\begin{align}
\tr{ \P1 \t1_1 \dots \t1_n } \defeq \ & \lambda e \phi . \P1 \t1_1 \dots \t1_n \land \phi e  \label{def:DTerms-FO-P}\\
\tr{ \neg \A1  } \defeq \ & \n2 \tr{\A1}  \label{def:DTerms-FO-neg}\\
\tr{ \A1 \land \B1} \defeq \ & \tr{\A1} \cnj2 \tr{\B1}  \label{def:DTerms-FO-cnj}\\
\tr{\exists( \lambda \x1. \P1 [\x1])} \defeq \ & \ex2( \lambda  \x1. \tr{\P1 [\x1]})  \label{def:DTerms-FO-ex}
\end{align}
\end{subequations}
\end{definition}
Equation~\eqref{def:DTerms-FO-P} defines the dynamization of a proposition $\P1 \t1_1 \dots \t1_n $ of type $o$ by adding a $\lambda$-abstraction with two arguments  $e$ and $\phi$  (of types $\gamma$ and $(\gamma \rightarrow o)$ respectively) and a conjunct $\phi e$. Therefore, the resulting dynamic term is of type $(\gamma \rightarrow (\gamma \rightarrow o) \rightarrow o )$, the type of a dynamic proposition. Equations~\eqref{def:DTerms-FO-neg},~\eqref{def:DTerms-FO-cnj} and~\eqref{def:DTerms-FO-ex} extend dynamization to non-atomic formulas.



%Note that Definitions~\ref{def:DLCnst-FO} and~\ref{def:DTerms-FO} allow representing de~Groote's~\cite{deGroote:2006:Towards-a-Montagovian-Account-of-Dynamics} dynamic terms in a compact way.  While interpretations in~\cite{deGroote:2006:Towards-a-Montagovian-Account-of-Dynamics} explicitly show extra parameters, i.e. contexts and continuations, these new definitions make it possible to hide these parameters. Moreover, the resulting compact dynamic terms are structurally analogous to their original static counterparts and, hence, are more intuitive.


\begin{remark} In equations below, terms on the left side of $\defeq$ abbreviate respective terms on the right side:
\begin{subequations}
\begin{align*} 
\A2 \impK2 \B2 \defeq \ & \n2 (\A2 \cnj2 \n2 \B2) \\
\all2( \lambda \x1. \P2[\x1]) \defeq \ & \n2 \ex2 ( \lambda \x1. \n2 \P2 [\x1] )
\end{align*} 
\end{subequations}
\label{remark:impforall}
\end{remark}



Proposition~\ref{thrm:scopeextension-G}  proves a useful $\beta$-equivalence that can be useful when computing interpretations of certain phrases containing an existentially quantified variable.
%%%%%%%%%%%%%%%%%%%%%%%%%%%%%%%%%%%%%%%%%%%%%%%%%%%%%%%%%%%%%%%%%%%%%%%%%%%%%%%%%%%%%
%%%%%%%%%%%%%%%%%%%%%%%%%%%%%%%%%%%%%%%%%%%%%%%%%%%%%%%%%%%%%%%%%%%%%%%%%%%%%%%%%%%%%
\begin{proposition} For all terms $\A1$ and $\B1$ of type $o$ such that  $\x1 \in FV(\A1)$ and $\x1 \notin FV(\B1)$ for an $\x1$ of type~$\iota$, the following equivalence holds: \label{thrm:scopeextension-G}
\begin{align*}
  {\ex2} (\lambda \x1.  \tr{\A1[\x1]}) \cnj2 \tr{\B1}  =_{\beta}  \ex2 (\lambda \x1.  \tr{\A1[\x1]}  \cnj2  \tr{\B1} ) 
\end{align*} 
\end{proposition}
%%%%%%%%%%%%%%%%%%%%%%%%%%%%%%%%%%%%%%%%%%%%%%%%%%%%%%%%%%%%%%%%%%%%%%%%%%%%%%%%%%%%%
%
\begin{proof} Since $\A1[\x1]$ and $\B1$ are terms of type $o$, their dynamic equivalents are terms of type $(\gamma \rightarrow (\gamma \rightarrow o) \rightarrow o)$ and have the following forms:
\begin{subequations}
\begin{align}
 \tr{\A1[\x1]}  & = \lambda e \phi. \A1'[x] \land \phi e \label{eq:Adynamic} \\
 \tr{\B1} & = \lambda e \phi. \B1' \land  \phi e \label{eq:Bdynamic} 
\end{align}
\end{subequations}

%
\begin{align}
 &  {\ex2} (\lambda \x1.  \tr{\A1[\x1]}) \cnj2 \tr{\B1}  \notag \\ 
 \bred \  &  ( \lambda e \phi.  \exists ( \lambda \x1. \A1'[x] \land \phi (x::e) ) )  \cnj2  \tr{\B1}  \tag{by~\eqref{def:DLCnst-FO-ex} and~\eqref{eq:Adynamic}}\\
= \ &  \lambda e \phi. ( \lambda e \phi.  \exists ( \lambda \x1. \A1'[x] \land \phi (x::e) ) ) e ( \lambda e. \tr{\B1} e \phi  ) \tag{by~\eqref{def:DLCnst-FO-cnj}} \\
\bred \ &   \lambda e \phi. ( \lambda e \phi.  \exists ( \lambda \x1. \A1'[x] \land \phi (x::e) ) ) e ( \lambda e. \B1' \land  \phi e  )  \tag{by~\eqref{eq:Bdynamic}}\\
\bred \ &   \lambda e \phi.  \exists ( \lambda \x1. \A1'[x] \land  (  \B1' \land  \phi  (x::e)  ) ) \notag %\label{eq:DLGexscope1}
\end{align}
%
\begin{align}
 &  \ex2 (\lambda \x1.  \tr{\A1[\x1]}  \cnj2  \tr{\B1} ) \notag \\
  = \ &    \ex2 (\lambda \x1.  \lambda e \phi. \tr{\A1[\x1]}  e (\lambda e. \tr{\B1}  e \phi)  ) \tag{by~\eqref{def:DLCnst-FO-cnj}} \\
  \bred \ &  \ex2 (\lambda \x1.  \lambda e \phi. (\lambda e \phi. \A1'[x] \land \phi e ) e (\lambda e. \B1' \land  \phi e)  ) \tag{by~\eqref{eq:Adynamic} and~\eqref{eq:Bdynamic}}\\
    \bred \ &   \ex2 (\lambda \x1.  \lambda e \phi.\A1'[x] \land  (\B1' \land  \phi e )   ) \notag \\
\bred \ &   \lambda e \phi.  \exists ( \lambda \x1.   \A1'[x] \land  (  \B1' \land  \phi ( \x1:: e ) ) ) \tag{by~\eqref{def:DLCnst-FO-ex}}
\end{align}
%
\end{proof}

%%%%%%%%%%%%%%%%%%%%%%%%%%%%%%%%%%%%%%%%%%%%%%%%%%%%%%%%%%%%%%%%%%%%%%%%%%%%%%%%%%%%%
%%%%%%%%%%%%%%%%%%%%%%%%%%%%%%%%%%%%%%%%%%%%%%%%%%%%%%%%%%%%%%%%%%%%%%%%%%%%%%%%%%%%%
\begin{proposition} For all $\h1$ and $\v1$ of type $o$, and for all $\u1$ of type $\gamma$, the following holds: $$\tr{\h1} \u1 (\lambda e.\v1) =  \h1 \land \v1$$ where $e$ is a variable of type $\gamma$. \label{DL-FO-handv}
\end{proposition}
%%%%%%%%%%%%%%%%%%%%%%%%%%%%%%%%%%%%%%%%%%%%%%%%%%%%%%%%%%%%%%%%%%%%%%%%%%%%%%%%%%%%%
\begin{proof}
The proof is by induction on the structure of the term $\h1$.
\begin{itemize}
\item $\h1$ is a proposition of the form $\P1 t_1 \dots t_n$.
\begin{align}
\tr{\P1 \t1_1 \dots \t1_n} \u1  (\lambda e.\v1)  = \ & (\lambda e \phi. \P1 \t1_1 \dots \t1_n \land \phi e) \u1  (\lambda e.\v1) \tag{by~\eqref{def:DTerms-FO-P}} \\
\bconv \ & (\lambda  \phi. \P1 \t1_1 \dots \t1_n \land \phi \u1)   (\lambda e.\v1) \notag \\
\bconv \ & \lambda  \phi. \P1 \t1_1 \dots \t1_n \land (\lambda e.\v1) \u1    \notag \\
\bconv \ & \lambda  \phi. \P1 \t1_1 \dots \t1_n \land \v1    \notag 
\end{align}

\item  $\h1$ is a negated proposition $ \neg \w1$. 
\begin{align}
\tr{\neg \w1} \u1 (\lambda e. \v1) = \ &  \n2 \tr{\w1}  \u1 (\lambda e. \v1)  \tag{by~\eqref{def:DTerms-FO-neg}} \\
= \ & ( \lambda e \phi. \neg (\tr{\w1} e (\lambda e'. \top) ) \land \phi e)  \u1 (\lambda e. \v1)  \tag{by~\eqref{def:DLCnst-FO-neg}} \\
\bconv \ & ( \lambda \phi.\neg (\tr{\w1} \u1 (\lambda e'. \top) ) \land \phi \u1 ) (\lambda e. \v1)  \notag \\
\bconv \ & \neg (\tr{\w1} \u1 (\lambda e'. \top) ) \land  (\lambda e. \v1) \u1   \notag \\
\bconv \ & \neg (\tr{\w1} \u1 (\lambda e'. \top) ) \land   \v1   \notag \\
= \ & \neg(\w1 \land \top) \land \v1 \tag{by I.H.} \\
= \ & \neg \w1 \land \v1
\end{align}

\item $\h1$ is a conjunction of two propositions $\w1 \land \z1$.
\begin{align}
(\tr{\w1 \land \z1}) \u1 (\lambda e. \v1)  = \ & (\tr{\w1} \cnj2 \tr{\z1}) \u1 (\lambda e. \v1)   \tag{by~\eqref{def:DTerms-FO-cnj}} \\
 = \ & (\lambda e\phi. \tr{\w1} e (\lambda e'. \tr{\z1} e' \phi) ) \u1 (\lambda e. \v1)   \tag{by~\eqref{def:DLCnst-FO-cnj}} \\
\bconv \ & (\lambda \phi. \tr{\w1} \u1 (\lambda e'. \tr{\z1} e' \phi) ) (\lambda e. \v1)   \notag \\
\bconv \ &  \tr{\w1} \u1 (\lambda e'. \tr{\z1} e'  (\lambda e. \v1) )  \notag \\
= \ &  \tr{\w1} \u1 (\lambda e'. \z1 \land \v1)  \tag{by I.H., $e \notin FV(\v1)$} \\
= \ &  \w1 \land (\z1 \land \v1)  \tag{by I.H., $e' \notin FV(\z1 \land \v1)$} \\
\logeq \ & (\w1 \land \z1) \land \v1 \notag
\end{align}

\item $\h1$ is an existentially quantified formula of the form $\exists( \lambda \x1. \P1 [\x1])$.
\begin{align}
\tr{\exists ( \lambda \x1. \P1 [\x1])} \u1 (\lambda e. \v1)  = \ & ( \ex2 ( \lambda \x1. \tr{\P1 [\x1]})  ) \u1 (\lambda e. \v1)   \tag{by~\eqref{def:DTerms-FO-ex}} \\
 = \ & ( \lambda e \phi. \exists ( \lambda \x1. \P2 [\x1] (x::e) \phi  )) \u1 (\lambda e. \v1)   \tag{by~\eqref{def:DLCnst-FO-ex}} \\
\bconv \ & ( \lambda  \phi. \exists ( \lambda \x1. \P2 [\x1] (x::\u1) \phi  ))  (\lambda e. \v1)  \notag \\
\bconv \ & \exists ( \lambda \x1. \P2 [\x1] (x::\u1) (\lambda e. \v1)  )  \notag \\
= \ & \exists ( \lambda \x1. \P1 [\x1] \land \v1  )  \tag{by I.H.} \\
= \ & \exists ( \lambda \x1. \P1 [\x1]) \land \v1    \tag{$\x1 \notin FV(\v1)$} 
\end{align}

\end{itemize}
\end{proof}


%%%%%%%%%%%%%%%%%%%%%%%%%%%%%%%%%%%%%%%%%%%%%%%%%%%%%%%%%%%%%%%%%%%%%%%%%%%%%%%%%%%%%
%%%%%%%%%%%%%%%%%%%%%%%%%%%%%%%%%%%%%%%%%%%%%%%%%%%%%%%%%%%%%%%%%%%%%%%%%%%%%%%%%%%%%
\begin{corollary} \label{cor:DL-FO-handtop} For all propositions $\t1$ of type $o$, and for all terms $\u1$ of type $\gamma$, the following folds: 
\begin{align}\tr{\t1} \u1 (\lambda e. \top) \logeq  \t1 \notag  %\label{eq:DL-FO-handtop}
\end{align}
where $e$ is a variable of type $\gamma$.
 \label{DL-FO-handtop}
\end{corollary}
%%%%%%%%%%%%%%%%%%%%%%%%%%%%%%%%%%%%%%%%%%%%%%%%%%%%%%%%%%%%%%%%%%%%%%%%%%%%%%%%%%%%%
\begin{proof} Take $\v1$ equal to $\top$ in Proposition~\ref{DL-FO-handv}. Then
$$\tr{\t1} \u1 (\lambda e. \top) = \t1 \land \top \logeq \t1$$
\end{proof}


\begin{definition} A dynamic proposition $\t2$ is true in a model $\mathcal{M}$, denoted
$\mathcal{M}~\models_{dyn}~\t2$, if and only if $\mathcal{M}$ $\models$ $\t2 u (\lambda e. \top)$ for every $u$ of type $\gamma$. \label{def:dyntrueinmodel}
\end{definition}

\begin{theorem}[Conservation] A proposition $\t1$ is true in a model $\mathcal{M}$  if and only if its dynamic variant $\tr{\t1}$ is true in the same model: 
 $$\mathcal{M} \models \t1 \ \text{iff} \ \mathcal{M} \models_{dyn} \tr{\t1}$$
 \label{th:conservation:FO}
\end{theorem}
\begin{proof}

If $\mathcal{M} \models \t1$, then, by Corollary~\ref{cor:DL-FO-handtop}, $\mathcal{M} \models \tr{\t1} \u1 (\lambda e.\top)$. Therefore, by Definition~\ref{def:dyntrueinmodel}, $\mathcal{M} \models_{dyn} \tr{\t1}$.

If $\mathcal{M} \models_{dyn} \tr{\t1}$, then, by Definition~\ref{def:dyntrueinmodel}, $\mathcal{M} \models \tr{\t1} \u1 (\lambda e.\top)$. Therefore, by Corollary~\ref{cor:DL-FO-handtop}, $\mathcal{M} \models \t1$.
\end{proof}

The conservation theorem proves that a static proposition and its dynamic version defined in this section hold in the same models.



\section{From static to dynamic meaning} \label{sec:examples}

\subsection{Dynamization of static interpretations}

In order to obtain a dynamic meaning in the {\FullName} from a static meaning of a word without linguistic side-effects, one should apply the bar-function introduced in Definition~\ref{def:DTerms-FO} to the static interpretation. The function modifies the structure of atomic, negated and existentialy quantified formulas; as well as of a conjunction of formulas and ignores other subterms of the term, as formalized in Definition~\ref{def:bar-propagation}. The last four rules in~\ref{def:bar-propagation} are exactly the rules from~\ref{def:DTerms-FO}. 

\begin{definition}[] \label{def:bar-propagation} 
Let $\P1$ be a term of type $(\iota_1 \rightarrow \dots \rightarrow \iota_n \rightarrow o)$, $\A1$ and $\B1$ be terms of type $o$, $\t1_1, \dots, \t1_n$ and $\x1$ be terms of type $\iota$, $M$ and $N$ be any terms, and $y$ be a variable. For each term $t$, its dynamic equivalent $\tr{t}$ is constructed according to the following rules:
\begin{subequations}
\begin{align}
\tr{ y} \defeq \ & y  \label{def:bar-var}\\
\tr{ \lambda y. M } \defeq \ & \lambda y. \tr{M} \label{def:bar-abs}\\
\tr{ M N  } \defeq \ & \tr{M} \ \tr{N} \label{def:bar-app} \\
\tr{ \P1 \t1_1 \dots \t1_n } \defeq \ & \lambda e \phi . \P1 \t1_1 \dots \t1_n \land \phi e  \label{def:bar-P}\\
\tr{ \neg \A1  } \defeq \ & \n2 \tr{\A1}  \label{def:bar-neg}\\
\tr{ \A1 \land \B1} \defeq \ & \tr{\A1} \cnj2 \tr{\B1}  \label{def:bar-cnj}\\
\tr{\exists( \lambda \x1. \P1 [\x1])} \defeq \ & \ex2( \lambda  \x1. \tr{\P1 [\x1]})  \label{def:bar-ex} 
\end{align}
\end{subequations} 
\end{definition}

Rule~\ref{def:bar-P} is very important, as this is the only rule that introduces the abstraction over a continuation $\phi$ and a context $e$. Since this rule defines the dinamization of an atomic formula, one should apply the $\eta$-conversion to non-logical constants in the static lexical interpretation before dynamizing them.
Example~\ref{ex:donkey} illustrates this on the interpretation of \txt{donkey}.

\begin{example} \label{ex:donkey}
Given a static interpretation $\I{donkey}$ of \txt{donkey} consisting of a single constant of type $(\iota \rightarrow o)$, as shown in~\eqref{eq:cdonkey}.
In order to obtain its dynamic version, we first apply the $\eta$-conversion to it: 
\begin{align}
\I{donkey} = \ & \cdonkey \label{eq:cdonkey} \\
\econv \ & \lambda x. \cdonkey x %\label{eq:etacdonkey}
\end{align}
Now the bar-function can be applied:
\begin{align}
\tr{\I{donkey}} = \ & \tr{ \lambda x. \cdonkey x } \notag \\
 = \  & \lambda x. \tr{  \cdonkey x } \tag{by~\eqref{def:bar-abs}} \\
  = \  & \lambda x.  \lambda e \phi.  \cdonkey x \land \phi e \tag{by~\eqref{def:bar-P}} 
\end{align}

\end{example}

When a term is complex, Definition~\ref{def:bar-propagation} can be applied directly to it. This is due to the fact that even if the static interpretation contains non-logical constants, they are guaranteed to be applied to the corresponding arguments within the body of the term. Next example shows how the dinamic meaning for \txt{owns} can be systematically obtained.
\begin{example} 
Given the static interpretation for the verb \I{owns}
\begin{align}
\I{owns} = \  \lambda \Y1 \X1. \X1 ( \lambda \x1. \Y1 (\lambda \y1.  \textbf{o}  \x1 \y1 )) \notag %\label{eq:cowns} \\
\end{align}
its dynamic equivalent is obtained as follows:
\begin{align}
\tr{\I{owns}} = \ & \tr{\lambda \Y1 \X1. \X1 ( \lambda \x1. \Y1 (\lambda \y1.  \textbf{o}  \x1 \y1 ))}  \notag \\
= \ &  \lambda \Y1 \X1. \tr{ \X1 ( \lambda \x1. \Y1 (\lambda \y1.  \textbf{o}  \x1 \y1 ))} \tag{by~\eqref{def:bar-abs}} \\
= \ &  \lambda \Y1 \X1. \tr{ \X1 } \ \tr{ ( \lambda \x1. \Y1 (\lambda \y1.  \textbf{o}  \x1 \y1 ))}  \tag{by~\eqref{def:bar-app}} \\
= \ &  \lambda \Y1 \X1.  \X1   ( \lambda \x1. \tr{ \Y1  (\lambda \y1.  \textbf{o}  \x1 \y1 ))}  \tag{by~\eqref{def:bar-var} and~\eqref{def:bar-abs}} \\
= \ &  \lambda \Y1 \X1.  \X1   ( \lambda \x1. \tr{ \Y1 } \ \tr{ (\lambda \y1.  \textbf{o}  \x1 \y1 )})  \tag{by~\eqref{def:bar-app}} \\
= \ &  \lambda \Y1 \X1.  \X1   ( \lambda \x1.  \Y1   (\lambda \y1. \tr{ \textbf{o}  \x1 \y1 }))  \tag{by~\eqref{def:bar-var} and~\eqref{def:bar-abs}}\\
= \ &  \lambda \Y1 \X1.  \X1   ( \lambda \x1.  \Y1   (\lambda \y1. \lambda \phi e. \textbf{o}  \x1 \y1 \land \phi e)) \tag{by~\eqref{def:bar-P}} 
\end{align}

\end{example}

\TODOK{Explain about meanings of words with linguistic side effects.}


\subsection{Computing meaning of a donkey sentence}

Tables~\ref{tbl:stat-FO-donkey} and~\ref{tbl:dyn-FO-donkey} show respectively static and dynamic interpretations for the lexical items in the donkey sentence~\eqref{donkey-every-FO}:
\enumsentence{ \txt{Every farmer who owns a donkey beats it.} \label{donkey-every-FO}}

Observe that the type of every dynamic term is analogous to its static type. The only difference is that each atomic type of a dynamic term is dynamized according to Definition~\ref{def:DTypes-FO}. Moreover dynamic interpretations structurally resemble static interpretations. This is achieved by applying the rules from Definition~\ref{def:DLCnst-FO} to original static interpretations from Table~\ref{tbl:stat-FO-donkey} in order to obtain their dynamic equivalents shown in Table~\ref{tbl:dyn-FO-donkey}. The only exception is the interpretation of the pronoun \txt{it}, which is an unconventional lexical item: there is an anaphor to be resolved. While its static interpretation was uncertain, its dynamic representation integrates the selection function $\selK$ responsible for solving anaphora.

%\footnote{Here and further on, dynamic interpretations of unconventional lexical items are marked with tilde.}

\begin{table}
\begin{tabular}{ l l l l }
  Lexical item &  Static type & Static interpretation \\
  \hline
  \\
  \txt{farmer} &  ${\iota} \rightarrow {o}$ &  $\cfarmer$   \\
  \txt{donkey} &  ${\iota} \rightarrow {o}$ &  $\cdonkey$   \\
  \txt{owns} & $((\iota \rightarrow o) \rightarrow o) \rightarrow ((\iota \rightarrow o) \rightarrow o) \rightarrow {o}$  & $ \lambda \Y1 \X1. \X1 ( \lambda \x1. \Y1 (\lambda \y1.  \textbf{o}  \x1 \y1 ))$ \\
    \txt{beats} & $((\iota \rightarrow o) \rightarrow o) \rightarrow ((\iota \rightarrow o) \rightarrow o) \rightarrow {o}$  & $ \lambda \Y1 \X1. \X1 ( \lambda \x1. \Y1 (\lambda \y1.  \textbf{b}  \x1 \y1 ))$ \\
   \txt{every} & $({\iota} \rightarrow {o}) \rightarrow ( ({\iota} \rightarrow {o}) \rightarrow {o}) $ & $\lambda \P1 \Q1. \forall (\lambda x. \P1 x \rightarrow \Q1 x ) $   \\
   \txt{a} & $({\iota} \rightarrow {o}) \rightarrow ( ({\iota} \rightarrow {o}) \rightarrow {o}) $ & $ \lambda \P1 \Q1. \ex1 (\lambda x.  \P1 x \cnj1 \Q1 x )$ \\
  \txt{who} & $( ( ({\iota} \rightarrow {o}) \rightarrow {o} ) \rightarrow o  )  \rightarrow (\iota \rightarrow o)  \rightarrow (\iota \rightarrow o) $ & $ \lambda \R1 \Q1 \x1. \Q1 \x1 \cnj1 \R1 (\lambda \P1. \P1 \x1) $  \\
   \txt{it} & $ ({\iota} \rightarrow {o}) \rightarrow {o} $  & $\lambda \P1. \P1 ?$ \\ 
 \end{tabular}
\caption{Static lexical interpretations.} \label{tbl:stat-FO-donkey}
\end{table}
\begin{table}
\begin{tabular}{ l l l l l}
  Lexical item & Dynamic type & Dynamic interpretation \\
  \hline
  \\
  \txt{farmer} &  $\tr{\iota} \rightarrow \tr{o}$ &  $\lambda \x1.\tr{\cfarmer\x1}$  \\
    \txt{donkey} &  $\tr{\iota} \rightarrow \tr{o}$ & $\lambda \x1. \tr{\cdonkey \x1}$  \\
   \txt{owns} & $((\tr{\iota} \rightarrow \tr{o}) \rightarrow \tr{o}) \rightarrow ((\tr{\iota} \rightarrow \tr{o}) \rightarrow \tr{o}) \rightarrow \tr{o}$  & $ \lambda \Y2 \X2. \X2 ( \lambda \x1. \Y2 (\lambda \y1.  \tr{\textbf{o}  \x1 \y1 } ))$ \\
      \txt{beats} & $((\tr{\iota} \rightarrow \tr{o}) \rightarrow \tr{o}) \rightarrow ((\tr{\iota} \rightarrow \tr{o}) \rightarrow \tr{o}) \rightarrow \tr{o}$  & $ \lambda \Y2 \X2. \X2 ( \lambda \x1. \Y2 (\lambda \y1.  \tr{\textbf{b}  \x1 \y1 } ))$ \\
   \txt{every} & $(\tr{\iota} \rightarrow \tr{o}) \rightarrow ( (\tr{\iota} \rightarrow \tr{o}) \rightarrow \tr{o}) $ & $\lambda \P2 \Q2. \all2 (\lambda \x1.  \P2 \x1  \impK2 \Q2 \x1 ) $ \\
    \txt{a} &  $(\tr{\iota} \rightarrow \tr{o}) \rightarrow ( (\tr{\iota} \rightarrow \tr{o}) \rightarrow \tr{o}) $  & $ \lambda \P2 \Q2. \ex2 (\lambda \x1.  \P2 \x1  \cnj2   \Q2 \x1 )$ \\
   \txt{who} & $( ( (\tr{\iota} \rightarrow \tr{o}) \rightarrow \tr{o} ) \rightarrow \tr{o}  )  \rightarrow (\tr{\iota} \rightarrow \tr{o})  \rightarrow (\tr{\iota} \rightarrow \tr{o}) $ & $\lambda \R2 \Q2 \x1. \Q2 \x1  \cnj2  \R2 (\lambda \P2. \P2 \x1 ) $\\
      \txt{it} & $ (\tr{\iota} \rightarrow \tr{o}) \rightarrow \tr{o} $ & $ \lambda \P2. \lambda e \phi. \P2 ( \selK_{it} e ) e \phi $ \\ 
   \end{tabular}
\caption{Dynamic lexical interpretations.} \label{tbl:dyn-FO-donkey}
\end{table}


According to the parse tree, shown in Figure~\ref{fig:donkey-copy}, the dynamic meaning of~\eqref{donkey-every-FO} can be computed by $\beta$-reducing the term~\eqref{term:donkey-dynamic}:
\begin{align}
\tr{\I{beats}}  \ \etr{\I{it}} (\tr{\I{every}} ( (\tr{\I{who}}  (\tr{\I{owns}}  (\tr{\I{a}} \ \tr{\I{donkey}} ) ) ) \tr{\I{farmer}}  )) \label{term:donkey-dynamic}
\end{align}
%The dynamic lexical interpretations (in compact form) for words in Sentence~\eqref{donkey-every} are shown in Table~\ref{tbl:donkey-lexical}. These interpretations comply with dynamization principles discussed in Sections~\ref{sec:StandardInterpretations} and~\ref{sec:ExceptionTriggers}. Since the meanings of the lexical items are taken in compact form, the computation of the meaning of the sentence is free of ``bureaucracy'' and more intuitive compared to the computation using the normalized lexical interpretations. 


\begin{figure}[h!]
 \centering
    \includegraphics[width=0.85\textwidth]{images/Donkey.pdf}
\caption{Syntactic parse tree of sentence \txt{Every farmer who owns a donkey beats it}.} \label{fig:donkey-copy}
\end{figure}

\begin{example}[Meaning of \txt{Every farmer who owns a donkey beats it}]

The meaning of the noun phrase \txt{a donkey} is computed by reducing the term $\tr{\I{a}} \ \tr{\I{donkey}} $:
\begin{align*}
 \tr{\I{a}} \ \tr{\I{donkey}}   = \ & (\lambda \P2 \Q2. \ex2 (\lambda \x1.  \P2 \x1 \cnj2 \Q2 \x1 ) ) \tr{\I{donkey}}   \\
 \bconv \ &  \lambda  \Q2. \ex2 (\lambda \x1.  \tr{\I{donkey}} \x1  \cnj2  \Q2 \x1 ) \\
 = \ &  \lambda  \Q2. \ex2 (\lambda \x1. (\lambda \x1.\tr{\cdonkey \x1}) \x1  \cnj2  \Q2 \x1 ) \\
\bconv \ &  \lambda  \Q2. \ex2 (\lambda \x1.  \tr{\cdonkey \x1} \cnj2  \Q2 \x1 )
\end{align*}



The term $ \tr{\I{owns}}  (\tr{\I{a}} \ \tr{\I{donkey}}) $ is $\beta$-reduced as follows:
\begin{align*}
\tr{\I{owns}}  (\tr{\I{a}} \ \tr{\I{donkey}} ) = \ & ( \lambda \Y2 \X2. \X2 ( \lambda \x1. \Y2 (\lambda \y1.  \tr{\cown  \x1 \y1} )))  (\tr{\I{a}} \ \tr{\I{donkey}} ) \\
\bconv \ &   \lambda \X2. \X2 ( \lambda \x1. \tr{\I{a}} \ \tr{\I{donkey}} (\lambda \y1.  \tr{\cown}  \x1 \y1 ))   \\
= \ &  \lambda \X2. \X2 ( \lambda \x1. ( \lambda  \Q2. \ex2 (\lambda \z1.  \tr{\cdonkey \z1}\cnj2 \Q2 \z1 ) ) (\lambda \y1.  \tr{\cown  \x1 \y1} ))   \\
\bconv \ &  \lambda \X2. \X2 ( \lambda \x1. \ex2 (\lambda \z1.  \tr{\cdonkey \z1}\cnj2  (\lambda \y1.  \tr{\cown  \x1 \y1 }) \z1   ))   \\
\bconv \ & \lambda \X2. \X2 ( \lambda \x1.  \ex2 (\lambda \z1. \tr{\cdonkey \z1}\cnj2  \tr{\cown  \x1 \z1 }))  
\end{align*}

The meaning of the relative clause \txt{who owns a donkey} is computed by $\beta$-reducing the term $\tr{\I{who}}  (\tr{\I{owns}}  (\tr{\I{a}} \ \tr{\I{donkey}} ))$:
\begin{align*}
 & \tr{\I{who}}  (\tr{\I{owns}}  (\tr{\I{a}} \ \tr{\I{donkey}} )) \\
 = \ & (\lambda \R2 \Q2 \y1.  \Q2 \y1 \cnj2  \R2 (\lambda \P2. \P2 \y1 ) ) (\tr{\I{owns}}  (\tr{\I{a}} \ \tr{\I{donkey}} )) \\
\bconv \ &  \lambda \Q2 \y1. \Q2 \y1 \cnj2  \tr{\I{owns}}  (\tr{\I{a}} \ \tr{\I{donkey}}) (\lambda \P2. \P2 \y1)   \\
= \ & \lambda \Q2 \y1. \Q2 \y1 \cnj2  ( \lambda \X2. \X2 ( \lambda \x1.  \ex2 (\lambda \z1.  \tr{\cdonkey \z1}\cnj2 \tr{\cown  \x1 \z1 } ) ) )  (\lambda \P2. \P2 \y1)   \\
\bconv \ & \lambda \Q2 \y1.  \Q2 \y1 \cnj2  ( \lambda \P2. \P2 \y1) ( \lambda \x1.  \ex2 (\lambda \z1.  \tr{\cdonkey \z1}\cnj2 \tr{\cown  \x1 \z1 } ) )     \\
\bconv \ & \lambda \Q2 \y1. \Q2 \y1 \cnj2  (   \lambda \x1.  \ex2 (\lambda \z1.   \tr{\cdonkey \z1}\cnj2 \tr{\cown  \x1 \z1 } )    \y1)   \\
\bconv \ & \lambda \Q2 \y1.  \Q2 \y1 \cnj2   \ex2 (\lambda \z1.  \tr{\cdonkey \z1}\cnj2 \tr{\cown  \y1 \z1 } ) 
\end{align*}

The meaning of \txt{farmer who owns a donkey} is computed by applying the resulting $\lambda$-term in the computation above to the interpretation of \txt{farmer}:
\begin{align*}
& (\tr{\I{who}}  (\tr{\I{owns}}  (\tr{\I{a}} \ \tr{\I{donkey}} ) ) )\tr{\I{farmer}} \\
 = \ & (\lambda \Q2 \y1.  \Q2 \y1 \cnj2   \ex2 (\lambda \z1.  \tr{\cdonkey \z1}\cnj2 \tr{\cown \y1 \z1}  )) \tr{\I{farmer}} \\
\bconv \ & \lambda  \y1.  \tr{\I{farmer}} \y1 \cnj2   \ex2 (\lambda \z1.  \tr{\cdonkey \z1}\cnj2 \tr{\cown \y1 \z1}  )   \\ 
\bconv \ & \lambda  \y1.  (\lambda \x1.\tr{\cfarmer\x1}) \y1 \cnj2   \ex2 (\lambda \z1.  \tr{\cdonkey \z1}\cnj2 \tr{\cown \y1 \z1}  )   \\ 
\bconv \ & \lambda  \y1.  \tr{\cfarmer \y1} \cnj2   \ex2 (\lambda \z1.  \tr{\cdonkey \z1}\cnj2 \tr{\cown \y1 \z1} )   
\end{align*}

The meaning of the noun phrase \txt{every farmer who owns a donkey} is computed by applying the interpretation of \txt{every} to the interpretation of \txt{farmer who owns a donkey}: 
\begin{align}
& \tr{\I{every}}  ((\tr{\I{who}}  (\tr{\I{owns}}  (\tr{\I{a}} \ \tr{\I{donkey}} ) )) \tr{\I{farmer}}  )  \notag \\
= \ & (\lambda \P2 \Q2. \all2 (\lambda \x1.  \P2 \x1 \impK2 \Q2 \x1 ) ) ((\tr{\I{who}}  (\tr{\I{owns}}  (\tr{\I{a}} \ \tr{\I{donkey}} ) )) \tr{\I{farmer}}  )  \notag \\
\bconv \ & \lambda \Q2. \all2 (\lambda \x1.  ((\tr{\I{who}}  (\tr{\I{owns}}  (\tr{\I{a}} \ \tr{\I{donkey}} ) )) \tr{\I{farmer}}  ) \x1 \impK2 \Q2 \x1 )   \notag \\
= \ & \lambda \Q2. \all2 (\lambda \x1.  (\lambda  \y1.  \tr{\cfarmer \y1} \cnj2   \ex2 (\lambda \z1. \tr{\cdonkey \z1} \cnj2  \tr{\cown \y1 \z1}  )   ) \x1 \impK2 \Q2 \x1)    \notag \\
\bconv \ & \lambda \Q2. \all2 (\lambda \x1.  ( \tr{\cfarmer \x1} \cnj2   \ex2 (\lambda \z1.  \tr{\cdonkey \z1} \cnj2  \tr{\cown \x1 \z1}  ) )  \impK2 \Q2 \x1)   \label{eq:everyfarmerwhoownsadonkey}
\end{align}

The meaning of the verb phrase \txt{beats it} is computed as follows:
\begin{align}
\tr{\I{beats}}  \etr{\I{it}}  = \ &  (\lambda \Y2 \X2. \X2 ( \lambda \x1. \Y2 (\lambda \y1.  \tr{\cbeat \x1 \y1} ))) \etr{\I{it}} \notag  \\
\bconv \ &  \lambda  \X2. \X2 ( \lambda \x1.  \etr{\I{it}}  (\lambda \y1.  \tr{\cbeat \x1 \y1} ))\notag  \\
= \ & \lambda  \X2. \X2 ( \lambda \x1.  ( \lambda \P2. \lambda e \phi. \P2 ( \selK_{it} e ) e \phi)    (\lambda \y1.  \tr{\cbeat \x1 \y1} ))\notag  \\
\bconv \ & \lambda  \X2. \X2 ( \lambda \x1.   \lambda e \phi.  (\lambda \y1.  \tr{\cbeat \x1 \y1} ) ( \selK_{it} e ) e \phi  )\notag  \\
\bconv \ & \lambda  \X2. \X2 ( \lambda \x1.   \lambda e \phi.  \tr{\cbeat \x1 ( \selK_{it} e )} e \phi  ) \label{eq:beatsit}
\end{align}

Finally, the meaning of the sentence is computed by applying the interpretation~\eqref{eq:beatsit} to the interpretation~\eqref{eq:everyfarmerwhoownsadonkey}:
\begin{flalign}
& \tr{\I{beats}}  \ \etr{\I{it}} (\tr{\I{every}}  ((\tr{\I{who}}  (\tr{\I{owns}}  (\tr{\I{a}} \ \tr{\I{donkey}} ) ) ) \tr{\I{farmer}}  )) & \notag \\
= \ & (\lambda  \X2. \X2 ( \lambda \x1.   \lambda e \phi.  \tr{\cbeat \x1 ( \selK_{it} e )} e \phi  ) ) (\tr{\I{every}}  ((\tr{\I{who}}  (\tr{\I{owns}}  (\tr{\I{a}} \ \tr{\I{donkey}} ) ) ) \tr{\I{farmer}}  )) &  \notag \\
\bconv \ &  (\tr{\I{every}}  ((\tr{\I{who}}  (\tr{\I{owns}}  (\tr{\I{a}} \ \tr{\I{donkey}} ) ) ) \tr{\I{farmer}}  )) ( \lambda \x1.   \lambda e \phi.  \tr{\cbeat \x1 ( \selK_{it} e )} e \phi  )   &  \notag \\
= \ & (\lambda \Q2. \all2 (\lambda \x1.  ( \tr{\cfarmer \x1} \cnj2   \ex2 (\lambda \z1.  \tr{\cdonkey \z1} \cnj2  \tr{\cown \x1 \z1}  ) )  \impK2 \Q2 \x1) ) ( \lambda \x1.   \lambda e \phi.  \tr{\cbeat \x1 ( \selK_{it} e )} e \phi  )   &  \notag \\
\bconv \ &  \all2 (\lambda \x1.  ( \tr{\cfarmer \x1} \cnj2   \ex2 (\lambda \z1.  \tr{\cdonkey \z1} \cnj2  \tr{\cown \x1 \z1}  ) )  \impK2 ( \lambda \x1.   \lambda e \phi.  \tr{\cbeat \x1 ( \selK_{it} e )} e \phi  )   \x1)  ) &  \notag \\
\bconv \ &  \all2 (\lambda \x1.  ( \tr{\cfarmer \x1} \cnj2   \ex2 (\lambda \z1.  \tr{\cdonkey \z1} \cnj2  \tr{\cown \x1 \z1}  ) )  \impK2 (  \lambda e \phi.  \tr{\cbeat \x1 ( \selK_{it} e )} e \phi ) ) &    \label{eq:donkey-compact-meaning}
\end{flalign}

\end{example}

Formula~\eqref{eq:donkey-compact-meaning} is the dynamic interpretation of Sentence~\eqref{donkey-every-FO}. 
Due to the compact forms of dynamic (sub-)terms, the computation and the resulting term are as intuitive as if static interpretations were used. However, the dynamic interpretation is advantageous. Particularly for donkey sentences, it not only provides means for compositionally solving the pronominal anaphora, but also avoids the quantifier scope problem. 

To illustrate that the quantifier scope problem is avoided two analyses are given below. We apply these analyses to 
term~\eqref{eq:donkey-compact-meaning-2} obtained from~\eqref{eq:donkey-compact-meaning}  using Remark~\ref{remark:impforall}. Interpretations~\eqref{eq:donkey-compact-meaning-2} and~\eqref{eq:donkey-compact-meaning} are logically equivalent.
\begin{align}
\n2 \ex2 (\lambda \x1.  \tr{\cfarmer \x1} \cnj2   \ex2 (\lambda \z1.  \tr{\cdonkey \z1}\cnj2 \tr{\cown \x1 \z1}  )  \cnj2  \n2 ( \lambda e \phi. \tr{\cbeat \x1  ( \selK_{it} e )} e \phi  ))   \label{eq:donkey-compact-meaning-2}
\end{align}

In the first analysis we simply apply Proposition~\ref{thrm:scopeextension-G} to~\eqref{eq:donkey-compact-meaning-2} and in the second analysis we examine the structure of the term and its dynamic subterms. %The third analysis normalizes the term by performing step-by-step $\beta$-reductions.

\subsection*{Analysis 1}


According to Proposition~\ref{thrm:scopeextension-G}, since $\z1 \notin FV(\n2 \tr{\cbeat}  \x1 (( \selK_{it} e ) e \phi))$, formula~\eqref{eq:donkey-compact-meaning-2} is $\beta$-equivalent to the following formula:
\begin{align}
\n2 \ex2 (\lambda \x1.  \tr{\cfarmer \x1} \cnj2   \ex2 (\lambda \z1.  \tr{\cdonkey \z1}\cnj2 \tr{\cown \x1 \z1}    \cnj2  \n2  ( \lambda e \phi. \tr{\cbeat \x1 ( \selK_{it} e )} e \phi ) )) \label{eq:donkey-compact-meaning-3}
\end{align}

Finally, by Remark~\eqref{remark:impforall}, formula~\eqref{eq:donkey-compact-meaning-3} can be represented in a more familiar way:
\begin{align}
 \all2 (\lambda \x1.  \tr{\cfarmer \x1} \impK2   \all2 (\lambda \z1. ( \tr{\cdonkey \z1}\cnj2 \tr{\cown \x1 \z1} )   \impK2    ( \lambda e \phi. \tr{\cbeat \x1 ( \selK_{it} e )} e \phi ) )) 
 \label{eq:donkey-compact-meaning-4}
\end{align}


\subsection*{Analysis 2}


Although it is sufficient to apply Proposition~\ref{thrm:scopeextension-G} to see that indeed~\eqref{eq:donkey-compact-meaning} is the desirable interpretation of the donkey sentence, it is worth to inspect the term itself for gaining more intuition why this is so.

In order to see that the proposition $\tr{\cbeat \x1 ( \selK_{it} e )}$ enters inside the scope of the existential quantifier related to \txt{a donkey} in~\eqref{eq:donkey-compact-meaning-2}, it is important to understand how the continuation of the subterm $\tr{\cfarmer \x1} \cnj2 \ex2 (\lambda \z1.  \tr{\cdonkey \z1}\cnj2 \tr{\cown \x1 \z1}  )$ ends up inside the scope of the existential quantifier. 

Recall formula~\eqref{def:DLCnst-FO-cnj}, which defines the dynamic conjunction in a way that its second argument $\B2$ is a subterm of the continuation $(\lambda e. \B2 e \phi)$ of the first argument $\A2$. As a result, the continuation of the term $\A2 \cnj2 \B2$ itself is deep inside the formula. Particularly, in the subterm $ \tr{\cdonkey \z1}\cnj2 \tr{\cown \x1 \z1} $, the term $\tr{\cown \x1 \z1}$ is a subterm of the continuation of $\tr{\cdonkey \z1}$. Consequently, the continuation $\phi$ of the term $\tr{\cdonkey \z1}\cnj2 \tr{\cown \x1 \z1}$ is deep inside the formula, as shown in~\eqref{eq:app1}.  
Consequently, arguments of $ \ex2 (\lambda \z1.  \tr{\cdonkey \z1}\cnj2 \tr{\cown \x1 \z1}  ) $ will end up getting bound by the existential quantifier.
\begin{align}
\tr{\cdonkey \z1}\cnj2 \tr{\cown \x1 \z1} \bred  \lambda e \phi. \cdonkey \z1 \cnj1 ( \cown \x1  \z1 \cnj1 \phi e)  \label{eq:app1}
\end{align}
 Now, consider Formula~\eqref{def:DLCnst-FO-ex}. $\ex2$ applied to $ ( \lambda \z1. \tr{\cdonkey \z1}\cnj2 \tr{\cown \x1 \z1})$ results in a formula shown in~\eqref{eq:app2}. Note that the continuation remains deep inside the formula and, what is the most important, it occurs \emph{inside the scope of the existential quantifier}:
 \begin{align}
 \ex2 (\lambda \z1.  \tr{\cdonkey \z1}\cnj2 \tr{\cown \x1 \z1}  ) \bred  \lambda e \phi. \ex1 ( \lambda \z1. \cdonkey \z1  \cnj1 ( \cown \x1 \z1 \cnj1 \phi (\z1::e)) )  \label{eq:app2}
 \end{align}

Consider now the dynamic conjunction $ \tr{\cfarmer \x1}  \cnj2 \ex2 (\lambda \z1.  \tr{\cdonkey \z1}\cnj2 \tr{\cown \x1 \z1}  )$. By~\eqref{def:DLCnst-FO-cnj}, the whole term $\ex2 (\lambda \z1.  \tr{\cdonkey \z1}\cnj2 \tr{\cown \x1 \z1}  ) $ is a subterm of the continuation of $\tr{\cfarmer \x1} $. The term normalizes as~\eqref{eq:app3}. Importantly, the continuation $\phi$ of the term continues to be deep inside the formula and \emph{inside} \emph{the scope of the existential quantifier}.
\begin{align}
 \tr{\cfarmer \x1}  \cnj2 \ex2 (\lambda \z1.  \tr{\cdonkey \z1}\cnj2 \tr{\cown \x1 \z1}  )  \bred  \lambda e \phi. \cfarmer \x1  \cnj1 \ex1 ( \lambda \z1. \cdonkey \z1  \cnj1 ( \cown \x1  \z1 \cnj1 \phi  (\z1::e) ) )  \label{eq:app3}
\end{align}

Next the subterms $(\tr{\cfarmer \x1} \cnj2  \ex2 (\lambda \z1.  \tr{\cdonkey \z1}\cnj2 \tr{\cown \x1 \z1}  ) )$ and $ \n2 ( \lambda e \phi. \tr{\cbeat \x1  ( \selK_{it} e )} e \phi  ) $ are dynamically conjoint. Similarly to above, by~\eqref{def:DLCnst-FO-cnj}, the term $ \n2 ( \lambda e \phi. \tr{\cbeat \x1  ( \selK_{it} e )} e \phi  )$ is a subterm of the continuation of the term $(\tr{\cfarmer \x1}  \cnj2   \ex2 (\lambda \z1.  \tr{\cdonkey \z1}\cnj2 \tr{\cown \x1 \z1}  ) ) $.  As shown in~\eqref{eq:app3}, the continuation $\phi$ is under the scope of the existential quantifier, and, therefore, this is the very reason why the formula interpreting \txt{beats it} goes (via $\beta$-reduction) inside the scope of the existential quantifier that binds the variable interpreting \txt{a donkey}. The normalized resulting term is shown in~\eqref{eq:app4}:
\begin{align}
& (\tr{\cfarmer \x1}  \cnj2   \ex2 (\lambda \z1.  \tr{\cdonkey \z1}\cnj2 \tr{\cown \x1 \z1}  ) )  \cnj2  \n2 ( \lambda e \phi. \tr{\cbeat \x1  ( \selK_{it} e )} e \phi  )  \notag \\
 \bred \ &  \lambda e \phi. \cfarmer \x1  \cnj1 \ex1 ( \lambda \z1. \cdonkey \z1  \cnj1 ( \cown \x1  \z1 \cnj1 ( \n1 \cbeat  \x1 ( \selK_{it} e ) \cnj1 \phi  (\z1::e) ) ))\label{eq:app4}
\end{align}
Since, due to the definition of $\cnj2$,  $\n1 \cbeat  \x1 ( \selK_{it} e )$ entered deep inside the formula $   \lambda e \phi. \cfarmer \x1  \cnj1 \ex1 ( \lambda \z1. \cdonkey \z1  \cnj1 ( \cown \x1  \z1 \cnj1 \phi  (\z1::e) ) )$, the selection function responsible for solving anaphor for \txt{it} can become bound with the quantified variable resulted from interpreting \txt{a donkey} without violating any law of logic. %In other words, pronoun \txt{it} can become bound to \txt{a donkey} and this is achieved by notion of $\beta$-reduction only. 

We continue with analizing $$\ex2 (\lambda \x1.  \tr{\cfarmer \x1} \cnj2   \ex2 (\lambda \z1.  \tr{\cdonkey \z1}\cnj2 \tr{\cown \x1 \z1}  )  \cnj2  \n2 ( \lambda e \phi. \tr{\cbeat \x1  ( \selK_{it} e )} e \phi  )) $$ which is, according to~\eqref{eq:app4}, $\beta$-equivalent to $$\ex2 (\lambda \x1. \lambda e \phi. \cfarmer \x1  \cnj1 \ex1 ( \lambda \z1. \cdonkey \z1  \cnj1 ( \cown \x1  \z1 \cnj1 ( \n1 \cbeat  \x1 ( \selK_{it} e ) \cnj1 \phi  (\z1::e) ) ))) $$ By~\eqref{def:DLCnst-FO-ex}, it is equivalent to the following term:
\begin{align}
\lambda e \phi. \ex1 (\lambda \x1.  \cfarmer \x1  \cnj1 \ex1 ( \lambda \z1. \cdonkey \z1  \cnj1 ( \cown \x1  \z1 \cnj1 ( \n1 \cbeat  \x1 ( \selK_{it} e ) \cnj1 \phi  (\z1::\x1::e) ) ))) \label{eq:app7}
\end{align}
Now we only need to apply $\n2$ to the term~\eqref{eq:app7} using Equation~\eqref{def:DLCnst-FO-neg}. This results in the final desired interpretation:
\begin{align}
& \lambda e \phi. \n1 \ex1 (\lambda \x1.  \cfarmer \x1  \cnj1 \ex1 ( \lambda \z1. \cdonkey \z1  \cnj1 ( \cown \x1  \z1 \cnj1 ( \n1 \cbeat  \x1 ( \selK_{it} e )  ) ))) \cnj1 \phi e \notag
\end{align}

\TODOK{Finish this example... Context $(\z1::\x1::e)$ was erased ...  mention DRT accessibility constraints...}

% consider commented text below!!!
\comments{

However, this is not always the case. For example, assuming accessibility constraint requirements of DRT, the individuals introduced by quantifiers in Sentence~\eqref{donkey-every-2006} should not be accessible for anaphoric triggers outside of the sentence. However, they clearly should be accessible for anaphoric pronouns within the sentence. 
\enumsentence{ \txt{Every farmer who owns a donkey beats it.} \label{donkey-every-2006}}
This accessibility constraint can also be implemented in the continuation-based dynamic approach. For example, lexical items of~\eqref{donkey-every-2006} can be assigned meanings, shown in Table~\ref{tbl:donkey-lexical-2006}, that lead to the desirable interpretation of the sentence, as demonstrated in Example~\ref{ex:donkeysentence}. %Since the lexical interpretations are dynamic, the resulting dynamic meaning of the donkey sentence does not suffer the drawbacks of the static meaning.

Note that in Equation~\eqref{eq:everyfarmerwhoownsadonkey-nf-2006} the context containing all the individuals with their properties collected during the computation is locally passed to the formula $\Q2$. The continuation $\phi$ receives only the context $e$ that is passed to the term as an argument; therefore, all individuals collected during the computation of the meaning of \txt{every farmer who owns a donkey} are not available outside the logical formula interpreting this phrase.


Note that the second argument of $\cbeat$, standing for the anaphoric pronoun, is not a free dummy variable, but a term $( \selK_{it} (\y1 :: \x1 :: e))$. This term consists of the selection function $\selK$ that takes as argument a context containing the available individuals ``collected'' during the computation. The function $\selK$, which implements an anaphora resolution algorithm, selects a required individual from the context, which, in the current case, is the individual $\y1$. This  leads to the final dynamic meaning~\eqref{eq:donkey-nf-meaning-3-2006} of Sentence~\eqref{donkey-every-2006}:
\begin{align}
\lambda e \phi. \forall ( \lambda \x1.  \cfarmer \x1  \rightarrow  \forall ( \lambda \y1.  ( \cdonkey \y1 \cnj1  \cown \x1 \y1 ) \rightarrow   \cbeat \x1 \y1  ) )  \cnj1 \phi e  \label{eq:donkey-nf-meaning-3-2006}
\end{align}

Moreover, the formula $\cbeat \x1 \y1 $ is within the scope of the quantifier binding the variable $y$, exactly as desired.  Furthermore, the quantifier binding $y$ has been changed during the computation from existential to universal. This is achieved by employing a continuation-passing technique. 

Finally, in accordance with DRT's accessibility constraint, the individuals bound by quantifiers are not accessible outside the sentence, because the continuation $\phi$ of the term simply receives non-updated context $e$ as an argument.






 Taking these dynamic interpretations to compute the meaning of Sentence~\eqref{donkey-every-FO}, term~\eqref{eq:FO-donkey-int0} $\beta$-reduces to term~\eqref{eq:FO-donkey-int1}, which normalizes to~\eqref{eq:FO-donkey-int1-equiv}:
\begin{align}
& \tr{\I{beats}}  \ \etr{\I{it}} (\tr{\I{every}}  ((\tr{\I{who}}  (\tr{\I{owns}}  (\tr{\I{a}} \ \tr{\I{donkey}} ) ) ) \tr{\I{farmer}}  ))  \label{eq:FO-donkey-int0} \\
\bred \ &  \all2 (\lambda \x1. (\tr{\cfarmer} \x1  \cnj2     \ex2 (\lambda \z1.   \tr{\cdonkey}   \z1  \cnj2      \tr{\textbf{o}}  \x1  \z1 ))    \impK2    (\lambda e \phi. \tr{\textbf{b}}  \x1 ( \selK_{it} e ) e \phi ) )  \label{eq:FO-donkey-int1} \\
\bred \ & \lambda e \phi. \forall (\lambda \x1. \cfarmer \x1  \rightarrow \forall  (\lambda \z1.  (\cdonkey  \z1 \cnj1 \cown  \x1 \z1 )   \rightarrow  \cbeat  \x1 ( \selK_{it}  (\upii{\x1}{\upii{\z1}{e}}) )))\cnj1 \phi e  \label{eq:FO-donkey-int1-equiv}
\end{align}


Resulting term~\eqref{eq:FO-donkey-int1-equiv} is equivalent to~\eqref{eq:donkey-nf-meaning-2-2006} obtained in framework {\G} interpretations. Indeed, framework {\GN} is equivalent to de Groote's~\cite{deGroote:2006:Towards-a-Montagovian-Account-of-Dynamics} framework {\G}. However, it is advantageous over {\G} due to the compact notations for dynamic terms. These notations significantly systematize the framework and make the interpretations more concise and intuitive. Moreover, the systematic translations of static terms into dynamic terms makes it possible to prove a conservation result~\ref{th:conservation:FO} for {\GN}, that guarantees that static and dynamic interpretations are satisfied in the same models. 

}



\section{Comparison with related work} \label{sec:dpl}

Constants, variables, reference markers: $\mathsf{C}, \mathsf{V}, \mathsf{M}$

continuations : $\cont = 2^\env$ 

DPL formula vs extended  DPL formula.

\subsection{Dynamic Predicate Logic}

Groenendijk and Stokhof~\cite{xxx} introduced Dynamic Predicate Logic 
(DPL, for short) as a compositional alternative to theories such as
Kamp's DRT~\cite{xxx} or Heim's file change semantics~\cite{xxx}. 
In fact, DPL may be seen as the dynamic logic that underlies DRT.
Indeed, the relation between DPL and DRT is rather tight:
DPL provides the same truth conditions and the same antecedent accessibility
constraints as DRT.  For this reason, we restrict comparison with related
theories to a technical comparison between Groenendijk and Stokhof's dynamic
logic and ours. In particular, we will show how DPL can be faithfully embedded
in TDL.

The syntax of DPL is the one of ordinary first-order logic (without functional
symbols).  A term is defined to be either an individual constant or a variable.
An atomic formula, $\app{\app{\app{R}{t_1}}{\ldots}}{t_n}$, consists of a 
$n$-ary relational symbol, $R$, applied to $n$ terms,  $t_1 \ldots t_n$.
Then, formulas are built from atomic formulas by means of conjunction, negation,
and existential quantification.\footnote{We consider disjunction, implication
and universal quantification to be defined using de Morgan's laws.}

A model of DPL, $\mathscr{M} = \langle D, \mathcal{I}\, \rangle$, is a usual
first-order structure. It is made of an non-empty set of individuals, $D$, 
and a function, $\mathcal{I}$, that yields an interpretation of the constants 
and the relational symbols. Thus, for every individual constants $c$, 
$\mathcal{I}(c) \in D$, and for every $n$-ary relational symbol $R$,
$\mathcal{I}(R) \subset D^n$.

Let $\mathsf{C}$ be the set of individual constants, and $\mathsf{M}$ be
the set of variables.  In the sequel, we will call the elements of $\mathsf{M}$
\textit{reference markers} in order not to confuse them with the 
$\lambda$-variables.

The set of environments (a.k.a., assignements) is defined to be 
$\env = D^{\mathsf{M}}$. Thus, an environment is a function that assigns an 
individual to each reference marker. Given such an environement, say $h$,
the interpretation $[t]_h$ of a term $t$ is defined as follows:

\begin{align*}
[t]_h &=
\left\{\!
\begin{array}[c]{ll}
\mathcal{I} (t) & \mbox{if } t\in\mathsf{C}\\
h (t) & \mbox{if } t\in\mathsf{M}
\end{array}
\right.
\end{align*}

Given two environments $h, g \in \env$, and a reference marker $i\in\mathsf{M}$,
the relation $g[i]h$ is defined as follows:
$$
g[i]h \;\mbox{ iff }\;
\uquant{j\in\mathsf{M}}{\imp{j\not=i}{g(j)=h(j)}}
$$

Finally, the interpretation of a dynamic formula is defined to be a relation
between environments as follows. It also explains why DPL 

\begin{align*}
\dplsem{\app{\app{\app{R}{t_1}}{\ldots}}{t_n}} &= 
\{\langle g,h\rangle  :
h=g \,\wedge\, 
\langle 
[ t_1 ]_g,
\dots,
[ t_n ]_g
\rangle \in \mathcal{I}(R)
\}
\\
\dplsem{\conj{P}{Q}} &= 
\{\langle g,h\rangle  : 
\equant{k}{
\langle g,k\rangle\in \dplsem{P}
\,\wedge\,
\langle k,h\rangle\in \dplsem{Q}
}\}
\\
\dplsem{\nega{P}} &=
\{\langle g,h\rangle  : 
h=g \,\wedge\, 
\uquant{k}{\langle g,k \rangle\not\in\dplsem{P}}
\}
\\
\dplsem{\equant{i}{P}} &=
\{\langle g,h\rangle  : 
\equant{k}{
k[i]g \,\wedge\,
\langle k,h \rangle \in \dplsem{P}
}\} 
\end{align*}

Remark that the reference marker $i$ occurs free in the right-hand side of
the last equation.  This shows that reference markers are not quite variables
in the usual sense. It also explains why DPL has a problem of destructive 
assignment.

\subsection{Environments as left contexts}


\begin{align*}
\tdlsem{\eta}{\abs{e\phi}{\app{\app{\app{R}{t_1}}{\ldots}}{t_n}}} &= 
\{\langle g,H\rangle   :
g \in H \, \wedge \,
\langle 
\llbracket t_1 \rrbracket_{\eta[e{:=}g]},
\dots,
\llbracket t_n \rrbracket_{\eta[e{:=}g]}
\rangle \in \mathcal{I}(R)
\}
\\
\tdlsem{\eta}{\dconj{P}{Q}} &= 
\{\langle g,H\rangle  :
\langle g, 
\{h : \langle h, H\rangle \in \tdlsem{\eta}{Q}\}
\rangle\in\tdlsem{\eta}{P}
\}
\\
\tdlsem{\eta}{\dnega{P}} &=
\{\langle g,H\rangle  :
g \in H \, \wedge \,
\langle g,\env\rangle\not\in \tdlsem{\eta}{P}
\}
\\
\tdlsem{\eta}{\dynexists_{i} x.\, P} &=
\{\langle g,H\rangle  :
\equant{a}{\langle g[i{:=}a],H\rangle\in\tdlsem{\eta[x{:=}a]}{P}}
\} 
\end{align*}


\subsection{Syntactic and semantic translation}

\begin{align*}
\synlift{c}_e &= c\\
\synlift{x}_e &= x\\
\synlift{\mathbf{i}}_e &= \app{\mathsf{sel}_i}{e}\\
\end{align*}

\begin{align*}
\synlift{\app{\app{\app{R}{t_1}}{\ldots}}{t_n}} &= 
\abs{e\phi}{\conj{
\app{\app{\app{R}{
\synlift{t_1}_e
}}{\ldots}}{
\synlift{t_n}_e}
}{\app{\phi}{e}}}
\\
\synlift{\conj{P}{Q}} &= 
\dconj{\synlift{P}}{\synlift{Q}}
\\
\synlift{\nega{P}} &= \dnega{\synlift{P}}
\\
\synlift{\equant{\mathbf{i}}{P}} &=
\dynexists_{\mathbf{i}} x.\,\synlift{P[\mathbf{i}{:=}x]}
\end{align*}

$$
\semlift{\mathrm{R}} = \{\langle a,B\rangle : 
\equant{b}{\conj{b\in B}{\langle a,b\rangle\in R}}\}
$$

\begin{fact}\label{updateenvir}
$h[\mathbf{i}]g$ iff $\equant{a}{h=g[\mathbf{i}{:=}a]}$
\end{fact}

\begin{lemma}\label{substlemma1}
Let $t$ be a term, and let $x$ be a variable that does not occur in $t$.
For all reference marker $\mathbf{i}$, all assignment $g$, and all valuation
$\eta$, the following holds:
$$
\llbracket t \rrbracket_{\eta,g} =
\llbracket t[\mathbf{i}{:=}x] \rrbracket_{\eta[x{:=}(\app{g}{\mathbf{i}})],g}
$$
\begin{proof}
If $t$ is a constant or a reference marker different from $\mathbf{i}$,
the result is immediate.
If $t$ is a variable $y$, we have that $y\not= x$. Consequently,
$\app{\eta}{y} = \app{\eta[x{:=}(\app{g}{i})]}{y}$.
Finally, if $t$ is $\mathbf{i}$, we have
$\llbracket \mathbf{i} \rrbracket_{\eta,g} =
\app{g}{\mathbf{i}} = 
\app{\eta[x{:=}(\app{g}{i})]}{x} = 
\llbracket x \rrbracket_{\eta[x{:=}(\app{g}{\mathbf{i}})],g} =
\llbracket\mathbf{i}[\mathbf{i}{:=}x] \rrbracket_{\eta[x{:=}(\app{g}{\mathbf{i}})],g}
$.
\end{proof}
\end{lemma}

\begin{lemma}\label{substlemma2}
Let $P$ be an extended DPL formula, and let $x$ be a variable that does not 
occur in $P$.
For all reference marker $\mathbf{i}$, all pair of assignments 
$\langle g, h \rangle$, and all valuation
$\eta$, the following holds:
$$
\langle g,h\rangle \in \dplsem{\eta}{P}
\mbox{ if and only if }
\langle g,h\rangle \in 
\dplsem{\eta[x{:=}(\app{g}{\mathbf{i}})]}{P[\mathbf{i}{:=}x]}
$$
\begin{proof}
A straightforward induction on the structure of $P$, using 
Lemma~\ref{substlemma1} for the base case.
\end{proof}
\end{lemma}

\begin{lemma}\label{commutlemma}
Let $P$ be an extended DPL formula, and $\eta$ be a valuation.
Then, the following holds:
$$
\semlift{\dplsem{\eta}{P}} = \tdlsem{\eta}{\synlift{P}}
$$
\begin{proof}
By induction on the structure of $P$.

\mbox{}

\noindent
1. $P \equiv \app{\app{\app{R}{t_1}}{\ldots}}{t_n}$.
\begin{align*}
&\semlift{\dplsem{\eta}{\app{\app{\app{R}{t_1}}{\ldots}}{t_n}}}\\
&\quad=
\{\langle g,H\rangle : 
\equant{h}{\conj{h\in H}{\langle g,h\rangle\in 
\dplsem{\eta}{\app{\app{\app{R}{t_1}}{\ldots}}{t_n}}}}\}
\\
&\quad=
\{\langle g,H\rangle : 
\equant{h}{
\conj{h\in H}{
\conj{h=g}{ 
\langle 
\llbracket t_1 \rrbracket_{\eta,g},
\dots,
\llbracket t_n \rrbracket_{\eta,g}
\rangle \in \mathcal{I}(R)}
}}\}
\\
&\quad=
\{\langle g,H\rangle : 
\conj{g\in H}{
\langle 
\llbracket t_1 \rrbracket_{\eta,g},
\dots,
\llbracket t_n \rrbracket_{\eta,g}
\rangle \in \mathcal{I}(R)
}\}
\\
&\quad=
\tdlsem{\eta}{\app{\app{\app{R}{t_1}}{\ldots}}{t_n}}
\\
&\quad=
\tdlsem{\eta}{\synlift{\app{\app{\app{R}{t_1}}{\ldots}}{t_n}}}
\end{align*}

\noindent
2. $P \equiv \conj{P_1}{P_2}$.
\begin{align*}
&\semlift{\dplsem{\eta}{\conj{P_1}{P_2}}}\\
&\quad=
\{\langle g,H\rangle : 
\equant{h}{\conj{h\in H}{\langle g,h\rangle\in
\dplsem{\eta}{\conj{P_1}{P_2}}
}}\}
\\
&\quad=
\{\langle g,H\rangle : 
\equant{h}{\conj{h\in H}{
\equant{k}{
\conj{\langle g,k\rangle\in \dplsem{\eta}{P_1}}{
\langle k,h\rangle\in \dplsem{\eta}{P_2}}}}}\}
\\
&\quad=
\{\langle g,H\rangle : 
\equant{k}{
\conj{\langle g,k\rangle\in \dplsem{\eta}{P_1}}{
\equant{h}{ 
\conj{h\in H}{
\langle k,h\rangle\in \dplsem{\eta}{P_2}}}}}
\}
\\
&\quad=
\{\langle g,H\rangle : 
\equant{k}{
\conj{\langle g,k\rangle\in \dplsem{\eta}{P_1}}{
\langle k,H\rangle\in \semlift{\dplsem{\eta}{P_2}}}
}\}
\\
&\quad=
\{\langle g,H\rangle : 
\equant{k}{
\conj{k\in\{k : \langle k,H\rangle\in \semlift{\dplsem{\eta}{P_2}}\}}{
\langle g,k\rangle\in \dplsem{\eta}{P_1}}}
\}
\\
&\quad=
\{\langle g,H\rangle : 
\langle g,
\{k : \langle k,H\rangle\in \semlift{\dplsem{\eta}{P_2}}\}
\rangle\in \semlift{\dplsem{\eta}{P_1}}
\}
\\
&\quad=
\{\langle g,H\rangle : 
\langle g,
\{k : \langle k,H\rangle\in \tdlsem{\eta}{\synlift{P_2}}\}
\rangle\in \tdlsem{\eta}{\synlift{P_1}}
\}
\\
&\makebox[.92\linewidth][r]{(by induction hypothesis)}\\
&\quad=
\tdlsem{\eta}{\dconj{\synlift{P_1}}{\synlift{P_2}}}
\\
&\quad=
\tdlsem{\eta}{\synlift{\conj{P_1}{P_2}}}
\end{align*}

3. $P \equiv \nega{P_1}$.
\begin{align*}
&\semlift{\dplsem{\eta}{\nega{P_1}}}\\
&\quad=
\{\langle g,H\rangle : 
\equant{h}{\conj{h\in H}{\langle g,h\rangle\in
\dplsem{\eta}{\nega{P_1}}
}}\}\\
&\quad=
\{\langle g,H\rangle : 
\equant{h}{\conj{h\in H}{
\conj{h=g}{
\uquant{k}{\langle g,k \rangle\not\in\dplsem{\eta}{P_1}}}
}}\}\\
&\quad=
\{\langle g,H\rangle : 
\conj{g\in H}{
\uquant{k}{\langle g,k \rangle\not\in\dplsem{\eta}{P_1}}
}\}\\
&\quad=
\{\langle g,H\rangle : 
\conj{g\in H}{
\nega{\equant{k}{\conj{k\in\env}{\langle g,k \rangle\in\dplsem{\eta}{P_1}}}}
}\}\\
&\quad=
\{\langle g,H\rangle : 
\conj{g\in H}{
\langle g,\env \rangle\not\in\semlift{\dplsem{\eta}{P_1}}
}\}\\
&\quad=
\{\langle g,H\rangle : 
\conj{g\in H}{
\langle g,\env \rangle\not\in\tdlsem{\eta}{\synlift{P_1}}
}\}\\
&\makebox[.92\linewidth][r]{(by induction hypothesis)}\\
&\quad=
\tdlsem{\eta}{\dnega{\synlift{P_1}}}
\\
&\quad=
\tdlsem{\eta}{\synlift{\nega{P_1}}}
\end{align*}

4. $P \equiv \equant{\mathbf{i}}{P_1}$.
\begin{align*}
&\semlift{\dplsem{\eta}{\equant{\mathbf{i}}{P_1}}}\\
&\quad=
\{\langle g,H\rangle : 
\equant{h}{\conj{h\in H}{\langle g,h\rangle\in 
\dplsem{\eta}{\equant{\mathbf{i}}{P_1}}
}}\}\\
&\quad=
\{\langle g,H\rangle : 
\equant{h}{\conj{h\in H}{
\equant{k}{
\conj{k[\mathbf{i}]g}{
\langle k,h \rangle \in \dplsem{\eta}{P_1}}}
}}\}\\
&\quad=
\{\langle g,H\rangle : 
\equant{h}{\conj{h\in H}{
\equant{k}{
\conj{
\equant{a}{k=g[\mathbf{i}{:=}a]}
}{
\langle k,h \rangle \in \dplsem{\eta}{P_1}}}
}}\}\\
&\makebox[.92\linewidth][r]{(by Fact \ref{updateenvir})}\\
&\quad=
\{\langle g,H\rangle : 
\equant{a}{
\equant{h}{\conj{h\in H}{
\equant{k}{
\conj{
k=g[\mathbf{i}{:=}a]
}{
\langle k,h \rangle \in \dplsem{\eta}{P_1}}}
}}}\}\\
&\quad=
\{\langle g,H\rangle : 
\equant{a}{\equant{h}{\conj{h\in H}{
\langle g[\mathbf{i}{:=}a],h \rangle \in 
\dplsem{\eta}{P_1}}
}}\}\\
&\quad=
\{\langle g,H\rangle : 
\equant{a}{\equant{h}{\conj{h\in H}{
\langle g[\mathbf{i}{:=}a],h \rangle \in 
\dplsem{\eta[x{:=}a]}{P_1[\mathbf{i}{:=}x]}}
}}\}\\
&\makebox[.92\linewidth][r]{(by Lemma \ref{substlemma2})}\\
&\quad=
\{\langle g,H\rangle : 
\equant{a}{
\langle g[\mathbf{i}{:=}a],H \rangle \in 
\semlift{\dplsem{\eta[x{:=}a]}{P_1[\mathbf{i}{:=}x]}}
}\}\\
&\quad=
\{\langle g,H\rangle : 
\equant{a}{
\langle g[\mathbf{i}{:=}a],H \rangle \in 
\tdlsem{\eta[x{:=}a]}{\synlift{P_1[\mathbf{i}{:=}x]}}
}\}\\
&\makebox[.92\linewidth][r]{(by induction hypothesis)}\\
&\quad=
\tdlsem{\eta}{\dynexists_{\mathbf{i}} x.\,\synlift{P_1[\mathbf{i}{:=}x]}}\\
&\quad=
\tdlsem{\eta}{\synlift{\equant{\mathbf{i}}{P_1}}}\\
\end{align*}
\end{proof}
\end{lemma}

\begin{proposition}
Let $P$ be a DPL formula, and $g$ be an assignment.  
Then, $\dplvalid{g}{P}$
if and only if
$\tdlvalid{g}{\synlift{P}}$.
\begin{proof}
By definition, $\dplvalid{g}{P}$ if and only if 
$\equant{h\in \env}{\langle g, h\rangle \in \dplsem{}{P}}$.
This is equivalent to $\langle g, \env\rangle \in \semlift{\dplsem{}{P}}$.
Then, by Lemma~\ref{commutlemma}, we have
$\langle g, \env\rangle \in \tdlsem{}{\synlift{P}}$.
This is, by definition, $\tdlvalid{g}{\synlift{P}}$.
\end{proof}
\end{proposition}



\section{The higher-order case}


\section{Conclusions} \label{sec:conclusions}


The list below summarizes the advantages of the continuation-based approach:

\begin{itemize}

\item The approach is independent of the intermediate language used to express meanings of the expressions. This allows to use mathematical and logical theories developed outside computational linguistics.\footnote{An extension of first logic language with $\lambda$ is used here because it is convenient and intuitive.} Therefore, natural language phenomena can be explained in terms of well-established and well-understood theories.

\item Variables do not have any special status and are variables in the usual mathematical sense. Therefore, the notions of free and bound variables are standard.  

\item There is no imperative dynamic notions as assignment functions, therefore destructive assignment problem does not hold. Meanings assigned to expressions are closed $\lambda$-terms. 

\item There is no need for rules that artificially extend the scope of quantifiers.

\item Context and content are regarded separately, but they do interact during the computation of the meaning of discourse.  

\item The approach does not depend on any specific structure given to the context. In contrast, context is defined as a term of type parameter $\gamma$ and, therefore, its structure can be altered when necessary.

\item The approach is truly compositional: the meaning of a complex expression is computed by $\beta$-reducing the composition of the meanings of its lexical items.

\end{itemize}


\bibliographystyle{abbrv}
\bibliography{biblio}

\end{document}

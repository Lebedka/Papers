\section{The underlying dynamic logic} \label{sec:dyn_log}

It is possible to construct dynamic meanings as those discussed in Section~\ref{sec:cont_based_dyn} in a concise and systematic way. This is achieved by defining a continuation-based dynamic logic and this section presents this logic.

Terms and types are given by Definitions~\ref{def:terms-FO} and~\ref{def:types-FO}:
\begin{definition}[$\lambda$-terms]\label{def:FO-lambda-terms}
 The set of \textbf{$\lambda$-terms} $\Lambda$ is constructed from an enumerable set of variables $V = \{ v, v_1, v_2, \dots \}$, logical constants $\land, \exists, \neg$, two special constants $::$ and $\selK$, an enumerable set of predicate symbols $R = \{ R_1, R_2, \dots  \}$ and an enumerable set of constants $K = \{ c, c_1, c_2, \dots \}$  using application and (function) abstraction:
\begin{center}
$
\begin{array}{rcl}
x \in V & \Longrightarrow & x \in \Lambda\\
c \in K & \Longrightarrow & c \in \Lambda\\
M,N \in \Lambda &  \Longrightarrow & (MN) \in \Lambda\\ 
 x \in V, M \in \Lambda&  \Longrightarrow  & (\lambda x.M) \in \Lambda \\
  M \in \Lambda&  \Longrightarrow  & (\exists M) \in \Lambda\\ 
M,N \in \Lambda &  \Longrightarrow  & (M \land N) \in \Lambda \\
M \in \Lambda &  \Longrightarrow  & (\neg M)  \in \Lambda \\
M,x \in \Lambda &  \Longrightarrow  & (M::x) \in \Lambda\\
M \in \Lambda &  \Longrightarrow  & (\selK(M))  \in \Lambda
\end{array} 
$
\end{center}
\label{def:terms-FO}
\end{definition}

%M,x \in \Lambda &  \Longrightarrow  & (\updtfo(M,x))  \in \Lambda\\
%M \in \Lambda &  \Longrightarrow  & (\selK(M))  \in \Lambda\\


\begin{definition}[Free variables] The set of \textbf{free variables} of $t$, $FV(t)$, is defined inductively as follows:
\begin{align*}
FV(x) = \ &  \{ x \} \\
FV(c) = \ & \{ \} \\
%FV(\exists M) =  \ & FV(M) \\
%FV(M \land N) = \ & FV(M) \cup FV(N)  \\
%FV(\neg M) =  \ & FV(M) \\
%FV( \updtfo (M,x)) = \ &  FV(M) \cup FV(x)  \\
%FV(\selK(M)) = \ & FV(M) \\
FV(M N) = \ & FV(M) \cup FV(N)  \\
FV( \lambda x. M) = \ & FV(M) - \{ x \}
\end{align*}
\end{definition}

\begin{definition}[Types]
The set $\Types$ of types is defined inductively as follows:
\begin{center} $
\begin{array}{rll}
\text{Atomic types:} & \iota \in \Types & \text{(static individuals)} \\
 & o \in \Types &  \text{(static propositions)} \\
&  \gamma \in \Types &  \text{(context)} \\\\
 \text{Complex types:} & \alpha, \beta \in \Types \Longrightarrow & (\alpha \rightarrow \beta) \in \Types \\
\end{array} 
$ \end{center}
\label{def:types-FO}
\end{definition}

Typing rules define typing relations between terms and types:
\begin{definition}[Typing rules] A statement $t:\delta$, meaning $t$ has type $\delta$, is \textbf{derivable} from the basis $\Delta$, i.e. $\Delta \vdash t: \delta$, if $\Delta \vdash t: \delta$ can be produced using the following rules:

\begin{prooftree}
\AXC{} \RightLabel{axiom}
\UIC{$\Gamma, x: \alpha \vdash x: \alpha $}
\end{prooftree}

\begin{prooftree}
\AXC{} \RightLabel{axiom}
\UIC{$\Gamma \vdash \top: o $}
\end{prooftree}

\begin{prooftree}
\AXC{} %\RightLabel{axiom}
\UIC{$\Gamma, M: o, N:o \vdash M \land N: o $}
\end{prooftree}

\begin{prooftree}
\AXC{} %\RightLabel{axiom}
\UIC{$\Gamma, M: \iota \rightarrow o \vdash \exists M: o $}
\end{prooftree}

\begin{prooftree}
\AXC{} %\RightLabel{axiom}
\UIC{$\Gamma, M: o \vdash \neg M: o $}
\end{prooftree}

\begin{prooftree}
\AXC{} %\RightLabel{axiom}
\UIC{$\Gamma, c: \gamma \vdash \selK \ c:  \iota $}
\end{prooftree}

\begin{prooftree}
\AXC{} %\RightLabel{axiom}
\UIC{$\Gamma, i: \iota, c: \gamma \vdash (i::c) :  \gamma$}
\end{prooftree}


\begin{prooftree}
\AXC{} \RightLabel{axiom}
\UIC{$\Gamma \vdash c_{iv}:   \iota \rightarrow o $}
\end{prooftree}


\begin{prooftree}
\AXC{} \RightLabel{axiom}
\UIC{$\Gamma \vdash c_{tv}:   \iota \rightarrow \iota \rightarrow o $}
\end{prooftree}

\begin{prooftree}
\AXC{} \RightLabel{axiom}
\UIC{$\Gamma \vdash c_{n}:   \iota \rightarrow o $}
\end{prooftree}

\begin{prooftree}
\AXC{} \RightLabel{axiom}
\UIC{$\Gamma \vdash c_{np}:   (\iota \rightarrow o) \rightarrow o $}
\end{prooftree}

%\begin{prooftree}
%\AXC{} \RightLabel{axiom}
%\UIC{$\Gamma \vdash c_{det}:   (\iota \rightarrow o) \rightarrow ( ( \iota \rightarrow o ) \rightarrow o  )$}
%\end{prooftree}

\begin{prooftree}
\AXC{} \RightLabel{axiom}
\UIC{$\Gamma \vdash c_{rp}:  (  ((\iota \rightarrow o) \rightarrow o)  \rightarrow o)  \rightarrow   ( \iota \rightarrow o ) \rightarrow ( \iota \rightarrow o )$}
\end{prooftree}


\begin{prooftree}
\AXC{$\Gamma \vdash v: \alpha \rightarrow \beta$}
\AXC{$\Gamma \vdash u: \alpha$} \RightLabel{app}
\BIC{$\Gamma \vdash vu: \beta$}
\end{prooftree}

\begin{prooftree}
\AXC{$\Gamma, x: \alpha \vdash v:\beta$} \RightLabel{abs}
\UIC{$\Gamma \vdash \lambda x.v: \alpha \rightarrow \beta$}
\end{prooftree}
where $c_{tv}$,  $c_{iv}$, $c_{n}$, $c_{np}$ and $c_{rp}$ are constants standing for transitive and intransitive  verbs, nouns, noun phrases and relative pronouns respectively. Typing rules for other syntactic categories can be added analogously.

\end{definition}

The first axiom and the two rules (application and abstraction) are standard typing relations in simply-typed lambda calculus. The second axiom determines the type of the logical $\top$ symbol. The constant $\selK$ has type $(\gamma \rightarrow \iota)$ and the constant $::$ has type $(\iota \rightarrow \gamma \rightarrow \gamma)$. 

For each logical constant ($\neg$, $\land$ and $\exists$), its dynamic counterpart is specified by the following definition: 
\begin{definition}[Dynamic logical constants] \label{def:DLCnst-FO} Let $\A2$ and $\B2$ be terms of type $(\gamma \rightarrow (\gamma \rightarrow o) \rightarrow o)$, $\P2$ be the term of type $(\iota \rightarrow \gamma \rightarrow (\gamma \rightarrow o) \rightarrow o)$, $e$ and  $e'$ be terms of type $\gamma$, $\phi$ be a term of type $(\gamma \rightarrow o)$, $\x1$ be the term of type $\iota$. \textbf{Dynamic negation, conjunction and existential quantification} are defined respectively as follows:
\begin{subequations}
\begin{align}
\n2 \A2 \defeq \ & \lambda e \phi. \neg (\A2 e (\lambda e'. \top) ) \land \phi e  \label{def:DLCnst-FO-neg} \\
\A2 \cnj2 \B2 \defeq \ & \lambda e\phi. \A2 e (\lambda e'. \B2 e' \phi) \label{def:DLCnst-FO-cnj} \\
\ex2 ( \lambda \x1. \P2[\x1]) \defeq \ & \lambda e \phi. \exists ( \lambda \x1. \P2 [\x1] \ (x::e) \ \phi )\label{def:DLCnst-FO-ex} 
\end{align}
\end{subequations}
\end{definition}
Dynamic negation of a dynamic proposition $\A2$, $\n2 \A2$, is an abbreviation used for the term shown in~\eqref{def:DLCnst-FO-neg}. Within the body of this term, the continuation and context of $\A2$ are ``erased'' (by giving the term $(\lambda e'. \top)$ as the second argument of $\x2$)\footnote{This can be more clearly seen from Corollary~\ref{cor:DL-FO-handtop}.}, the resulting static proposition is negated, and the conjunct $\phi e$ is added. Note that both $\phi$ and $e$ are variables bound by $\lambda$. Therefore, the context of $\A2$ is not available to the continuation of the resulting term $\n2 \A2$. 

Dynamic existentially quantified term $\ex2 ( \lambda \x1. \P2[\x1]) $ is an abbreviation for the term shown in~\eqref{def:DLCnst-FO-ex}. It has a $\lambda$-abstraction over variables $e$ and $\phi$ and an existentially quantified variable $\x1$. Its body contains $\P2[\x1]$, which is given $e$ updated with $\x1$, $(x::e)$, and $\phi$ as arguments.  Note that $\n2$ and $\ex2$ are defined respectively via $\neg$ and $\exists$.
Dynamic conjunction is defined as a composition of two dynamic terms. The logical conjunction of static propositions is provided by the fact that each dynamic proposition has a conjunct $\phi e $ in its body.


 Static negation, conjunction and existential quantification can now be translated into dynamic ones: 
\begin{definition}[Embedding of first order logic into dynamic logic]\label{def:DTerms-FO} Let $\P1$ be a term of type $(\iota_1 \rightarrow \dots \rightarrow \iota_n \rightarrow o)$, $\A1$ and $\B1$ be terms of type $o$, $\t1_1, \dots, \t1_n$ and $\x1$ be terms of type $\iota$. Then propositions, negated propositions, conjunctions of two propositions and existentially quantified propositions are dynamized in the following way:
\begin{subequations}
\begin{align}
\tr{ \P1 \t1_1 \dots \t1_n } \defeq \ & \lambda e \phi . \P1 \t1_1 \dots \t1_n \land \phi e  \label{def:DTerms-FO-P}\\
\tr{ \neg \A1  } \defeq \ & \n2 \tr{\A1}  \label{def:DTerms-FO-neg}\\
\tr{ \A1 \land \B1} \defeq \ & \tr{\A1} \cnj2 \tr{\B1}  \label{def:DTerms-FO-cnj}\\
\tr{\exists( \lambda \x1. \P1 [\x1])} \defeq \ & \ex2( \lambda  \x1. \tr{\P1 [\x1]})  \label{def:DTerms-FO-ex}
\end{align}
\end{subequations}
\end{definition}
Equation~\eqref{def:DTerms-FO-P} defines the dynamization of a proposition $\P1 \t1_1 \dots \t1_n $ of type $o$ by adding a $\lambda$-abstraction with two arguments  $e$ and $\phi$  (of types $\gamma$ and $(\gamma \rightarrow o)$ respectively) and a conjunct $\phi e$. Therefore, the resulting dynamic term is of type $(\gamma \rightarrow (\gamma \rightarrow o) \rightarrow o )$, the type of a dynamic proposition. Equations~\eqref{def:DTerms-FO-neg},~\eqref{def:DTerms-FO-cnj} and~\eqref{def:DTerms-FO-ex} extend dynamization to non-atomic formulas.



%Note that Definitions~\ref{def:DLCnst-FO} and~\ref{def:DTerms-FO} allow representing de~Groote's~\cite{deGroote:2006:Towards-a-Montagovian-Account-of-Dynamics} dynamic terms in a compact way.  While interpretations in~\cite{deGroote:2006:Towards-a-Montagovian-Account-of-Dynamics} explicitly show extra parameters, i.e. contexts and continuations, these new definitions make it possible to hide these parameters. Moreover, the resulting compact dynamic terms are structurally analogous to their original static counterparts and, hence, are more intuitive.


\begin{remark} In equations below, terms on the left side of $\defeq$ abbreviate respective terms on the right side:
\begin{subequations}
\begin{align*} 
\A2 \impK2 \B2 \defeq \ & \n2 (\A2 \cnj2 \n2 \B2) \\
\all2( \lambda \x1. \P2[\x1]) \defeq \ & \n2 \ex2 ( \lambda \x1. \n2 \P2 [\x1] )
\end{align*} 
\end{subequations}
\label{remark:impforall}
\end{remark}



Proposition~\ref{thrm:scopeextension-G}  proves a useful $\beta$-equivalence that can be useful when computing interpretations of certain phrases containing an existentially quantified variable.
%%%%%%%%%%%%%%%%%%%%%%%%%%%%%%%%%%%%%%%%%%%%%%%%%%%%%%%%%%%%%%%%%%%%%%%%%%%%%%%%%%%%%
%%%%%%%%%%%%%%%%%%%%%%%%%%%%%%%%%%%%%%%%%%%%%%%%%%%%%%%%%%%%%%%%%%%%%%%%%%%%%%%%%%%%%
\begin{proposition} For all terms $\A1$ and $\B1$ of type $o$ such that  $\x1 \in FV(\A1)$ and $\x1 \notin FV(\B1)$ for an $\x1$ of type~$\iota$, the following equivalence holds: \label{thrm:scopeextension-G}
\begin{align*}
  {\ex2} (\lambda \x1.  \tr{\A1[\x1]}) \cnj2 \tr{\B1}  =_{\beta}  \ex2 (\lambda \x1.  \tr{\A1[\x1]}  \cnj2  \tr{\B1} ) 
\end{align*} 
\end{proposition}
%%%%%%%%%%%%%%%%%%%%%%%%%%%%%%%%%%%%%%%%%%%%%%%%%%%%%%%%%%%%%%%%%%%%%%%%%%%%%%%%%%%%%
%
\begin{proof} Since $\A1[\x1]$ and $\B1$ are terms of type $o$, their dynamic equivalents are terms of type $(\gamma \rightarrow (\gamma \rightarrow o) \rightarrow o)$ and have the following forms:
\begin{subequations}
\begin{align}
 \tr{\A1[\x1]}  & = \lambda e \phi. \A1'[x] \land \phi e \label{eq:Adynamic} \\
 \tr{\B1} & = \lambda e \phi. \B1' \land  \phi e \label{eq:Bdynamic} 
\end{align}
\end{subequations}

%
\begin{align}
 &  {\ex2} (\lambda \x1.  \tr{\A1[\x1]}) \cnj2 \tr{\B1}  \notag \\ 
 \bred \  &  ( \lambda e \phi.  \exists ( \lambda \x1. \A1'[x] \land \phi (x::e) ) )  \cnj2  \tr{\B1}  \tag{by~\eqref{def:DLCnst-FO-ex} and~\eqref{eq:Adynamic}}\\
= \ &  \lambda e \phi. ( \lambda e \phi.  \exists ( \lambda \x1. \A1'[x] \land \phi (x::e) ) ) e ( \lambda e. \tr{\B1} e \phi  ) \tag{by~\eqref{def:DLCnst-FO-cnj}} \\
\bred \ &   \lambda e \phi. ( \lambda e \phi.  \exists ( \lambda \x1. \A1'[x] \land \phi (x::e) ) ) e ( \lambda e. \B1' \land  \phi e  )  \tag{by~\eqref{eq:Bdynamic}}\\
\bred \ &   \lambda e \phi.  \exists ( \lambda \x1. \A1'[x] \land  (  \B1' \land  \phi  (x::e)  ) ) \notag %\label{eq:DLGexscope1}
\end{align}
%
\begin{align}
 &  \ex2 (\lambda \x1.  \tr{\A1[\x1]}  \cnj2  \tr{\B1} ) \notag \\
  = \ &    \ex2 (\lambda \x1.  \lambda e \phi. \tr{\A1[\x1]}  e (\lambda e. \tr{\B1}  e \phi)  ) \tag{by~\eqref{def:DLCnst-FO-cnj}} \\
  \bred \ &  \ex2 (\lambda \x1.  \lambda e \phi. (\lambda e \phi. \A1'[x] \land \phi e ) e (\lambda e. \B1' \land  \phi e)  ) \tag{by~\eqref{eq:Adynamic} and~\eqref{eq:Bdynamic}}\\
    \bred \ &   \ex2 (\lambda \x1.  \lambda e \phi.\A1'[x] \land  (\B1' \land  \phi e )   ) \notag \\
\bred \ &   \lambda e \phi.  \exists ( \lambda \x1.   \A1'[x] \land  (  \B1' \land  \phi ( \x1:: e ) ) ) \tag{by~\eqref{def:DLCnst-FO-ex}}
\end{align}
%
\end{proof}

%%%%%%%%%%%%%%%%%%%%%%%%%%%%%%%%%%%%%%%%%%%%%%%%%%%%%%%%%%%%%%%%%%%%%%%%%%%%%%%%%%%%%
%%%%%%%%%%%%%%%%%%%%%%%%%%%%%%%%%%%%%%%%%%%%%%%%%%%%%%%%%%%%%%%%%%%%%%%%%%%%%%%%%%%%%
\begin{proposition} For all $\h1$ and $\v1$ of type $o$, and for all $\u1$ of type $\gamma$, the following holds: $$\tr{\h1} \u1 (\lambda e.\v1) =  \h1 \land \v1$$ where $e$ is a variable of type $\gamma$. \label{DL-FO-handv}
\end{proposition}
%%%%%%%%%%%%%%%%%%%%%%%%%%%%%%%%%%%%%%%%%%%%%%%%%%%%%%%%%%%%%%%%%%%%%%%%%%%%%%%%%%%%%
\begin{proof}
The proof is by induction on the structure of the term $\h1$.
\begin{itemize}
\item $\h1$ is a proposition of the form $\P1 t_1 \dots t_n$.
\begin{align}
\tr{\P1 \t1_1 \dots \t1_n} \u1  (\lambda e.\v1)  = \ & (\lambda e \phi. \P1 \t1_1 \dots \t1_n \land \phi e) \u1  (\lambda e.\v1) \tag{by~\eqref{def:DTerms-FO-P}} \\
\bconv \ & (\lambda  \phi. \P1 \t1_1 \dots \t1_n \land \phi \u1)   (\lambda e.\v1) \notag \\
\bconv \ & \lambda  \phi. \P1 \t1_1 \dots \t1_n \land (\lambda e.\v1) \u1    \notag \\
\bconv \ & \lambda  \phi. \P1 \t1_1 \dots \t1_n \land \v1    \notag 
\end{align}

\item  $\h1$ is a negated proposition $ \neg \w1$. 
\begin{align}
\tr{\neg \w1} \u1 (\lambda e. \v1) = \ &  \n2 \tr{\w1}  \u1 (\lambda e. \v1)  \tag{by~\eqref{def:DTerms-FO-neg}} \\
= \ & ( \lambda e \phi. \neg (\tr{\w1} e (\lambda e'. \top) ) \land \phi e)  \u1 (\lambda e. \v1)  \tag{by~\eqref{def:DLCnst-FO-neg}} \\
\bconv \ & ( \lambda \phi.\neg (\tr{\w1} \u1 (\lambda e'. \top) ) \land \phi \u1 ) (\lambda e. \v1)  \notag \\
\bconv \ & \neg (\tr{\w1} \u1 (\lambda e'. \top) ) \land  (\lambda e. \v1) \u1   \notag \\
\bconv \ & \neg (\tr{\w1} \u1 (\lambda e'. \top) ) \land   \v1   \notag \\
= \ & \neg(\w1 \land \top) \land \v1 \tag{by I.H.} \\
= \ & \neg \w1 \land \v1
\end{align}

\item $\h1$ is a conjunction of two propositions $\w1 \land \z1$.
\begin{align}
(\tr{\w1 \land \z1}) \u1 (\lambda e. \v1)  = \ & (\tr{\w1} \cnj2 \tr{\z1}) \u1 (\lambda e. \v1)   \tag{by~\eqref{def:DTerms-FO-cnj}} \\
 = \ & (\lambda e\phi. \tr{\w1} e (\lambda e'. \tr{\z1} e' \phi) ) \u1 (\lambda e. \v1)   \tag{by~\eqref{def:DLCnst-FO-cnj}} \\
\bconv \ & (\lambda \phi. \tr{\w1} \u1 (\lambda e'. \tr{\z1} e' \phi) ) (\lambda e. \v1)   \notag \\
\bconv \ &  \tr{\w1} \u1 (\lambda e'. \tr{\z1} e'  (\lambda e. \v1) )  \notag \\
= \ &  \tr{\w1} \u1 (\lambda e'. \z1 \land \v1)  \tag{by I.H., $e \notin FV(\v1)$} \\
= \ &  \w1 \land (\z1 \land \v1)  \tag{by I.H., $e' \notin FV(\z1 \land \v1)$} \\
\logeq \ & (\w1 \land \z1) \land \v1 \notag
\end{align}

\item $\h1$ is an existentially quantified formula of the form $\exists( \lambda \x1. \P1 [\x1])$.
\begin{align}
\tr{\exists ( \lambda \x1. \P1 [\x1])} \u1 (\lambda e. \v1)  = \ & ( \ex2 ( \lambda \x1. \tr{\P1 [\x1]})  ) \u1 (\lambda e. \v1)   \tag{by~\eqref{def:DTerms-FO-ex}} \\
 = \ & ( \lambda e \phi. \exists ( \lambda \x1. \P2 [\x1] (x::e) \phi  )) \u1 (\lambda e. \v1)   \tag{by~\eqref{def:DLCnst-FO-ex}} \\
\bconv \ & ( \lambda  \phi. \exists ( \lambda \x1. \P2 [\x1] (x::\u1) \phi  ))  (\lambda e. \v1)  \notag \\
\bconv \ & \exists ( \lambda \x1. \P2 [\x1] (x::\u1) (\lambda e. \v1)  )  \notag \\
= \ & \exists ( \lambda \x1. \P1 [\x1] \land \v1  )  \tag{by I.H.} \\
= \ & \exists ( \lambda \x1. \P1 [\x1]) \land \v1    \tag{$\x1 \notin FV(\v1)$} 
\end{align}

\end{itemize}
\end{proof}


%%%%%%%%%%%%%%%%%%%%%%%%%%%%%%%%%%%%%%%%%%%%%%%%%%%%%%%%%%%%%%%%%%%%%%%%%%%%%%%%%%%%%
%%%%%%%%%%%%%%%%%%%%%%%%%%%%%%%%%%%%%%%%%%%%%%%%%%%%%%%%%%%%%%%%%%%%%%%%%%%%%%%%%%%%%
\begin{corollary} \label{cor:DL-FO-handtop} For all propositions $\t1$ of type $o$, and for all terms $\u1$ of type $\gamma$, the following folds: 
\begin{align}\tr{\t1} \u1 (\lambda e. \top) \logeq  \t1 \notag  %\label{eq:DL-FO-handtop}
\end{align}
where $e$ is a variable of type $\gamma$.
 \label{DL-FO-handtop}
\end{corollary}
%%%%%%%%%%%%%%%%%%%%%%%%%%%%%%%%%%%%%%%%%%%%%%%%%%%%%%%%%%%%%%%%%%%%%%%%%%%%%%%%%%%%%
\begin{proof} Take $\v1$ equal to $\top$ in Proposition~\ref{DL-FO-handv}. Then
$$\tr{\t1} \u1 (\lambda e. \top) = \t1 \land \top \logeq \t1$$
\end{proof}


\begin{definition} A dynamic proposition $\t2$ is true in a model $\mathcal{M}$, denoted
$\mathcal{M}~\models_{dyn}~\t2$, if and only if $\mathcal{M}$ $\models$ $\t2 u (\lambda e. \top)$ for every $u$ of type $\gamma$. \label{def:dyntrueinmodel}
\end{definition}

\begin{theorem}[Conservation] A proposition $\t1$ is true in a model $\mathcal{M}$  if and only if its dynamic variant $\tr{\t1}$ is true in the same model: 
 $$\mathcal{M} \models \t1 \ \text{iff} \ \mathcal{M} \models_{dyn} \tr{\t1}$$
 \label{th:conservation:FO}
\end{theorem}
\begin{proof}

If $\mathcal{M} \models \t1$, then, by Corollary~\ref{cor:DL-FO-handtop}, $\mathcal{M} \models \tr{\t1} \u1 (\lambda e.\top)$. Therefore, by Definition~\ref{def:dyntrueinmodel}, $\mathcal{M} \models_{dyn} \tr{\t1}$.

If $\mathcal{M} \models_{dyn} \tr{\t1}$, then, by Definition~\ref{def:dyntrueinmodel}, $\mathcal{M} \models \tr{\t1} \u1 (\lambda e.\top)$. Therefore, by Corollary~\ref{cor:DL-FO-handtop}, $\mathcal{M} \models \t1$.
\end{proof}

The conservation theorem proves that a static proposition and its dynamic version defined in this section hold in the same models.



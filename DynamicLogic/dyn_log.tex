\section{The underlying dynamic logic} \label{sec:dyn_log}

Although compositional dynamic framework {\G}, introduced in~\cite{deGroote:2006:Towards-a-Montagovian-Account-of-Dynamics} and reviewed in the previous section, have shown itself to be promising by successfully handling donkey anaphora, its interpretations look complex and the computation of the meaning can be difficult to understand. In his later talks, de~Groote proposed an improvement of framework {\G}, called here framework {\GN}, that represents his semantics in a more concise and systematic way. To do so, de~Groote defined a continuation-based dynamic logic and this section presents this logic.

\subsection{Formal Details}

Terms and types are given by Definitions~\ref{def:terms-FO} and~\ref{def:types-FO}:
\begin{definition}[$\lambda$-terms]\label{def:FO-lambda-terms}
 The set of \textbf{$\lambda$-terms} $\Lambda$ is constructed from an enumerable set of variables $V = \{ v, v_1, v_2, \dots \}$, logical constants $\land, \exists, \neg$, two special constants $::$ and $\sel$, an enumerable set of predicate symbols $R = \{ R_1, R_2, \dots  \}$ and an enumerable set of constants $K = \{ c, c_1, c_2, \dots \}$  using application and (function) abstraction:
\begin{center}
$
\begin{array}{rcl}
x \in V & \Longrightarrow & x \in \Lambda\\
c \in K & \Longrightarrow & c \in \Lambda\\
M,N \in \Lambda &  \Longrightarrow & (MN) \in \Lambda\\ 
 x \in V, M \in \Lambda&  \Longrightarrow  & (\lambda x.M) \in \Lambda \\
  M \in \Lambda&  \Longrightarrow  & (\exists M) \in \Lambda\\ 
M,N \in \Lambda &  \Longrightarrow  & (M \land N) \in \Lambda \\
M \in \Lambda &  \Longrightarrow  & (\neg M)  \in \Lambda \\
M,x \in \Lambda &  \Longrightarrow  & (M::x) \in \Lambda\\
M \in \Lambda &  \Longrightarrow  & (\sel(M))  \in \Lambda
\end{array} 
$
\end{center}
\label{def:terms-FO}
\end{definition}

%M,x \in \Lambda &  \Longrightarrow  & (\updtfo(M,x))  \in \Lambda\\
%M \in \Lambda &  \Longrightarrow  & (\sel(M))  \in \Lambda\\


\begin{definition}[Free variables] The set of \textbf{free variables} of $t$, $FV(t)$, is defined inductively as follows:
\begin{align*}
FV(x) = \ &  \{ x \} \\
FV(c) = \ & \{ \} \\
%FV(\exists M) =  \ & FV(M) \\
%FV(M \land N) = \ & FV(M) \cup FV(N)  \\
%FV(\neg M) =  \ & FV(M) \\
%FV( \updtfo (M,x)) = \ &  FV(M) \cup FV(x)  \\
%FV(\sel(M)) = \ & FV(M) \\
FV(M N) = \ & FV(M) \cup FV(N)  \\
FV( \lambda x. M) = \ & FV(M) - \{ x \}
\end{align*}
\end{definition}

\begin{definition}[Types]
The set $\Types$ of types is defined inductively as follows:
\begin{center} $
\begin{array}{rll}
\text{Atomic types:} & \iota \in \Types & \text{(static individuals)} \\
 & o \in \Types &  \text{(static propositions)} \\
&  \gamma \in \Types &  \text{(context)} \\\\
 \text{Complex types:} & \alpha, \beta \in \Types \Longrightarrow & (\alpha \rightarrow \beta) \in \Types \\
\end{array} 
$ \end{center}
\label{def:types-FO}
\end{definition}

Typing rules define typing relations between terms and types:
\begin{definition}[Typing rules] A statement $t:\delta$, meaning $t$ has type $\delta$, is \textbf{derivable} from the basis $\Delta$, i.e. $\Delta \vdash t: \delta$, if $\Delta \vdash t: \delta$ can be produced using the following rules:

\begin{prooftree}
\AXC{} \RightLabel{axiom}
\UIC{$\Gamma, x: \alpha \vdash x: \alpha $}
\end{prooftree}

\begin{prooftree}
\AXC{} \RightLabel{axiom}
\UIC{$\Gamma \vdash \top: o $}
\end{prooftree}

\begin{prooftree}
\AXC{} %\RightLabel{axiom}
\UIC{$\Gamma, M: o, N:o \vdash M \land N: o $}
\end{prooftree}

\begin{prooftree}
\AXC{} %\RightLabel{axiom}
\UIC{$\Gamma, M: \iota \rightarrow o \vdash \exists M: o $}
\end{prooftree}

\begin{prooftree}
\AXC{} %\RightLabel{axiom}
\UIC{$\Gamma, M: o \vdash \neg M: o $}
\end{prooftree}

\begin{prooftree}
\AXC{} %\RightLabel{axiom}
\UIC{$\Gamma, c: \gamma \vdash \sel \ c:  \iota $}
\end{prooftree}

\begin{prooftree}
\AXC{} %\RightLabel{axiom}
\UIC{$\Gamma, i: \iota, c: \gamma \vdash (i::c) :  \gamma$}
\end{prooftree}


\begin{prooftree}
\AXC{} \RightLabel{axiom}
\UIC{$\Gamma \vdash c_{iv}:   \iota \rightarrow o $}
\end{prooftree}


\begin{prooftree}
\AXC{} \RightLabel{axiom}
\UIC{$\Gamma \vdash c_{tv}:   \iota \rightarrow \iota \rightarrow o $}
\end{prooftree}

\begin{prooftree}
\AXC{} \RightLabel{axiom}
\UIC{$\Gamma \vdash c_{n}:   \iota \rightarrow o $}
\end{prooftree}

\begin{prooftree}
\AXC{} \RightLabel{axiom}
\UIC{$\Gamma \vdash c_{np}:   (\iota \rightarrow o) \rightarrow o $}
\end{prooftree}

%\begin{prooftree}
%\AXC{} \RightLabel{axiom}
%\UIC{$\Gamma \vdash c_{det}:   (\iota \rightarrow o) \rightarrow ( ( \iota \rightarrow o ) \rightarrow o  )$}
%\end{prooftree}

\begin{prooftree}
\AXC{} \RightLabel{axiom}
\UIC{$\Gamma \vdash c_{rp}:  (  ((\iota \rightarrow o) \rightarrow o)  \rightarrow o)  \rightarrow   ( \iota \rightarrow o ) \rightarrow ( \iota \rightarrow o )$}
\end{prooftree}


\begin{prooftree}
\AXC{$\Gamma \vdash v: \alpha \rightarrow \beta$}
\AXC{$\Gamma \vdash u: \alpha$} \RightLabel{app}
\BIC{$\Gamma \vdash vu: \beta$}
\end{prooftree}

\begin{prooftree}
\AXC{$\Gamma, x: \alpha \vdash v:\beta$} \RightLabel{abs}
\UIC{$\Gamma \vdash \lambda x.v: \alpha \rightarrow \beta$}
\end{prooftree}
where $c_{tv}$,  $c_{iv}$, $c_{n}$, $c_{np}$ and $c_{rp}$ are constants standing for transitive and intransitive  verbs, nouns, noun phrases and relative pronouns respectively. Typing rules for other syntactic categories can be added analogously.

\end{definition}

The first axiom and the two rules (application and abstraction) are standard typing relations in simply-typed lambda calculus. The second axiom determines the type of the logical $\top$ symbol. The constant $\sel$ has type $(\gamma \rightarrow \iota)$ and the constant $::$ has type $(\iota \rightarrow \gamma \rightarrow \gamma)$. 

Each static type can be dynamized in the following way:
\begin{definition}[Dynamization of types]\label{def:DTypes-FO} Let $\iota$ and $o$ be atomic types, $\gamma$ be a type parameter, $\alpha$ and $\beta$ be arbitrary types. Then the types are \textbf{dynamized} in the following way:
\begin{subequations}
\begin{align}
\tr{\iota} \defeq \ &  \iota  \label{def:DTypes-FO:iota}\\
 \tr{o} \defeq \ & \gamma \rightarrow (\gamma \rightarrow o) \rightarrow o \label{def:DTypes-FO:o}\\
  \tr{\alpha \rightarrow \beta} \defeq \ & \tr{\alpha}  \rightarrow \tr{\beta} \label{def:DTypes-FO:atob}
\end{align}
\end{subequations}
\end{definition}
Note that the type $o$ of a static proposition is transformed to type $(\gamma \rightarrow (\gamma \rightarrow o) \rightarrow o)$. Type $(\gamma \rightarrow (\gamma \rightarrow o) \rightarrow o)$ is thus the type of a dynamic proposition. Therefore, dynamic propositions are functions from $\gamma$ (type of context) and $(\gamma \rightarrow o)$ (type of continuation) to $o$ (type of static proposition).
Static and dynamic individuals are defined to have the same type.

For each logical constant ($\neg$, $\land$ and $\exists$), its dynamic counterpart is specified by the following definition: 
\begin{definition}[Dynamic logical constants] \label{def:DLCnst-FO} Let $\A2$ and $\B2$ be terms of type $(\gamma \rightarrow (\gamma \rightarrow o) \rightarrow o)$, $\P2$ be the term of type $(\iota \rightarrow \gamma \rightarrow (\gamma \rightarrow o) \rightarrow o)$, $e$ and  $e'$ be terms of type $\gamma$, $\phi$ be a term of type $(\gamma \rightarrow o)$, $\x1$ be the term of type $\iota$. \textbf{Dynamic negation, conjunction and existential quantification} are defined respectively as follows:
\begin{subequations}
\begin{align}
\n2 \A2 \defeq \ & \lambda e \phi. \neg (\A2 e (\lambda e'. \top) ) \land \phi e  \label{def:DLCnst-FO-neg} \\
\A2 \cnj2 \B2 \defeq \ & \lambda e\phi. \A2 e (\lambda e'. \B2 e' \phi) \label{def:DLCnst-FO-cnj} \\
\ex2 ( \lambda \x1. \P2[\x1]) \defeq \ & \lambda e \phi. \exists ( \lambda \x1. \P2 [\x1] \ (x::e) \ \phi )\label{def:DLCnst-FO-ex} 
\end{align}
\end{subequations}
\end{definition}
Dynamic negation of a dynamic proposition $\A2$, $\n2 \A2$, is an abbreviation used for the term shown in~\eqref{def:DLCnst-FO-neg}. Within the body of this term, the continuation and context of $\A2$ are ``erased'' (by giving the term $(\lambda e'. \top)$ as the second argument of $\A2$)\footnote{This can be more clearly seen from Corollary~\ref{cor:DL-FO-handtop}.}, the resulting static proposition is negated, and the conjunct $\phi e$ is added. Note that both $\phi$ and $e$ are variables bound by $\lambda$. Therefore, the context of $\A2$ is not available to the continuation of the resulting term $\n2 \A2$. % and thus sentences as~\eqref{sen:itisred} are considered infelicitous:\footnote{It is a common practice to ``reject'' sentences as~\eqref{sen:itisred} by simply not providing interpretations for them. For example, according to DRT, there is no suitable marker available as an antecedent for \txt{it} in the context (remember that the content and context are represented by the same structure) and DRT does not explain how to proceed in this case. DPL and DMG require prior indexing and it is not clear which index should be assigned to \txt{it}. However, when a human being tries to comprehend sentences like~\eqref{sen:itisred}, it is resonable to assume that he/she must first interpret them (into some internal mental representation) before realizing that there is something wrong (i.e. that he/she cannot resolve the anaphora for \txt{it}). Therefore, under this assumption, frameworks that fail to provide interpretations for sentences like~\eqref{sen:itisred} can be considered flawed.}
%\enumsentence{?\txt{John does not have a car and it is red.} \label{sen:itisred}}
Dynamic existentially quantified term $\ex2 ( \lambda \x1. \P2[\x1]) $ is an abbreviation for the term shown in~\eqref{def:DLCnst-FO-ex}. It has a $\lambda$-abstraction over variables $e$ and $\phi$ and an existentially quantified variable $\x1$. Its body contains $\P2[\x1]$, which is given $e$ updated with $\x1$, $(x::e)$, and $\phi$ as arguments.  Note that $\n2$ and $\ex2$ are defined respectively via $\neg$ and $\exists$.
Dynamic conjunction is defined as a composition of two dynamic terms. The logical conjunction of static propositions is provided by the fact that each dynamic proposition has a conjunct $\phi e $ in its body, as defined in~\eqref{def:DTerms-FO-P} of the next definition.

 Given dynamic logical connectives, all static terms can be dynamized:
\begin{definition}[Dynamization of terms]\label{def:DTerms-FO} Let $\P1$ be a term of type $(\iota_1 \rightarrow \dots \rightarrow \iota_n \rightarrow o)$, $\A1$ and $\B1$ be terms of type $o$, $\t1_1, \dots, \t1_n$ and $\x1$ be terms of type $\iota$. Then propositions, negated propositions, conjunctions of two propositions and existentially quantified propositions are dynamized in the following way:
\begin{subequations}
\begin{align}
\tr{ \P1 \t1_1 \dots \t1_n } \defeq \ & \lambda e \phi . \P1 \t1_1 \dots \t1_n \land \phi e  \label{def:DTerms-FO-P}\\
\tr{ \neg \A1  } \defeq \ & \n2 \tr{\A1}  \label{def:DTerms-FO-neg}\\
\tr{ \A1 \land \B1} \defeq \ & \tr{\A1} \cnj2 \tr{\B1}  \label{def:DTerms-FO-cnj}\\
\tr{\exists( \lambda \x1. \P1 [\x1])} \defeq \ & \ex2( \lambda  \x1. \tr{\P1 [\x1]})  \label{def:DTerms-FO-ex}
\end{align}
\end{subequations}
\end{definition}
Equation~\eqref{def:DTerms-FO-P} defines the dynamization of a proposition $\P1 \t1_1 \dots \t1_n $ of type $o$ by adding a $\lambda$-abstraction with two arguments  $e$ and $\phi$  (of types $\gamma$ and $(\gamma \rightarrow o)$ respectively) and a conjunct $\phi e$. Therefore, the resulting dynamic term is of type $(\gamma \rightarrow (\gamma \rightarrow o) \rightarrow o )$, the type of a dynamic proposition. Equations~\ref{def:DTerms-FO-neg},~\ref{def:DTerms-FO-cnj} and~\ref{def:DTerms-FO-ex} extend dynamization to non-atomic formulas.

Note that Definitions~\ref{def:DLCnst-FO} and~\ref{def:DTerms-FO} allow representing de~Groote's~\cite{deGroote:2006:Towards-a-Montagovian-Account-of-Dynamics} (Section~\ref{sec:cont_based_dyn}) dynamic terms in a compact way.  While interpretations in~\cite{deGroote:2006:Towards-a-Montagovian-Account-of-Dynamics} explicitly show extra parameters, i.e. contexts and continuations, these new definitions make it possible to hide these parameters. Moreover, the resulting compact dynamic terms are structurally analogous to their original static counterparts and, hence, are more intuitive.

\begin{remark} In equations below, terms on the left side of $\defeq$ abbreviate respective terms on the right side:
\begin{align} 
\A2 \imp2 \B2 \defeq \ & \n2 (\A2 \cnj2 \n2 \B2) \\
\all2( \lambda \x1. \P2[\x1]) \defeq \ & \n2 \ex2 ( \lambda \x1. \n2 \P2 [\x1] )
\end{align}
\end{remark}


Proposition~\ref{thrm:scopeextension-G}  proves an important $\beta$-equivalence that can be useful when computing interpretations of certain phrases containing an existentially quantified variable.
%%%%%%%%%%%%%%%%%%%%%%%%%%%%%%%%%%%%%%%%%%%%%%%%%%%%%%%%%%%%%%%%%%%%%%%%%%%%%%%%%%%%%
%%%%%%%%%%%%%%%%%%%%%%%%%%%%%%%%%%%%%%%%%%%%%%%%%%%%%%%%%%%%%%%%%%%%%%%%%%%%%%%%%%%%%
\begin{proposition} For all terms $\A1$ and $\B1$ of type $o$ such that  $\x1 \in FV(\A1)$ and $\x1 \notin FV(\B1)$ for an $\x1$ of type~$\iota$, the following equivalence holds: \label{thrm:scopeextension-G}
\begin{align*}
  {\ex2} (\lambda \x1.  \tr{\A1[\x1]}) \cnj2 \tr{\B1}  =_{\beta}  \ex2 (\lambda \x1.  \tr{\A1[\x1]}  \cnj2  \tr{\B1} ) 
\end{align*} 
\end{proposition}
%%%%%%%%%%%%%%%%%%%%%%%%%%%%%%%%%%%%%%%%%%%%%%%%%%%%%%%%%%%%%%%%%%%%%%%%%%%%%%%%%%%%%
%
\begin{proof}
%
\begin{align}
 &  {\ex2} (\lambda \x1.  \tr{\A1[\x1]}) \cnj2 \tr{\B1}  \notag \\ 
 \bred \  &  ( \lambda e \phi.  \exists ( \lambda \x1. \A1[\x1] \land \phi (x::e) ) )  \cnj2  \tr{\B1}  \tag{by~\eqref{def:DLCnst-FO-ex}}\\
= \ &  \lambda e \phi. ( \lambda e \phi.  \exists ( \lambda \x1. \A1[\x1] \land \phi (x::e) ) ) e ( \lambda e. \tr{\B1} e \phi  ) \tag{by~\eqref{def:DLCnst-FO-cnj}} \\
\bred \ &   \lambda e \phi. ( \lambda e \phi.  \exists ( \lambda \x1. \A1[\x1] \land \phi (x::e) ) ) e ( \lambda e. \B1 \land  \phi e  )  \tag{by~\eqref{def:DTerms-FO-P}}\\
\bred \ &   \lambda e \phi.  \exists ( \lambda \x1. \A1[\x1] \land  (  \B1 \land  \phi  (x::e)  ) ) \label{eq:DLGexscope1}
\end{align}
%
\begin{align}
 &  \ex2 (\lambda \x1.  \tr{\A1[\x1]}  \cnj2  \tr{\B1} ) \notag \\
  = \ &    \ex2 (\lambda \x1.  \lambda e \phi. \tr{\A1[\x1]}  e (\lambda e. \tr{\B1}  e \phi)  ) \tag{by~\eqref{def:DLCnst-FO-cnj}} \\
  \bred \ &  \ex2 (\lambda \x1.  \lambda e \phi. (\lambda e \phi. \A1[\x1] \land \phi e ) e (\lambda e. \B1 \land  \phi e)  ) \tag{by~\eqref{def:DTerms-FO-P}}\\
    \bred \ &   \ex2 (\lambda \x1.  \lambda e \phi. \A1[\x1] \land  (\B1 \land  \phi e )   ) \\
\bred \ &   \lambda e \phi.  \exists ( \lambda \x1.   \A1[\x1] \land  (  \B1 \land  \phi ( \x1:: e ) ) ) \tag{by~\eqref{def:DLCnst-FO-ex}}
\end{align}
%
\end{proof}

\begin{proposition} For all $\h1$ and $\v1$ of type $o$, and for all $\u1$ of type $\gamma$, the following holds: $$\tr{\h1} \u1 (\lambda e.\v1) =  \h1 \land \v1$$ where $e$ is a variable of type $\gamma$. \label{DL-FO-handv}
\end{proposition}
\begin{proof}
The proof is by induction on the structure of the term $\h1$.
\begin{itemize}
\item $\h1$ is a proposition of the form $\P1 t_1 \dots t_n$.
\begin{align}
\tr{\P1 \t1_1 \dots \t1_n} \u1  (\lambda e.\v1)  = \ & (\lambda e \phi. \P1 \t1_1 \dots \t1_n \land \phi e) \u1  (\lambda e.\v1) \tag{by~\eqref{def:DTerms-FO-P}} \\
\bconv \ & (\lambda  \phi. \P1 \t1_1 \dots \t1_n \land \phi \u1)   (\lambda e.\v1) \notag \\
\bconv \ & \lambda  \phi. \P1 \t1_1 \dots \t1_n \land (\lambda e.\v1) \u1    \notag \\
\bconv \ & \lambda  \phi. \P1 \t1_1 \dots \t1_n \land \v1    \notag 
\end{align}

\item  $\h1$ is a negated proposition $ \neg \w1$. 
\begin{align}
\tr{\neg \w1} \u1 (\lambda e. \v1) = \ &  \n2 \tr{\w1}  \u1 (\lambda e. \v1)  \tag{by~\eqref{def:DTerms-FO-neg}} \\
= \ & ( \lambda e \phi. \neg (\tr{\w1} e (\lambda e'. \top) ) \land \phi e)  \u1 (\lambda e. \v1)  \tag{by~\eqref{def:DLCnst-FO-neg}} \\
\bconv \ & ( \lambda \phi.\neg (\tr{\w1} \u1 (\lambda e'. \top) ) \land \phi \u1 ) (\lambda e. \v1)  \notag \\
\bconv \ & \neg (\tr{\w1} \u1 (\lambda e'. \top) ) \land  (\lambda e. \v1) \u1   \notag \\
\bconv \ & \neg (\tr{\w1} \u1 (\lambda e'. \top) ) \land   \v1   \notag \\
= \ & \neg(\w1 \land \top) \land \v1 \tag{by I.H.} \\
= \ & \neg \w1 \land \v1
\end{align}

\item $\h1$ is a conjunction of two propositions $\w1 \land \z1$.
\begin{align}
(\tr{\w1 \land \z1}) \u1 (\lambda e. \v1)  = \ & (\tr{\w1} \cnj2 \tr{\z1}) \u1 (\lambda e. \v1)   \tag{by~\eqref{def:DTerms-FO-cnj}} \\
 = \ & (\lambda e\phi. \tr{\w1} e (\lambda e'. \tr{\z1} e' \phi) ) \u1 (\lambda e. \v1)   \tag{by~\eqref{def:DLCnst-FO-cnj}} \\
\bconv \ & (\lambda \phi. \tr{\w1} \u1 (\lambda e'. \tr{\z1} e' \phi) ) (\lambda e. \v1)   \notag \\
\bconv \ &  \tr{\w1} \u1 (\lambda e'. \tr{\z1} e'  (\lambda e. \v1) )  \notag \\
= \ &  \tr{\w1} \u1 (\lambda e'. \z1 \land \v1)  \tag{by I.H., $e \notin FV(\v1)$} \\
= \ &  \w1 \land (\z1 \land \v1)  \tag{by I.H., $e' \notin FV(\z1 \land \v1)$} \\
\logeq \ & (\w1 \land \z1) \land \v1 \notag
\end{align}

\item $\h1$ is an existentially quantified formula of the form $\exists( \lambda \x1. \P1 [\x1])$.
\begin{align}
\tr{\exists ( \lambda \x1. \P1 [\x1])} \u1 (\lambda e. \v1)  = \ & ( \ex2 ( \lambda \x1. \tr{\P1 [\x1]})  ) \u1 (\lambda e. \v1)   \tag{by~\eqref{def:DTerms-FO-ex}} \\
 = \ & ( \lambda e \phi. \exists ( \lambda \x1. \P2 [\x1] (x::e) \phi  )) \u1 (\lambda e. \v1)   \tag{by~\eqref{def:DLCnst-FO-ex}} \\
\bconv \ & ( \lambda  \phi. \exists ( \lambda \x1. \P2 [\x1] (x::\u1) \phi  ))  (\lambda e. \v1)  \notag \\
\bconv \ & \exists ( \lambda \x1. \P2 [\x1] (x::\u1) (\lambda e. \v1)  )  \notag \\
= \ & \exists ( \lambda \x1. \P1 [\x1] \land \v1  )  \tag{by I.H.} \\
= \ & \exists ( \lambda \x1. \P1 [\x1]) \land \v1    \tag{$\x1 \notin FV(\v1)$} 
\end{align}

\end{itemize}
\end{proof}

\begin{corollary} \label{cor:DL-FO-handtop} For all propositions $\t1$ of type $o$, and for all terms $\u1$ of type $\gamma$, the following folds: 
\begin{align}\tr{\t1} \u1 (\lambda e. \top) \logeq  \t1 \notag  %\label{eq:DL-FO-handtop}
\end{align}
where $e$ is a variable of type $\gamma$.
 \label{DL-FO-handtop}
\end{corollary}
\begin{proof} Take $\v1$ equal to $\top$ in Proposition~\ref{DL-FO-handv}. Then
$$\tr{\t1} \u1 (\lambda e. \top) = \t1 \land \top \logeq \t1$$
\end{proof}


\begin{definition} A dynamic proposition $\t2$ is true in a model $\mathcal{M}$, denoted
$\mathcal{M}~\models_{dyn}~\t2$, if and only if $\mathcal{M}$ $\models$ $\t2 u (\lambda e. \top)$ for every $u$ of type $\gamma$. \label{def:dyntrueinmodel}
\end{definition}

\begin{theorem}[Conservation] A proposition $\t1$ is true in a model $\mathcal{M}$  if and only if its dynamic variant $\tr{\t1}$ is true in the same model: 
 $$\mathcal{M} \models \t1 \ \text{iff} \ \mathcal{M} \models_{dyn} \tr{\t1}$$
 \label{th:conservation:FO}
\end{theorem}
\begin{proof}

If $\mathcal{M} \models \t1$, then, by Corollary~\ref{cor:DL-FO-handtop}, $\mathcal{M} \models \tr{\t1} \u1 (\lambda e.\top)$. Therefore, by Definition~\ref{def:dyntrueinmodel}, $\mathcal{M} \models_{dyn} \tr{\t1}$.

If $\mathcal{M} \models_{dyn} \tr{\t1}$, then, by Definition~\ref{def:dyntrueinmodel}, $\mathcal{M} \models \tr{\t1} \u1 (\lambda e.\top)$. Therefore, by Corollary~\ref{cor:DL-FO-handtop}, $\mathcal{M} \models \t1$.
\end{proof}

The conservation theorem proves that a static proposition and its dynamic version defined in this section hold in the same models.


\subsection{Donkey Sentences}

Tables~\ref{tbl:stat-FO-donkey} and~\ref{tbl:dyn-FO-donkey} show respectively static and dynamic (according to {\GN}) interpretations for the lexical items in the donkey sentence~\eqref{donkey-every-FO}:
\enumsentence{ \txt{Every farmer who owns a donkey beats it.} \label{donkey-every-FO}}
Note that the type of every dynamic term is analogous to its static type. The only difference is that each atomic type of a dynamic term is dynamized according to Definition~\ref{def:DTypes-FO}. All terms in Table~\ref{tbl:dyn-FO-donkey}, except the interpretation of the pronoun \txt{it}, are dynamized following the rules in Definition~\ref{def:DTerms-FO}. These rules allow the presentation of dynamic terms in a compact way. They ensure that dynamic terms are structurally analogous to their static counterparts and, therefore, are more intuitive. The dynamic interpretation of \txt{it} is constructed not by directly following the dynamization rules, because \txt{it} is a unconventional lexical item: there is an anaphor to be solved. Therefore, $\etr{\I{\txt{it}}}$\footnote{Here and further on, dynamic interpretations of unconventional lexical items are marked with tilde.} contains the selection function $\sel$ that takes a context (from which a referent has to be chosen) as an argument.

 Taking these dynamic interpretations to compute the meaning of Sentence~\eqref{donkey-every-FO}, term~\eqref{eq:FO-donkey-int0} $\beta$-reduces to term~\eqref{eq:FO-donkey-int1}, which normalizes to~\eqref{eq:FO-donkey-int1-equiv}:
\begin{align}
& \tr{\I{beats}}  \ \etr{\I{it}} (\tr{\I{every}}  ((\tr{\I{who}}  (\tr{\I{owns}}  (\tr{\I{a}} \ \tr{\I{donkey}} ) ) ) \tr{\I{farmer}}  ))  \label{eq:FO-donkey-int0} \\
\bred \ &  \all2 (\lambda \x1. (\tr{\textbf{f}} \x1  \cnj2     \ex2 (\lambda \z1.   \tr{\textbf{d}}   \z1  \cnj2      \tr{\textbf{o}}  \x1  \z1 ))    \imp2    (\lambda e \phi. \tr{\textbf{b}}  \x1 ( \sel_{it} e ) e \phi ) )  \label{eq:FO-donkey-int1} \\
\bred \ & \lambda e \phi. \forall (\lambda \x1. \cfarmer \x1  \rightarrow \forall  (\lambda \z1.  (\cdonkey  \z1 \land \cown  \x1 \z1 )   \rightarrow  \cbeat  \x1 ( \sel_{it}  (\upii{\x1}{\upii{\z1}{e}}) )))\land \phi e  \label{eq:FO-donkey-int1-equiv}
\end{align}


Resulting term~\eqref{eq:FO-donkey-int1-equiv} is equivalent to~\eqref{eq:donkey-nf-meaning-2-2006} obtained in framework {\G} interpretations. Indeed, framework {\GN} is equivalent to de Groote's~\cite{deGroote:2006:Towards-a-Montagovian-Account-of-Dynamics} framework {\G}. However, it is advantageous over {\G} due to the compact notations for dynamic terms. These notations significantly systematize the framework and make the interpretations more concise and intuitive. Moreover, the systematic translations of static terms into dynamic terms makes it possible to prove a conservation result~\ref{th:conservation:FO} for {\GN}, that guarantees that static and dynamic interpretations are satisfied in the same models. 




\begin{table}
\begin{tabular}{ l l l l }
  Lexical item &  Static type & Static interpretation \\
  \hline
  \\
  \txt{farmer} &  ${\iota} \rightarrow {o}$ &  $\textbf{f}$   \\
  \txt{donkey} &  ${\iota} \rightarrow {o}$ &  $\textbf{d}$   \\
  \txt{owns} & $((\iota \rightarrow o) \rightarrow o) \rightarrow ((\iota \rightarrow o) \rightarrow o) \rightarrow {o}$  & $ \lambda \Y1 \X1. \X1 ( \lambda \x1. \Y1 (\lambda \y1.  \textbf{o}  \x1 \y1 ))$ \\
    \txt{beats} & $((\iota \rightarrow o) \rightarrow o) \rightarrow ((\iota \rightarrow o) \rightarrow o) \rightarrow {o}$  & $ \lambda \Y1 \X1. \X1 ( \lambda \x1. \Y1 (\lambda \y1.  \textbf{b}  \x1 \y1 ))$ \\
   \txt{every} & $({\iota} \rightarrow {o}) \rightarrow ( ({\iota} \rightarrow {o}) \rightarrow {o}) $ & $\lambda \P1 \Q1. \forall (\lambda x. \P1 x \rightarrow \Q1 x ) $   \\
   \txt{a} & $({\iota} \rightarrow {o}) \rightarrow ( ({\iota} \rightarrow {o}) \rightarrow {o}) $ & $ \lambda \P1 \Q1. \exists (\lambda x.  \P1 x \land \Q1 x )$ \\
  \txt{who} & $( ( ({\iota} \rightarrow {o}) \rightarrow {o} ) \rightarrow o  )  \rightarrow (\iota \rightarrow o)  \rightarrow (\iota \rightarrow o) $ & $ \lambda \R1 \Q1 \x1. \Q1 \x1 \land \R1 (\lambda \P1. \P1 \x1) $  \\
   \txt{it} & $ ({\iota} \rightarrow {o}) \rightarrow {o} $  & $\lambda \P1. \P1 ?$ \\ 
 \end{tabular}
\caption{Static lexical interpretations.} \label{tbl:stat-FO-donkey}
\end{table}
\begin{table}
\begin{tabular}{ l l l l l}
  Lexical item & Dynamic type & Dynamic interpretation in {\GN} \\
  \hline
  \\
  \txt{farmer} &  $\tr{\iota} \rightarrow \tr{o}$ &  $\tr{\textbf{f}}$  \\
    \txt{donkey} &  $\tr{\iota} \rightarrow \tr{o}$ &  $\tr{\textbf{d}}$  \\
   \txt{owns} & $((\tr{\iota} \rightarrow \tr{o}) \rightarrow \tr{o}) \rightarrow ((\tr{\iota} \rightarrow \tr{o}) \rightarrow \tr{o}) \rightarrow \tr{o}$  & $ \lambda \Y2 \X2. \X2 ( \lambda \x1. \Y2 (\lambda \y1.  \tr{\textbf{o}}  \x1 \y1 ))$ \\
      \txt{beats} & $((\tr{\iota} \rightarrow \tr{o}) \rightarrow \tr{o}) \rightarrow ((\tr{\iota} \rightarrow \tr{o}) \rightarrow \tr{o}) \rightarrow \tr{o}$  & $ \lambda \Y2 \X2. \X2 ( \lambda \x1. \Y2 (\lambda \y1.  \tr{\textbf{b}}  \x1 \y1 ))$ \\
   \txt{every} & $(\tr{\iota} \rightarrow \tr{o}) \rightarrow ( (\tr{\iota} \rightarrow \tr{o}) \rightarrow \tr{o}) $ & $\lambda \P2 \Q2. \all2 (\lambda \x1.  \P2 \x1  \imp2 \Q2 \x1 ) $ \\
    \txt{a} &  $(\tr{\iota} \rightarrow \tr{o}) \rightarrow ( (\tr{\iota} \rightarrow \tr{o}) \rightarrow \tr{o}) $  & $ \lambda \P2 \Q2. \ex2 (\lambda \x1.  \P2 \x1  \cnj2   \Q2 \x1 )$ \\
   \txt{who} & $( ( (\tr{\iota} \rightarrow \tr{o}) \rightarrow \tr{o} ) \rightarrow \tr{o}  )  \rightarrow (\tr{\iota} \rightarrow \tr{o})  \rightarrow (\tr{\iota} \rightarrow \tr{o}) $ & $\lambda \R2 \Q2 \x1. \Q2 \x1  \cnj2  \R2 (\lambda \P2. \P2 \x1 ) $\\
      \txt{it} & $ \lambda \P2. \lambda e \phi. \P2 ( \sel_{it} e ) e \phi $ \\ 
   \end{tabular}
\caption{Dynamic lexical interpretations in framework {\GN}.} \label{tbl:dyn-FO-donkey}
\end{table}
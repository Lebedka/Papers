\section{Comparison with related work} \label{sec:dpl}

Constants, variables, reference markers: $\mathsf{C}, \mathsf{V}, \mathsf{M}$

model: $\mathscr{M} = \langle D, \mathcal{I}\, \rangle$

left contexts : $\env = D^{\mathsf{M}}$

continuations : $\cont = 2^\env$ 

nota: a tdl term may contain ref. markers (this stands for $\sel{}$)

DPL formula vs extended  DPL formula.

\begin{align*}
\dplsem{\eta}{\app{\app{\app{R}{t_1}}{\ldots}}{t_n}} &= 
\{\langle g,h\rangle  :
h=g \,\wedge\, 
\langle 
\llbracket t_1 \rrbracket_{\eta,g},
\dots,
\llbracket t_n \rrbracket_{\eta,g}
\rangle \in \mathcal{I}(R)
\}
\\
\dplsem{\eta}{\conj{P}{Q}} &= 
\{\langle g,h\rangle  : 
\equant{k}{
\langle g,k\rangle\in \dplsem{\eta}{P}
\,\wedge\,
\langle k,h\rangle\in \dplsem{\eta}{Q}
}\}
\\
\dplsem{\eta}{\nega{P}} &=
\{\langle g,h\rangle  : 
h=g \,\wedge\, 
\uquant{k}{\langle g,k \rangle\not\in\dplsem{\eta}{P}}
\}
\\
\dplsem{\eta}{\equant{\mathbf{i}}{P}} &=
\{\langle g,h\rangle  : 
\equant{k}{
k[\mathbf{i}]g \,\wedge\,
\langle k,h \rangle \in \dplsem{\eta}{P}
}\} 
\end{align*}

\begin{align*}
\synlift{c}_e &= c\\
\synlift{x}_e &= x\\
\synlift{\mathbf{i}}_e &= \app{\mathsf{sel}_i}{e}\\
\end{align*}

\begin{align*}
\synlift{\app{\app{\app{R}{t_1}}{\ldots}}{t_n}} &= 
\abs{e\phi}{\conj{
\app{\app{\app{R}{
\synlift{t_1}_e
}}{\ldots}}{
\synlift{t_n}_e}
}{\app{\phi}{e}}}
\\
\synlift{\conj{P}{Q}} &= 
\dconj{\synlift{P}}{\synlift{Q}}
\\
\synlift{\nega{P}} &= \dnega{\synlift{P}}
\\
\synlift{\equant{\mathbf{i}}{P}} &=
\dynexists_{\mathbf{i}} x.\,\synlift{P[\mathbf{i}{:=}x]}
\end{align*}

\begin{align*}
\tdlsem{\eta}{\app{\app{\app{R}{t_1}}{\ldots}}{t_n}} &= 
\{\langle g,H\rangle   :
g \in H \, \wedge \,
\langle 
\llbracket t_1 \rrbracket_{\eta,g},
\dots,
\llbracket t_n \rrbracket_{\eta,g}
\rangle \in \mathcal{I}(R)
\}
\\
\tdlsem{\eta}{\dconj{P}{Q}} &= 
\{\langle g,H\rangle  :
\langle g, 
\{h : \langle h, H\rangle \in \tdlsem{\eta}{Q}\}
\rangle\in\tdlsem{\eta}{P}
\}
\\
\tdlsem{\eta}{\dnega{P}} &=
\{\langle g,H\rangle  :
g \in H \, \wedge \,
\langle g,\env\rangle\not\in \tdlsem{\eta}{P}
\}
\\
\tdlsem{\eta}{\dynexists_{\mathbf{i}} x.\, P} &=
\{\langle g,H\rangle  :
\equant{a}{\langle g[\mathbf{i}{:=}a],H\rangle\in\tdlsem{\eta[x{:=}a]}{P}}
\} 
\end{align*}

$$
\semlift{\mathrm{R}} = \{\langle a,B\rangle : 
\equant{b}{\conj{b\in B}{\langle a,b\rangle\in R}}\}
$$

\begin{fact}\label{updateenvir}
$h[\mathbf{i}]g$ iff $\equant{a}{h=g[\mathbf{i}{:=}a]}$
\end{fact}

\begin{lemma}\label{substlemma1}
Let $t$ be a term, and let $x$ be a variable that does not occur in $t$.
For all reference marker $\mathbf{i}$, all assignment $g$, and all valuation
$\eta$, the following holds:
$$
\llbracket t \rrbracket_{\eta,g} =
\llbracket t[\mathbf{i}{:=}x] \rrbracket_{\eta[x{:=}(\app{g}{\mathbf{i}})],g}
$$
\begin{proof}
If $t$ is a constant or a reference marker different from $\mathbf{i}$,
the result is immediate.
If $t$ is a variable $y$, we have that $y\not= x$. Consequently,
$\app{\eta}{y} = \app{\eta[x{:=}(\app{g}{i})]}{y}$.
Finally, if $t$ is $\mathbf{i}$, we have
$\llbracket \mathbf{i} \rrbracket_{\eta,g} =
\app{g}{\mathbf{i}} = 
\app{\eta[x{:=}(\app{g}{i})]}{x} = 
\llbracket x \rrbracket_{\eta[x{:=}(\app{g}{\mathbf{i}})],g} =
\llbracket\mathbf{i}[\mathbf{i}{:=}x] \rrbracket_{\eta[x{:=}(\app{g}{\mathbf{i}})],g}
$.
\end{proof}
\end{lemma}

\begin{lemma}\label{substlemma2}
Let $P$ be an extended DPL formula, and let $x$ be a variable that does not 
occur in $P$.
For all reference marker $\mathbf{i}$, all pair of assignments 
$\langle g, h \rangle$, and all valuation
$\eta$, the following holds:
$$
\langle g,h\rangle \in \dplsem{\eta}{P}
\mbox{ if and only if }
\langle g,h\rangle \in 
\dplsem{\eta[x{:=}(\app{g}{\mathbf{i}})]}{P[\mathbf{i}{:=}x]}
$$
\begin{proof}
A straightforward induction on the structure of $P$, using 
Lemma~\ref{substlemma1} for the base case.
\end{proof}
\end{lemma}

\begin{lemma}\label{commutlemma}
Let $P$ be an extended DPL formula, and $\eta$ be a valuation.
Then, the following holds:
$$
\semlift{\dplsem{\eta}{P}} = \tdlsem{\eta}{\synlift{P}}
$$
\begin{proof}
By induction on the structure of $P$.

\mbox{}

\noindent
1. $P \equiv \app{\app{\app{R}{t_1}}{\ldots}}{t_n}$.
\begin{align*}
&\semlift{\dplsem{\eta}{\app{\app{\app{R}{t_1}}{\ldots}}{t_n}}}\\
&\quad=
\{\langle g,H\rangle : 
\equant{h}{\conj{h\in H}{\langle g,h\rangle\in 
\dplsem{\eta}{\app{\app{\app{R}{t_1}}{\ldots}}{t_n}}}}\}
\\
&\quad=
\{\langle g,H\rangle : 
\equant{h}{
\conj{h\in H}{
\conj{h=g}{ 
\langle 
\llbracket t_1 \rrbracket_{\eta,g},
\dots,
\llbracket t_n \rrbracket_{\eta,g}
\rangle \in \mathcal{I}(R)}
}}\}
\\
&\quad=
\{\langle g,H\rangle : 
\conj{g\in H}{
\langle 
\llbracket t_1 \rrbracket_{\eta,g},
\dots,
\llbracket t_n \rrbracket_{\eta,g}
\rangle \in \mathcal{I}(R)
}\}
\\
&\quad=
\tdlsem{\eta}{\app{\app{\app{R}{t_1}}{\ldots}}{t_n}}
\\
&\quad=
\tdlsem{\eta}{\synlift{\app{\app{\app{R}{t_1}}{\ldots}}{t_n}}}
\end{align*}

\noindent
2. $P \equiv \conj{P_1}{P_2}$.
\begin{align*}
&\semlift{\dplsem{\eta}{\conj{P_1}{P_2}}}\\
&\quad=
\{\langle g,H\rangle : 
\equant{h}{\conj{h\in H}{\langle g,h\rangle\in
\dplsem{\eta}{\conj{P_1}{P_2}}
}}\}
\\
&\quad=
\{\langle g,H\rangle : 
\equant{h}{\conj{h\in H}{
\equant{k}{
\conj{\langle g,k\rangle\in \dplsem{\eta}{P_1}}{
\langle k,h\rangle\in \dplsem{\eta}{P_2}}}}}\}
\\
&\quad=
\{\langle g,H\rangle : 
\equant{k}{
\conj{\langle g,k\rangle\in \dplsem{\eta}{P_1}}{
\equant{h}{ 
\conj{h\in H}{
\langle k,h\rangle\in \dplsem{\eta}{P_2}}}}}
\}
\\
&\quad=
\{\langle g,H\rangle : 
\equant{k}{
\conj{\langle g,k\rangle\in \dplsem{\eta}{P_1}}{
\langle k,H\rangle\in \semlift{\dplsem{\eta}{P_2}}}
}\}
\\
&\quad=
\{\langle g,H\rangle : 
\equant{k}{
\conj{k\in\{k : \langle k,H\rangle\in \semlift{\dplsem{\eta}{P_2}}\}}{
\langle g,k\rangle\in \dplsem{\eta}{P_1}}}
\}
\\
&\quad=
\{\langle g,H\rangle : 
\langle g,
\{k : \langle k,H\rangle\in \semlift{\dplsem{\eta}{P_2}}\}
\rangle\in \semlift{\dplsem{\eta}{P_1}}
\}
\\
&\quad=
\{\langle g,H\rangle : 
\langle g,
\{k : \langle k,H\rangle\in \tdlsem{\eta}{\synlift{P_2}}\}
\rangle\in \tdlsem{\eta}{\synlift{P_1}}
\}
\\
&\makebox[.92\linewidth][r]{(by induction hypothesis)}\\
&\quad=
\tdlsem{\eta}{\dconj{\synlift{P_1}}{\synlift{P_2}}}
\\
&\quad=
\tdlsem{\eta}{\synlift{\conj{P_1}{P_2}}}
\end{align*}

3. $P \equiv \nega{P_1}$.
\begin{align*}
&\semlift{\dplsem{\eta}{\nega{P_1}}}\\
&\quad=
\{\langle g,H\rangle : 
\equant{h}{\conj{h\in H}{\langle g,h\rangle\in
\dplsem{\eta}{\nega{P_1}}
}}\}\\
&\quad=
\{\langle g,H\rangle : 
\equant{h}{\conj{h\in H}{
\conj{h=g}{
\uquant{k}{\langle g,k \rangle\not\in\dplsem{\eta}{P_1}}}
}}\}\\
&\quad=
\{\langle g,H\rangle : 
\conj{g\in H}{
\uquant{k}{\langle g,k \rangle\not\in\dplsem{\eta}{P_1}}
}\}\\
&\quad=
\{\langle g,H\rangle : 
\conj{g\in H}{
\nega{\equant{k}{\conj{k\in\env}{\langle g,k \rangle\in\dplsem{\eta}{P_1}}}}
}\}\\
&\quad=
\{\langle g,H\rangle : 
\conj{g\in H}{
\langle g,\env \rangle\not\in\semlift{\dplsem{\eta}{P_1}}
}\}\\
&\quad=
\{\langle g,H\rangle : 
\conj{g\in H}{
\langle g,\env \rangle\not\in\tdlsem{\eta}{\synlift{P_1}}
}\}\\
&\makebox[.92\linewidth][r]{(by induction hypothesis)}\\
&\quad=
\tdlsem{\eta}{\dnega{\synlift{P_1}}}
\\
&\quad=
\tdlsem{\eta}{\synlift{\nega{P_1}}}
\end{align*}

4. $P \equiv \equant{\mathbf{i}}{P_1}$.
\begin{align*}
&\semlift{\dplsem{\eta}{\equant{\mathbf{i}}{P_1}}}\\
&\quad=
\{\langle g,H\rangle : 
\equant{h}{\conj{h\in H}{\langle g,h\rangle\in 
\dplsem{\eta}{\equant{\mathbf{i}}{P_1}}
}}\}\\
&\quad=
\{\langle g,H\rangle : 
\equant{h}{\conj{h\in H}{
\equant{k}{
\conj{k[\mathbf{i}]g}{
\langle k,h \rangle \in \dplsem{\eta}{P_1}}}
}}\}\\
&\quad=
\{\langle g,H\rangle : 
\equant{h}{\conj{h\in H}{
\equant{k}{
\conj{
\equant{a}{k=g[\mathbf{i}{:=}a]}
}{
\langle k,h \rangle \in \dplsem{\eta}{P_1}}}
}}\}\\
&\makebox[.92\linewidth][r]{(by Fact \ref{updateenvir})}\\
&\quad=
\{\langle g,H\rangle : 
\equant{a}{
\equant{h}{\conj{h\in H}{
\equant{k}{
\conj{
k=g[\mathbf{i}{:=}a]
}{
\langle k,h \rangle \in \dplsem{\eta}{P_1}}}
}}}\}\\
&\quad=
\{\langle g,H\rangle : 
\equant{a}{\equant{h}{\conj{h\in H}{
\langle g[\mathbf{i}{:=}a],h \rangle \in 
\dplsem{\eta}{P_1}}
}}\}\\
&\quad=
\{\langle g,H\rangle : 
\equant{a}{\equant{h}{\conj{h\in H}{
\langle g[\mathbf{i}{:=}a],h \rangle \in 
\dplsem{\eta[x{:=}a]}{P_1[\mathbf{i}{:=}x]}}
}}\}\\
&\makebox[.92\linewidth][r]{(by Lemma \ref{substlemma2})}\\
&\quad=
\{\langle g,H\rangle : 
\equant{a}{
\langle g[\mathbf{i}{:=}a],H \rangle \in 
\semlift{\dplsem{\eta[x{:=}a]}{P_1[\mathbf{i}{:=}x]}}
}\}\\
&\quad=
\{\langle g,H\rangle : 
\equant{a}{
\langle g[\mathbf{i}{:=}a],H \rangle \in 
\tdlsem{\eta[x{:=}a]}{\synlift{P_1[\mathbf{i}{:=}x]}}
}\}\\
&\makebox[.92\linewidth][r]{(by induction hypothesis)}\\
&\quad=
\tdlsem{\eta}{\dynexists_{\mathbf{i}} x.\,\synlift{P_1[\mathbf{i}{:=}x]}}\\
&\quad=
\tdlsem{\eta}{\synlift{\equant{\mathbf{i}}{P_1}}}\\
\end{align*}
\end{proof}
\end{lemma}

\begin{proposition}
Let $P$ be a DPL formula, and $g$ be an assignment.  
Then, $\dplvalid{g}{P}$
if and only if
$\tdlvalid{g}{\synlift{P}}$.
\begin{proof}
By definition, $\dplvalid{g}{P}$ if and only if 
$\equant{h\in \env}{\langle g, h\rangle \in \dplsem{}{P}}$.
This is equivalent to $\langle g, \env\rangle \in \semlift{\dplsem{}{P}}$.
Then, by Lemma~\ref{commutlemma}, we have
$\langle g, \env\rangle \in \tdlsem{}{\synlift{P}}$.
This is, by definition, $\tdlvalid{g}{\synlift{P}}$.
\end{proof}
\end{proposition}



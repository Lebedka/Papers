\section{Comparison with related work} \label{sec:dpl}

Constants, variables, reference markers: $\mathsf{C}, \mathsf{V}, \mathsf{M}$

continuations : $\cont = 2^\env$ 

DPL formula vs extended  DPL formula.

\subsection{Dynamic Predicate Logic}

Groenendijk and Stokhof~\cite{xxx} introduced Dynamic Predicate Logic 
(DPL, for short) as a compositional alternative to theories such as
Kamp's DRT~\cite{xxx} or Heim's file change semantics~\cite{xxx}. 
In fact, DPL may be seen as the dynamic logic that underlies DRT.
Indeed, the relation between DPL and DRT is rather tight:
DPL provides the same truth conditions and the same antecedent accessibility
constraints as DRT.  For this reason, we restrict comparison with related
theories to a technical comparison between Groenendijk and Stokhof's dynamic
logic and ours. In particular, we will show how DPL can be faithfully embedded
in TDL.

The syntax of DPL is the one of ordinary first-order logic (without functional
symbols).  A term is defined to be either an individual constant or a variable.
An atomic formula, $\app{\app{\app{R}{t_1}}{\ldots}}{t_n}$, consists of a 
$n$-ary relational symbol, $R$, applied to $n$ terms,  $t_1 \ldots t_n$.
Then, formulas are built from atomic formulas by means of conjunction, negation,
and existential quantification.\footnote{We consider disjunction, implication
and universal quantification to be defined using de Morgan's laws.}

A model of DPL, $\mathscr{M} = \langle D, \mathcal{I}\, \rangle$, is a usual
first-order structure. It is made of an non-empty set of individuals, $D$, 
and a function, $\mathcal{I}$, that yields an interpretation of the constants 
and the relational symbols. Thus, for every individual constants $c$, 
$\mathcal{I}(c) \in D$, and for every $n$-ary relational symbol $R$,
$\mathcal{I}(R) \subset D^n$.

Let $\mathsf{C}$ be the set of individual constants, and $\mathsf{M}$ be
the set of variables.  In the sequel, we will call the elements of $\mathsf{M}$
\textit{reference markers} in order not to confuse them with the 
$\lambda$-variables.

The set of environments (a.k.a., assignements) is defined to be 
$\env = D^{\mathsf{M}}$. Thus, an environment is a function that assigns an 
individual to each reference marker. Given such an environement, say $h$,
the interpretation $[t]_h$ of a term $t$ is defined as follows:

\begin{align*}
[t]_h &=
\left\{\!
\begin{array}[c]{ll}
\mathcal{I} (t) & \mbox{if } t\in\mathsf{C}\\
h (t) & \mbox{if } t\in\mathsf{M}
\end{array}
\right.
\end{align*}

Given two environments $h, g \in \env$, and a reference marker $i\in\mathsf{M}$,
the relation $g[i]h$ is defined as follows:
$$
g[i]h \;\mbox{ iff }\;
\uquant{j\in\mathsf{M}}{\imp{j\not=i}{g(j)=h(j)}}
$$

Finally, the interpretation of a dynamic formula is defined to be a relation
between environments as follows. It also explains why DPL 

\begin{align*}
\dplsem{\app{\app{\app{R}{t_1}}{\ldots}}{t_n}} &= 
\{\langle g,h\rangle  :
h=g \,\wedge\, 
\langle 
[ t_1 ]_g,
\dots,
[ t_n ]_g
\rangle \in \mathcal{I}(R)
\}
\\
\dplsem{\conj{P}{Q}} &= 
\{\langle g,h\rangle  : 
\equant{k}{
\langle g,k\rangle\in \dplsem{P}
\,\wedge\,
\langle k,h\rangle\in \dplsem{Q}
}\}
\\
\dplsem{\nega{P}} &=
\{\langle g,h\rangle  : 
h=g \,\wedge\, 
\uquant{k}{\langle g,k \rangle\not\in\dplsem{P}}
\}
\\
\dplsem{\equant{i}{P}} &=
\{\langle g,h\rangle  : 
\equant{k}{
k[i]g \,\wedge\,
\langle k,h \rangle \in \dplsem{P}
}\} 
\end{align*}

Remark that the reference marker $i$ occurs free in the right-hand side of
the last equation.  This shows that reference markers are not quite variables
in the usual sense. It also explains why DPL has a problem of destructive 
assignment.

\subsection{Environments as left contexts}


\begin{align*}
\tdlsem{\eta}{\abs{e\phi}{\app{\app{\app{R}{t_1}}{\ldots}}{t_n}}} &= 
\{\langle g,H\rangle   :
g \in H \, \wedge \,
\langle 
\llbracket t_1 \rrbracket_{\eta[e{:=}g]},
\dots,
\llbracket t_n \rrbracket_{\eta[e{:=}g]}
\rangle \in \mathcal{I}(R)
\}
\\
\tdlsem{\eta}{\dconj{P}{Q}} &= 
\{\langle g,H\rangle  :
\langle g, 
\{h : \langle h, H\rangle \in \tdlsem{\eta}{Q}\}
\rangle\in\tdlsem{\eta}{P}
\}
\\
\tdlsem{\eta}{\dnega{P}} &=
\{\langle g,H\rangle  :
g \in H \, \wedge \,
\langle g,\env\rangle\not\in \tdlsem{\eta}{P}
\}
\\
\tdlsem{\eta}{\dynexists_{i} x.\, P} &=
\{\langle g,H\rangle  :
\equant{a}{\langle g[i{:=}a],H\rangle\in\tdlsem{\eta[x{:=}a]}{P}}
\} 
\end{align*}


\subsection{Syntactic and semantic translation}

\begin{align*}
\synlift{c}_e &= c\\
\synlift{x}_e &= x\\
\synlift{\mathbf{i}}_e &= \app{\mathsf{sel}_i}{e}\\
\end{align*}

\begin{align*}
\synlift{\app{\app{\app{R}{t_1}}{\ldots}}{t_n}} &= 
\abs{e\phi}{\conj{
\app{\app{\app{R}{
\synlift{t_1}_e
}}{\ldots}}{
\synlift{t_n}_e}
}{\app{\phi}{e}}}
\\
\synlift{\conj{P}{Q}} &= 
\dconj{\synlift{P}}{\synlift{Q}}
\\
\synlift{\nega{P}} &= \dnega{\synlift{P}}
\\
\synlift{\equant{\mathbf{i}}{P}} &=
\dynexists_{\mathbf{i}} x.\,\synlift{P[\mathbf{i}{:=}x]}
\end{align*}

$$
\semlift{\mathrm{R}} = \{\langle a,B\rangle : 
\equant{b}{\conj{b\in B}{\langle a,b\rangle\in R}}\}
$$

\begin{fact}\label{updateenvir}
$h[\mathbf{i}]g$ iff $\equant{a}{h=g[\mathbf{i}{:=}a]}$
\end{fact}

\begin{lemma}\label{substlemma1}
Let $t$ be a term, and let $x$ be a variable that does not occur in $t$.
For all reference marker $\mathbf{i}$, all assignment $g$, and all valuation
$\eta$, the following holds:
$$
\llbracket t \rrbracket_{\eta,g} =
\llbracket t[\mathbf{i}{:=}x] \rrbracket_{\eta[x{:=}(\app{g}{\mathbf{i}})],g}
$$
\begin{proof}
If $t$ is a constant or a reference marker different from $\mathbf{i}$,
the result is immediate.
If $t$ is a variable $y$, we have that $y\not= x$. Consequently,
$\app{\eta}{y} = \app{\eta[x{:=}(\app{g}{i})]}{y}$.
Finally, if $t$ is $\mathbf{i}$, we have
$\llbracket \mathbf{i} \rrbracket_{\eta,g} =
\app{g}{\mathbf{i}} = 
\app{\eta[x{:=}(\app{g}{i})]}{x} = 
\llbracket x \rrbracket_{\eta[x{:=}(\app{g}{\mathbf{i}})],g} =
\llbracket\mathbf{i}[\mathbf{i}{:=}x] \rrbracket_{\eta[x{:=}(\app{g}{\mathbf{i}})],g}
$.
\end{proof}
\end{lemma}

\begin{lemma}\label{substlemma2}
Let $P$ be an extended DPL formula, and let $x$ be a variable that does not 
occur in $P$.
For all reference marker $\mathbf{i}$, all pair of assignments 
$\langle g, h \rangle$, and all valuation
$\eta$, the following holds:
$$
\langle g,h\rangle \in \dplsem{\eta}{P}
\mbox{ if and only if }
\langle g,h\rangle \in 
\dplsem{\eta[x{:=}(\app{g}{\mathbf{i}})]}{P[\mathbf{i}{:=}x]}
$$
\begin{proof}
A straightforward induction on the structure of $P$, using 
Lemma~\ref{substlemma1} for the base case.
\end{proof}
\end{lemma}

\begin{lemma}\label{commutlemma}
Let $P$ be an extended DPL formula, and $\eta$ be a valuation.
Then, the following holds:
$$
\semlift{\dplsem{\eta}{P}} = \tdlsem{\eta}{\synlift{P}}
$$
\begin{proof}
By induction on the structure of $P$.

\mbox{}

\noindent
1. $P \equiv \app{\app{\app{R}{t_1}}{\ldots}}{t_n}$.
\begin{align*}
&\semlift{\dplsem{\eta}{\app{\app{\app{R}{t_1}}{\ldots}}{t_n}}}\\
&\quad=
\{\langle g,H\rangle : 
\equant{h}{\conj{h\in H}{\langle g,h\rangle\in 
\dplsem{\eta}{\app{\app{\app{R}{t_1}}{\ldots}}{t_n}}}}\}
\\
&\quad=
\{\langle g,H\rangle : 
\equant{h}{
\conj{h\in H}{
\conj{h=g}{ 
\langle 
\llbracket t_1 \rrbracket_{\eta,g},
\dots,
\llbracket t_n \rrbracket_{\eta,g}
\rangle \in \mathcal{I}(R)}
}}\}
\\
&\quad=
\{\langle g,H\rangle : 
\conj{g\in H}{
\langle 
\llbracket t_1 \rrbracket_{\eta,g},
\dots,
\llbracket t_n \rrbracket_{\eta,g}
\rangle \in \mathcal{I}(R)
}\}
\\
&\quad=
\tdlsem{\eta}{\app{\app{\app{R}{t_1}}{\ldots}}{t_n}}
\\
&\quad=
\tdlsem{\eta}{\synlift{\app{\app{\app{R}{t_1}}{\ldots}}{t_n}}}
\end{align*}

\noindent
2. $P \equiv \conj{P_1}{P_2}$.
\begin{align*}
&\semlift{\dplsem{\eta}{\conj{P_1}{P_2}}}\\
&\quad=
\{\langle g,H\rangle : 
\equant{h}{\conj{h\in H}{\langle g,h\rangle\in
\dplsem{\eta}{\conj{P_1}{P_2}}
}}\}
\\
&\quad=
\{\langle g,H\rangle : 
\equant{h}{\conj{h\in H}{
\equant{k}{
\conj{\langle g,k\rangle\in \dplsem{\eta}{P_1}}{
\langle k,h\rangle\in \dplsem{\eta}{P_2}}}}}\}
\\
&\quad=
\{\langle g,H\rangle : 
\equant{k}{
\conj{\langle g,k\rangle\in \dplsem{\eta}{P_1}}{
\equant{h}{ 
\conj{h\in H}{
\langle k,h\rangle\in \dplsem{\eta}{P_2}}}}}
\}
\\
&\quad=
\{\langle g,H\rangle : 
\equant{k}{
\conj{\langle g,k\rangle\in \dplsem{\eta}{P_1}}{
\langle k,H\rangle\in \semlift{\dplsem{\eta}{P_2}}}
}\}
\\
&\quad=
\{\langle g,H\rangle : 
\equant{k}{
\conj{k\in\{k : \langle k,H\rangle\in \semlift{\dplsem{\eta}{P_2}}\}}{
\langle g,k\rangle\in \dplsem{\eta}{P_1}}}
\}
\\
&\quad=
\{\langle g,H\rangle : 
\langle g,
\{k : \langle k,H\rangle\in \semlift{\dplsem{\eta}{P_2}}\}
\rangle\in \semlift{\dplsem{\eta}{P_1}}
\}
\\
&\quad=
\{\langle g,H\rangle : 
\langle g,
\{k : \langle k,H\rangle\in \tdlsem{\eta}{\synlift{P_2}}\}
\rangle\in \tdlsem{\eta}{\synlift{P_1}}
\}
\\
&\makebox[.92\linewidth][r]{(by induction hypothesis)}\\
&\quad=
\tdlsem{\eta}{\dconj{\synlift{P_1}}{\synlift{P_2}}}
\\
&\quad=
\tdlsem{\eta}{\synlift{\conj{P_1}{P_2}}}
\end{align*}

3. $P \equiv \nega{P_1}$.
\begin{align*}
&\semlift{\dplsem{\eta}{\nega{P_1}}}\\
&\quad=
\{\langle g,H\rangle : 
\equant{h}{\conj{h\in H}{\langle g,h\rangle\in
\dplsem{\eta}{\nega{P_1}}
}}\}\\
&\quad=
\{\langle g,H\rangle : 
\equant{h}{\conj{h\in H}{
\conj{h=g}{
\uquant{k}{\langle g,k \rangle\not\in\dplsem{\eta}{P_1}}}
}}\}\\
&\quad=
\{\langle g,H\rangle : 
\conj{g\in H}{
\uquant{k}{\langle g,k \rangle\not\in\dplsem{\eta}{P_1}}
}\}\\
&\quad=
\{\langle g,H\rangle : 
\conj{g\in H}{
\nega{\equant{k}{\conj{k\in\env}{\langle g,k \rangle\in\dplsem{\eta}{P_1}}}}
}\}\\
&\quad=
\{\langle g,H\rangle : 
\conj{g\in H}{
\langle g,\env \rangle\not\in\semlift{\dplsem{\eta}{P_1}}
}\}\\
&\quad=
\{\langle g,H\rangle : 
\conj{g\in H}{
\langle g,\env \rangle\not\in\tdlsem{\eta}{\synlift{P_1}}
}\}\\
&\makebox[.92\linewidth][r]{(by induction hypothesis)}\\
&\quad=
\tdlsem{\eta}{\dnega{\synlift{P_1}}}
\\
&\quad=
\tdlsem{\eta}{\synlift{\nega{P_1}}}
\end{align*}

4. $P \equiv \equant{\mathbf{i}}{P_1}$.
\begin{align*}
&\semlift{\dplsem{\eta}{\equant{\mathbf{i}}{P_1}}}\\
&\quad=
\{\langle g,H\rangle : 
\equant{h}{\conj{h\in H}{\langle g,h\rangle\in 
\dplsem{\eta}{\equant{\mathbf{i}}{P_1}}
}}\}\\
&\quad=
\{\langle g,H\rangle : 
\equant{h}{\conj{h\in H}{
\equant{k}{
\conj{k[\mathbf{i}]g}{
\langle k,h \rangle \in \dplsem{\eta}{P_1}}}
}}\}\\
&\quad=
\{\langle g,H\rangle : 
\equant{h}{\conj{h\in H}{
\equant{k}{
\conj{
\equant{a}{k=g[\mathbf{i}{:=}a]}
}{
\langle k,h \rangle \in \dplsem{\eta}{P_1}}}
}}\}\\
&\makebox[.92\linewidth][r]{(by Fact \ref{updateenvir})}\\
&\quad=
\{\langle g,H\rangle : 
\equant{a}{
\equant{h}{\conj{h\in H}{
\equant{k}{
\conj{
k=g[\mathbf{i}{:=}a]
}{
\langle k,h \rangle \in \dplsem{\eta}{P_1}}}
}}}\}\\
&\quad=
\{\langle g,H\rangle : 
\equant{a}{\equant{h}{\conj{h\in H}{
\langle g[\mathbf{i}{:=}a],h \rangle \in 
\dplsem{\eta}{P_1}}
}}\}\\
&\quad=
\{\langle g,H\rangle : 
\equant{a}{\equant{h}{\conj{h\in H}{
\langle g[\mathbf{i}{:=}a],h \rangle \in 
\dplsem{\eta[x{:=}a]}{P_1[\mathbf{i}{:=}x]}}
}}\}\\
&\makebox[.92\linewidth][r]{(by Lemma \ref{substlemma2})}\\
&\quad=
\{\langle g,H\rangle : 
\equant{a}{
\langle g[\mathbf{i}{:=}a],H \rangle \in 
\semlift{\dplsem{\eta[x{:=}a]}{P_1[\mathbf{i}{:=}x]}}
}\}\\
&\quad=
\{\langle g,H\rangle : 
\equant{a}{
\langle g[\mathbf{i}{:=}a],H \rangle \in 
\tdlsem{\eta[x{:=}a]}{\synlift{P_1[\mathbf{i}{:=}x]}}
}\}\\
&\makebox[.92\linewidth][r]{(by induction hypothesis)}\\
&\quad=
\tdlsem{\eta}{\dynexists_{\mathbf{i}} x.\,\synlift{P_1[\mathbf{i}{:=}x]}}\\
&\quad=
\tdlsem{\eta}{\synlift{\equant{\mathbf{i}}{P_1}}}\\
\end{align*}
\end{proof}
\end{lemma}

\begin{proposition}
Let $P$ be a DPL formula, and $g$ be an assignment.  
Then, $\dplvalid{g}{P}$
if and only if
$\tdlvalid{g}{\synlift{P}}$.
\begin{proof}
By definition, $\dplvalid{g}{P}$ if and only if 
$\equant{h\in \env}{\langle g, h\rangle \in \dplsem{}{P}}$.
This is equivalent to $\langle g, \env\rangle \in \semlift{\dplsem{}{P}}$.
Then, by Lemma~\ref{commutlemma}, we have
$\langle g, \env\rangle \in \tdlsem{}{\synlift{P}}$.
This is, by definition, $\tdlvalid{g}{\synlift{P}}$.
\end{proof}
\end{proposition}



\section{From static to dynamic meaning} \label{sec:examples}

\subsection{Dynamization of static interpretations}

In order to obtain a dynamic meaning in the {\FullName} from a static meaning of a word without linguistic side-effects, one should apply the bar-function introduced in Definition~\ref{def:DTerms-FO} to the static interpretation. The function modifies the structure of atomic, negated and existentialy quantified formulas; as well as of a conjunction of formulas and ignores other subterms of the term, as formalized in Definition~\ref{def:bar-propagation}. The last four rules in~\ref{def:bar-propagation} are exactly the rules from~\ref{def:DTerms-FO}. 

\begin{definition}[] \label{def:bar-propagation} 
Let $\P1$ be a term of type $(\iota_1 \rightarrow \dots \rightarrow \iota_n \rightarrow o)$, $\A1$ and $\B1$ be terms of type $o$, $\t1_1, \dots, \t1_n$ and $\x1$ be terms of type $\iota$, $M$ and $N$ be any terms, and $y$ be a variable. For each term $t$, its dynamic equivalent $\tr{t}$ is constructed according to the following rules:
\begin{subequations}
\begin{align}
\tr{ y} \defeq \ & y  \label{def:bar-var}\\
\tr{ \lambda y. M } \defeq \ & \lambda y. \tr{M} \label{def:bar-abs}\\
\tr{ M N  } \defeq \ & \tr{M} \ \tr{N} \label{def:bar-app} \\
\tr{ \P1 \t1_1 \dots \t1_n } \defeq \ & \lambda e \phi . \P1 \t1_1 \dots \t1_n \land \phi e  \label{def:bar-P}\\
\tr{ \neg \A1  } \defeq \ & \n2 \tr{\A1}  \label{def:bar-neg}\\
\tr{ \A1 \land \B1} \defeq \ & \tr{\A1} \cnj2 \tr{\B1}  \label{def:bar-cnj}\\
\tr{\exists( \lambda \x1. \P1 [\x1])} \defeq \ & \ex2( \lambda  \x1. \tr{\P1 [\x1]})  \label{def:bar-ex} 
\end{align}
\end{subequations} 
\end{definition}

Rule~\ref{def:bar-P} is very important, as this is the only rule that introduces the abstraction over a continuation $\phi$ and a context $e$. Since this rule defines the dinamization of an atomic formula, one should apply the $\eta$-conversion to non-logical constants in the static lexical interpretation before dynamizing them.
Example~\ref{ex:donkey} illustrates this on the interpretation of \txt{donkey}.

\begin{example} \label{ex:donkey}
Given a static interpretation $\I{donkey}$ of \txt{donkey} consisting of a single constant of type $(\iota \rightarrow o)$, as shown in~\eqref{eq:cdonkey}.
In order to obtain its dynamic version, we first apply the $\eta$-conversion to it: 
\begin{align}
\I{donkey} = \ & \cdonkey \label{eq:cdonkey} \\
\econv \ & \lambda x. \cdonkey x %\label{eq:etacdonkey}
\end{align}
Now the bar-function can be applied:
\begin{align}
\tr{\I{donkey}} = \ & \tr{ \lambda x. \cdonkey x } \notag \\
 = \  & \lambda x. \tr{  \cdonkey x } \tag{by~\eqref{def:bar-abs}} \\
  = \  & \lambda x.  \lambda e \phi.  \cdonkey x \land \phi e \tag{by~\eqref{def:bar-P}} 
\end{align}

\end{example}

When a term is complex, Definition~\ref{def:bar-propagation} can be applied directly to it. This is due to the fact that even if the static interpretation contains non-logical constants, they are guaranteed to be applied to the corresponding arguments within the body of the term. Next example shows how the dinamic meaning for \txt{owns} can be systematically obtained.
\begin{example} 
Given the static interpretation for the verb \I{owns}
\begin{align}
\I{owns} = \  \lambda \Y1 \X1. \X1 ( \lambda \x1. \Y1 (\lambda \y1.  \textbf{o}  \x1 \y1 )) \notag %\label{eq:cowns} \\
\end{align}
its dynamic equivalent is obtained as follows:
\begin{align}
\tr{\I{owns}} = \ & \tr{\lambda \Y1 \X1. \X1 ( \lambda \x1. \Y1 (\lambda \y1.  \textbf{o}  \x1 \y1 ))}  \notag \\
= \ &  \lambda \Y1 \X1. \tr{ \X1 ( \lambda \x1. \Y1 (\lambda \y1.  \textbf{o}  \x1 \y1 ))} \tag{by~\eqref{def:bar-abs}} \\
= \ &  \lambda \Y1 \X1. \tr{ \X1 } \ \tr{ ( \lambda \x1. \Y1 (\lambda \y1.  \textbf{o}  \x1 \y1 ))}  \tag{by~\eqref{def:bar-app}} \\
= \ &  \lambda \Y1 \X1.  \X1   ( \lambda \x1. \tr{ \Y1  (\lambda \y1.  \textbf{o}  \x1 \y1 ))}  \tag{by~\eqref{def:bar-var} and~\eqref{def:bar-abs}} \\
= \ &  \lambda \Y1 \X1.  \X1   ( \lambda \x1. \tr{ \Y1 } \ \tr{ (\lambda \y1.  \textbf{o}  \x1 \y1 )})  \tag{by~\eqref{def:bar-app}} \\
= \ &  \lambda \Y1 \X1.  \X1   ( \lambda \x1.  \Y1   (\lambda \y1. \tr{ \textbf{o}  \x1 \y1 }))  \tag{by~\eqref{def:bar-var} and~\eqref{def:bar-abs}}\\
= \ &  \lambda \Y1 \X1.  \X1   ( \lambda \x1.  \Y1   (\lambda \y1. \lambda \phi e. \textbf{o}  \x1 \y1 \land \phi e)) \tag{by~\eqref{def:bar-P}} 
\end{align}

\end{example}

\TODOK{Explain about meanings of words with linguistic side effects.}


\subsection{Computing meaning of a donkey sentence}

Tables~\ref{tbl:stat-FO-donkey} and~\ref{tbl:dyn-FO-donkey} show respectively static and dynamic interpretations for the lexical items in the donkey sentence~\eqref{donkey-every-FO}:
\enumsentence{ \txt{Every farmer who owns a donkey beats it.} \label{donkey-every-FO}}

Observe that the type of every dynamic term is analogous to its static type. The only difference is that each atomic type of a dynamic term is dynamized according to Definition~\ref{def:DTypes-FO}. Moreover dynamic interpretations structurally resemble static interpretations. This is achieved by applying the rules from Definition~\ref{def:DLCnst-FO} to original static interpretations from Table~\ref{tbl:stat-FO-donkey} in order to obtain their dynamic equivalents shown in Table~\ref{tbl:dyn-FO-donkey}. The only exception is the interpretation of the pronoun \txt{it}, which is an unconventional lexical item: there is an anaphor to be resolved. While its static interpretation was uncertain, its dynamic representation integrates the selection function $\selK$ responsible for solving anaphora.

%\footnote{Here and further on, dynamic interpretations of unconventional lexical items are marked with tilde.}

\begin{table}
\begin{tabular}{ l l l l }
  Lexical item &  Static type & Static interpretation \\
  \hline
  \\
  \txt{farmer} &  ${\iota} \rightarrow {o}$ &  $\cfarmer$   \\
  \txt{donkey} &  ${\iota} \rightarrow {o}$ &  $\cdonkey$   \\
  \txt{owns} & $((\iota \rightarrow o) \rightarrow o) \rightarrow ((\iota \rightarrow o) \rightarrow o) \rightarrow {o}$  & $ \lambda \Y1 \X1. \X1 ( \lambda \x1. \Y1 (\lambda \y1.  \textbf{o}  \x1 \y1 ))$ \\
    \txt{beats} & $((\iota \rightarrow o) \rightarrow o) \rightarrow ((\iota \rightarrow o) \rightarrow o) \rightarrow {o}$  & $ \lambda \Y1 \X1. \X1 ( \lambda \x1. \Y1 (\lambda \y1.  \textbf{b}  \x1 \y1 ))$ \\
   \txt{every} & $({\iota} \rightarrow {o}) \rightarrow ( ({\iota} \rightarrow {o}) \rightarrow {o}) $ & $\lambda \P1 \Q1. \forall (\lambda x. \P1 x \rightarrow \Q1 x ) $   \\
   \txt{a} & $({\iota} \rightarrow {o}) \rightarrow ( ({\iota} \rightarrow {o}) \rightarrow {o}) $ & $ \lambda \P1 \Q1. \ex1 (\lambda x.  \P1 x \cnj1 \Q1 x )$ \\
  \txt{who} & $( ( ({\iota} \rightarrow {o}) \rightarrow {o} ) \rightarrow o  )  \rightarrow (\iota \rightarrow o)  \rightarrow (\iota \rightarrow o) $ & $ \lambda \R1 \Q1 \x1. \Q1 \x1 \cnj1 \R1 (\lambda \P1. \P1 \x1) $  \\
   \txt{it} & $ ({\iota} \rightarrow {o}) \rightarrow {o} $  & $\lambda \P1. \P1 ?$ \\ 
 \end{tabular}
\caption{Static lexical interpretations.} \label{tbl:stat-FO-donkey}
\end{table}
\begin{table}
\begin{tabular}{ l l l l l}
  Lexical item & Dynamic type & Dynamic interpretation \\
  \hline
  \\
  \txt{farmer} &  $\tr{\iota} \rightarrow \tr{o}$ &  $\lambda \x1.\tr{\cfarmer\x1}$  \\
    \txt{donkey} &  $\tr{\iota} \rightarrow \tr{o}$ & $\lambda \x1. \tr{\cdonkey \x1}$  \\
   \txt{owns} & $((\tr{\iota} \rightarrow \tr{o}) \rightarrow \tr{o}) \rightarrow ((\tr{\iota} \rightarrow \tr{o}) \rightarrow \tr{o}) \rightarrow \tr{o}$  & $ \lambda \Y2 \X2. \X2 ( \lambda \x1. \Y2 (\lambda \y1.  \tr{\textbf{o}  \x1 \y1 } ))$ \\
      \txt{beats} & $((\tr{\iota} \rightarrow \tr{o}) \rightarrow \tr{o}) \rightarrow ((\tr{\iota} \rightarrow \tr{o}) \rightarrow \tr{o}) \rightarrow \tr{o}$  & $ \lambda \Y2 \X2. \X2 ( \lambda \x1. \Y2 (\lambda \y1.  \tr{\textbf{b}  \x1 \y1 } ))$ \\
   \txt{every} & $(\tr{\iota} \rightarrow \tr{o}) \rightarrow ( (\tr{\iota} \rightarrow \tr{o}) \rightarrow \tr{o}) $ & $\lambda \P2 \Q2. \all2 (\lambda \x1.  \P2 \x1  \impK2 \Q2 \x1 ) $ \\
    \txt{a} &  $(\tr{\iota} \rightarrow \tr{o}) \rightarrow ( (\tr{\iota} \rightarrow \tr{o}) \rightarrow \tr{o}) $  & $ \lambda \P2 \Q2. \ex2 (\lambda \x1.  \P2 \x1  \cnj2   \Q2 \x1 )$ \\
   \txt{who} & $( ( (\tr{\iota} \rightarrow \tr{o}) \rightarrow \tr{o} ) \rightarrow \tr{o}  )  \rightarrow (\tr{\iota} \rightarrow \tr{o})  \rightarrow (\tr{\iota} \rightarrow \tr{o}) $ & $\lambda \R2 \Q2 \x1. \Q2 \x1  \cnj2  \R2 (\lambda \P2. \P2 \x1 ) $\\
      \txt{it} & $ (\tr{\iota} \rightarrow \tr{o}) \rightarrow \tr{o} $ & $ \lambda \P2. \lambda e \phi. \P2 ( \selK_{it} e ) e \phi $ \\ 
   \end{tabular}
\caption{Dynamic lexical interpretations.} \label{tbl:dyn-FO-donkey}
\end{table}


According to the parse tree, shown in Figure~\ref{fig:donkey-copy}, the dynamic meaning of~\eqref{donkey-every-FO} can be computed by $\beta$-reducing the term~\eqref{term:donkey-dynamic}:
\begin{align}
\tr{\I{beats}}  \ \etr{\I{it}} (\tr{\I{every}} ( (\tr{\I{who}}  (\tr{\I{owns}}  (\tr{\I{a}} \ \tr{\I{donkey}} ) ) ) \tr{\I{farmer}}  )) \label{term:donkey-dynamic}
\end{align}
%The dynamic lexical interpretations (in compact form) for words in Sentence~\eqref{donkey-every} are shown in Table~\ref{tbl:donkey-lexical}. These interpretations comply with dynamization principles discussed in Sections~\ref{sec:StandardInterpretations} and~\ref{sec:ExceptionTriggers}. Since the meanings of the lexical items are taken in compact form, the computation of the meaning of the sentence is free of ``bureaucracy'' and more intuitive compared to the computation using the normalized lexical interpretations. 


\begin{figure}[h!]
 \centering
    \includegraphics[width=0.85\textwidth]{images/Donkey.pdf}
\caption{Syntactic parse tree of sentence \txt{Every farmer who owns a donkey beats it}.} \label{fig:donkey-copy}
\end{figure}

\begin{example}[Meaning of \txt{Every farmer who owns a donkey beats it}]

The meaning of the noun phrase \txt{a donkey} is computed by reducing the term $\tr{\I{a}} \ \tr{\I{donkey}} $:
\begin{align*}
 \tr{\I{a}} \ \tr{\I{donkey}}   = \ & (\lambda \P2 \Q2. \ex2 (\lambda \x1.  \P2 \x1 \cnj2 \Q2 \x1 ) ) \tr{\I{donkey}}   \\
 \bconv \ &  \lambda  \Q2. \ex2 (\lambda \x1.  \tr{\I{donkey}} \x1  \cnj2  \Q2 \x1 ) \\
 = \ &  \lambda  \Q2. \ex2 (\lambda \x1. (\lambda \x1.\tr{\cdonkey \x1}) \x1  \cnj2  \Q2 \x1 ) \\
\bconv \ &  \lambda  \Q2. \ex2 (\lambda \x1.  \tr{\cdonkey \x1} \cnj2  \Q2 \x1 )
\end{align*}



The term $ \tr{\I{owns}}  (\tr{\I{a}} \ \tr{\I{donkey}}) $ is $\beta$-reduced as follows:
\begin{align*}
\tr{\I{owns}}  (\tr{\I{a}} \ \tr{\I{donkey}} ) = \ & ( \lambda \Y2 \X2. \X2 ( \lambda \x1. \Y2 (\lambda \y1.  \tr{\cown  \x1 \y1} )))  (\tr{\I{a}} \ \tr{\I{donkey}} ) \\
\bconv \ &   \lambda \X2. \X2 ( \lambda \x1. \tr{\I{a}} \ \tr{\I{donkey}} (\lambda \y1.  \tr{\cown}  \x1 \y1 ))   \\
= \ &  \lambda \X2. \X2 ( \lambda \x1. ( \lambda  \Q2. \ex2 (\lambda \z1.  \tr{\cdonkey \z1}\cnj2 \Q2 \z1 ) ) (\lambda \y1.  \tr{\cown  \x1 \y1} ))   \\
\bconv \ &  \lambda \X2. \X2 ( \lambda \x1. \ex2 (\lambda \z1.  \tr{\cdonkey \z1}\cnj2  (\lambda \y1.  \tr{\cown  \x1 \y1 }) \z1   ))   \\
\bconv \ & \lambda \X2. \X2 ( \lambda \x1.  \ex2 (\lambda \z1. \tr{\cdonkey \z1}\cnj2  \tr{\cown  \x1 \z1 }))  
\end{align*}

The meaning of the relative clause \txt{who owns a donkey} is computed by $\beta$-reducing the term $\tr{\I{who}}  (\tr{\I{owns}}  (\tr{\I{a}} \ \tr{\I{donkey}} ))$:
\begin{align*}
 & \tr{\I{who}}  (\tr{\I{owns}}  (\tr{\I{a}} \ \tr{\I{donkey}} )) \\
 = \ & (\lambda \R2 \Q2 \y1.  \Q2 \y1 \cnj2  \R2 (\lambda \P2. \P2 \y1 ) ) (\tr{\I{owns}}  (\tr{\I{a}} \ \tr{\I{donkey}} )) \\
\bconv \ &  \lambda \Q2 \y1. \Q2 \y1 \cnj2  \tr{\I{owns}}  (\tr{\I{a}} \ \tr{\I{donkey}}) (\lambda \P2. \P2 \y1)   \\
= \ & \lambda \Q2 \y1. \Q2 \y1 \cnj2  ( \lambda \X2. \X2 ( \lambda \x1.  \ex2 (\lambda \z1.  \tr{\cdonkey \z1}\cnj2 \tr{\cown  \x1 \z1 } ) ) )  (\lambda \P2. \P2 \y1)   \\
\bconv \ & \lambda \Q2 \y1.  \Q2 \y1 \cnj2  ( \lambda \P2. \P2 \y1) ( \lambda \x1.  \ex2 (\lambda \z1.  \tr{\cdonkey \z1}\cnj2 \tr{\cown  \x1 \z1 } ) )     \\
\bconv \ & \lambda \Q2 \y1. \Q2 \y1 \cnj2  (   \lambda \x1.  \ex2 (\lambda \z1.   \tr{\cdonkey \z1}\cnj2 \tr{\cown  \x1 \z1 } )    \y1)   \\
\bconv \ & \lambda \Q2 \y1.  \Q2 \y1 \cnj2   \ex2 (\lambda \z1.  \tr{\cdonkey \z1}\cnj2 \tr{\cown  \y1 \z1 } ) 
\end{align*}

The meaning of \txt{farmer who owns a donkey} is computed by applying the resulting $\lambda$-term in the computation above to the interpretation of \txt{farmer}:
\begin{align*}
& (\tr{\I{who}}  (\tr{\I{owns}}  (\tr{\I{a}} \ \tr{\I{donkey}} ) ) )\tr{\I{farmer}} \\
 = \ & (\lambda \Q2 \y1.  \Q2 \y1 \cnj2   \ex2 (\lambda \z1.  \tr{\cdonkey \z1}\cnj2 \tr{\cown \y1 \z1}  )) \tr{\I{farmer}} \\
\bconv \ & \lambda  \y1.  \tr{\I{farmer}} \y1 \cnj2   \ex2 (\lambda \z1.  \tr{\cdonkey \z1}\cnj2 \tr{\cown \y1 \z1}  )   \\ 
\bconv \ & \lambda  \y1.  (\lambda \x1.\tr{\cfarmer\x1}) \y1 \cnj2   \ex2 (\lambda \z1.  \tr{\cdonkey \z1}\cnj2 \tr{\cown \y1 \z1}  )   \\ 
\bconv \ & \lambda  \y1.  \tr{\cfarmer \y1} \cnj2   \ex2 (\lambda \z1.  \tr{\cdonkey \z1}\cnj2 \tr{\cown \y1 \z1} )   
\end{align*}

The meaning of the noun phrase \txt{every farmer who owns a donkey} is computed by applying the interpretation of \txt{every} to the interpretation of \txt{farmer who owns a donkey}: 
\begin{align}
& \tr{\I{every}}  ((\tr{\I{who}}  (\tr{\I{owns}}  (\tr{\I{a}} \ \tr{\I{donkey}} ) )) \tr{\I{farmer}}  )  \notag \\
= \ & (\lambda \P2 \Q2. \all2 (\lambda \x1.  \P2 \x1 \impK2 \Q2 \x1 ) ) ((\tr{\I{who}}  (\tr{\I{owns}}  (\tr{\I{a}} \ \tr{\I{donkey}} ) )) \tr{\I{farmer}}  )  \notag \\
\bconv \ & \lambda \Q2. \all2 (\lambda \x1.  ((\tr{\I{who}}  (\tr{\I{owns}}  (\tr{\I{a}} \ \tr{\I{donkey}} ) )) \tr{\I{farmer}}  ) \x1 \impK2 \Q2 \x1 )   \notag \\
= \ & \lambda \Q2. \all2 (\lambda \x1.  (\lambda  \y1.  \tr{\cfarmer \y1} \cnj2   \ex2 (\lambda \z1. \tr{\cdonkey \z1} \cnj2  \tr{\cown \y1 \z1}  )   ) \x1 \impK2 \Q2 \x1)    \notag \\
\bconv \ & \lambda \Q2. \all2 (\lambda \x1.  ( \tr{\cfarmer \x1} \cnj2   \ex2 (\lambda \z1.  \tr{\cdonkey \z1} \cnj2  \tr{\cown \x1 \z1}  ) )  \impK2 \Q2 \x1)   \label{eq:everyfarmerwhoownsadonkey}
\end{align}

The meaning of the verb phrase \txt{beats it} is computed as follows:
\begin{align}
\tr{\I{beats}}  \etr{\I{it}}  = \ &  (\lambda \Y2 \X2. \X2 ( \lambda \x1. \Y2 (\lambda \y1.  \tr{\cbeat \x1 \y1} ))) \etr{\I{it}} \notag  \\
\bconv \ &  \lambda  \X2. \X2 ( \lambda \x1.  \etr{\I{it}}  (\lambda \y1.  \tr{\cbeat \x1 \y1} ))\notag  \\
= \ & \lambda  \X2. \X2 ( \lambda \x1.  ( \lambda \P2. \lambda e \phi. \P2 ( \selK_{it} e ) e \phi)    (\lambda \y1.  \tr{\cbeat \x1 \y1} ))\notag  \\
\bconv \ & \lambda  \X2. \X2 ( \lambda \x1.   \lambda e \phi.  (\lambda \y1.  \tr{\cbeat \x1 \y1} ) ( \selK_{it} e ) e \phi  )\notag  \\
\bconv \ & \lambda  \X2. \X2 ( \lambda \x1.   \lambda e \phi.  \tr{\cbeat \x1 ( \selK_{it} e )} e \phi  ) \label{eq:beatsit}
\end{align}

Finally, the meaning of the sentence is computed by applying the interpretation~\eqref{eq:beatsit} to the interpretation~\eqref{eq:everyfarmerwhoownsadonkey}:
\begin{flalign}
& \tr{\I{beats}}  \ \etr{\I{it}} (\tr{\I{every}}  ((\tr{\I{who}}  (\tr{\I{owns}}  (\tr{\I{a}} \ \tr{\I{donkey}} ) ) ) \tr{\I{farmer}}  )) & \notag \\
= \ & (\lambda  \X2. \X2 ( \lambda \x1.   \lambda e \phi.  \tr{\cbeat \x1 ( \selK_{it} e )} e \phi  ) ) (\tr{\I{every}}  ((\tr{\I{who}}  (\tr{\I{owns}}  (\tr{\I{a}} \ \tr{\I{donkey}} ) ) ) \tr{\I{farmer}}  )) &  \notag \\
\bconv \ &  (\tr{\I{every}}  ((\tr{\I{who}}  (\tr{\I{owns}}  (\tr{\I{a}} \ \tr{\I{donkey}} ) ) ) \tr{\I{farmer}}  )) ( \lambda \x1.   \lambda e \phi.  \tr{\cbeat \x1 ( \selK_{it} e )} e \phi  )   &  \notag \\
= \ & (\lambda \Q2. \all2 (\lambda \x1.  ( \tr{\cfarmer \x1} \cnj2   \ex2 (\lambda \z1.  \tr{\cdonkey \z1} \cnj2  \tr{\cown \x1 \z1}  ) )  \impK2 \Q2 \x1) ) ( \lambda \x1.   \lambda e \phi.  \tr{\cbeat \x1 ( \selK_{it} e )} e \phi  )   &  \notag \\
\bconv \ &  \all2 (\lambda \x1.  ( \tr{\cfarmer \x1} \cnj2   \ex2 (\lambda \z1.  \tr{\cdonkey \z1} \cnj2  \tr{\cown \x1 \z1}  ) )  \impK2 ( \lambda \x1.   \lambda e \phi.  \tr{\cbeat \x1 ( \selK_{it} e )} e \phi  )   \x1)  ) &  \notag \\
\bconv \ &  \all2 (\lambda \x1.  ( \tr{\cfarmer \x1} \cnj2   \ex2 (\lambda \z1.  \tr{\cdonkey \z1} \cnj2  \tr{\cown \x1 \z1}  ) )  \impK2 (  \lambda e \phi.  \tr{\cbeat \x1 ( \selK_{it} e )} e \phi ) ) &    \label{eq:donkey-compact-meaning}
\end{flalign}

\end{example}

Formula~\eqref{eq:donkey-compact-meaning} is the dynamic interpretation of Sentence~\eqref{donkey-every-FO}. 
Due to the compact forms of dynamic (sub-)terms, the computation and the resulting term are as intuitive as if static interpretations were used. However, the dynamic interpretation is advantageous. Particularly for donkey sentences, it not only provides means for compositionally solving the pronominal anaphora, but also avoids the quantifier scope problem. 

To illustrate that the quantifier scope problem is avoided two analyses are given below. We apply these analyses to 
term~\eqref{eq:donkey-compact-meaning-2} obtained from~\eqref{eq:donkey-compact-meaning}  using Remark~\ref{remark:impforall}. Interpretations~\eqref{eq:donkey-compact-meaning-2} and~\eqref{eq:donkey-compact-meaning} are logically equivalent.
\begin{align}
\n2 \ex2 (\lambda \x1.  \tr{\cfarmer \x1} \cnj2   \ex2 (\lambda \z1.  \tr{\cdonkey \z1}\cnj2 \tr{\cown \x1 \z1}  )  \cnj2  \n2 ( \lambda e \phi. \tr{\cbeat \x1  ( \selK_{it} e )} e \phi  ))   \label{eq:donkey-compact-meaning-2}
\end{align}

In the first analysis we simply apply Proposition~\ref{thrm:scopeextension-G} to~\eqref{eq:donkey-compact-meaning-2} and in the second analysis we examine the structure of the term and its dynamic subterms. %The third analysis normalizes the term by performing step-by-step $\beta$-reductions.

\subsection*{Analysis 1}


According to Proposition~\ref{thrm:scopeextension-G}, since $\z1 \notin FV(\n2 \tr{\cbeat}  \x1 (( \selK_{it} e ) e \phi))$, formula~\eqref{eq:donkey-compact-meaning-2} is $\beta$-equivalent to the following formula:
\begin{align}
\n2 \ex2 (\lambda \x1.  \tr{\cfarmer \x1} \cnj2   \ex2 (\lambda \z1.  \tr{\cdonkey \z1}\cnj2 \tr{\cown \x1 \z1}    \cnj2  \n2  ( \lambda e \phi. \tr{\cbeat \x1 ( \selK_{it} e )} e \phi ) )) \label{eq:donkey-compact-meaning-3}
\end{align}

Finally, by Remark~\eqref{remark:impforall}, formula~\eqref{eq:donkey-compact-meaning-3} can be represented in a more familiar way:
\begin{align}
 \all2 (\lambda \x1.  \tr{\cfarmer \x1} \impK2   \all2 (\lambda \z1. ( \tr{\cdonkey \z1}\cnj2 \tr{\cown \x1 \z1} )   \impK2    ( \lambda e \phi. \tr{\cbeat \x1 ( \selK_{it} e )} e \phi ) )) 
 \label{eq:donkey-compact-meaning-4}
\end{align}


\subsection*{Analysis 2}


Although it is sufficient to apply Proposition~\ref{thrm:scopeextension-G} to see that indeed~\eqref{eq:donkey-compact-meaning} is the desirable interpretation of the donkey sentence, it is worth to inspect the term itself for gaining more intuition why this is so.

In order to see that the proposition $\tr{\cbeat \x1 ( \selK_{it} e )}$ enters inside the scope of the existential quantifier related to \txt{a donkey} in~\eqref{eq:donkey-compact-meaning-2}, it is important to understand how the continuation of the subterm $\tr{\cfarmer \x1} \cnj2 \ex2 (\lambda \z1.  \tr{\cdonkey \z1}\cnj2 \tr{\cown \x1 \z1}  )$ ends up inside the scope of the existential quantifier. 

Recall formula~\eqref{def:DLCnst-FO-cnj}, which defines the dynamic conjunction in a way that its second argument $\B2$ is a subterm of the continuation $(\lambda e. \B2 e \phi)$ of the first argument $\A2$. As a result, the continuation of the term $\A2 \cnj2 \B2$ itself is deep inside the formula. Particularly, in the subterm $ \tr{\cdonkey \z1}\cnj2 \tr{\cown \x1 \z1} $, the term $\tr{\cown \x1 \z1}$ is a subterm of the continuation of $\tr{\cdonkey \z1}$. Consequently, the continuation $\phi$ of the term $\tr{\cdonkey \z1}\cnj2 \tr{\cown \x1 \z1}$ is deep inside the formula, as shown in~\eqref{eq:app1}.  
Consequently, arguments of $ \ex2 (\lambda \z1.  \tr{\cdonkey \z1}\cnj2 \tr{\cown \x1 \z1}  ) $ will end up getting bound by the existential quantifier.
\begin{align}
\tr{\cdonkey \z1}\cnj2 \tr{\cown \x1 \z1} \bred  \lambda e \phi. \cdonkey \z1 \cnj1 ( \cown \x1  \z1 \cnj1 \phi e)  \label{eq:app1}
\end{align}
 Now, consider Formula~\eqref{def:DLCnst-FO-ex}. $\ex2$ applied to $ ( \lambda \z1. \tr{\cdonkey \z1}\cnj2 \tr{\cown \x1 \z1})$ results in a formula shown in~\eqref{eq:app2}. Note that the continuation remains deep inside the formula and, what is the most important, it occurs \emph{inside the scope of the existential quantifier}:
 \begin{align}
 \ex2 (\lambda \z1.  \tr{\cdonkey \z1}\cnj2 \tr{\cown \x1 \z1}  ) \bred  \lambda e \phi. \ex1 ( \lambda \z1. \cdonkey \z1  \cnj1 ( \cown \x1 \z1 \cnj1 \phi (\z1::e)) )  \label{eq:app2}
 \end{align}

Consider now the dynamic conjunction $ \tr{\cfarmer \x1}  \cnj2 \ex2 (\lambda \z1.  \tr{\cdonkey \z1}\cnj2 \tr{\cown \x1 \z1}  )$. By~\eqref{def:DLCnst-FO-cnj}, the whole term $\ex2 (\lambda \z1.  \tr{\cdonkey \z1}\cnj2 \tr{\cown \x1 \z1}  ) $ is a subterm of the continuation of $\tr{\cfarmer \x1} $. The term normalizes as~\eqref{eq:app3}. Importantly, the continuation $\phi$ of the term continues to be deep inside the formula and \emph{inside} \emph{the scope of the existential quantifier}.
\begin{align}
 \tr{\cfarmer \x1}  \cnj2 \ex2 (\lambda \z1.  \tr{\cdonkey \z1}\cnj2 \tr{\cown \x1 \z1}  )  \bred  \lambda e \phi. \cfarmer \x1  \cnj1 \ex1 ( \lambda \z1. \cdonkey \z1  \cnj1 ( \cown \x1  \z1 \cnj1 \phi  (\z1::e) ) )  \label{eq:app3}
\end{align}

Next the subterms $(\tr{\cfarmer \x1} \cnj2  \ex2 (\lambda \z1.  \tr{\cdonkey \z1}\cnj2 \tr{\cown \x1 \z1}  ) )$ and $ \n2 ( \lambda e \phi. \tr{\cbeat \x1  ( \selK_{it} e )} e \phi  ) $ are dynamically conjoint. Similarly to above, by~\eqref{def:DLCnst-FO-cnj}, the term $ \n2 ( \lambda e \phi. \tr{\cbeat \x1  ( \selK_{it} e )} e \phi  )$ is a subterm of the continuation of the term $(\tr{\cfarmer \x1}  \cnj2   \ex2 (\lambda \z1.  \tr{\cdonkey \z1}\cnj2 \tr{\cown \x1 \z1}  ) ) $.  As shown in~\eqref{eq:app3}, the continuation $\phi$ is under the scope of the existential quantifier, and, therefore, this is the very reason why the formula interpreting \txt{beats it} goes (via $\beta$-reduction) inside the scope of the existential quantifier that binds the variable interpreting \txt{a donkey}. The normalized resulting term is shown in~\eqref{eq:app4}:
\begin{align}
& (\tr{\cfarmer \x1}  \cnj2   \ex2 (\lambda \z1.  \tr{\cdonkey \z1}\cnj2 \tr{\cown \x1 \z1}  ) )  \cnj2  \n2 ( \lambda e \phi. \tr{\cbeat \x1  ( \selK_{it} e )} e \phi  )  \notag \\
 \bred \ &  \lambda e \phi. \cfarmer \x1  \cnj1 \ex1 ( \lambda \z1. \cdonkey \z1  \cnj1 ( \cown \x1  \z1 \cnj1 ( \n1 \cbeat  \x1 ( \selK_{it} e ) \cnj1 \phi  (\z1::e) ) ))\label{eq:app4}
\end{align}
Since, due to the definition of $\cnj2$,  $\n1 \cbeat  \x1 ( \selK_{it} e )$ entered deep inside the formula $   \lambda e \phi. \cfarmer \x1  \cnj1 \ex1 ( \lambda \z1. \cdonkey \z1  \cnj1 ( \cown \x1  \z1 \cnj1 \phi  (\z1::e) ) )$, the selection function responsible for solving anaphor for \txt{it} can become bound with the quantified variable resulted from interpreting \txt{a donkey} without violating any law of logic. %In other words, pronoun \txt{it} can become bound to \txt{a donkey} and this is achieved by notion of $\beta$-reduction only. 

We continue with analizing $$\ex2 (\lambda \x1.  \tr{\cfarmer \x1} \cnj2   \ex2 (\lambda \z1.  \tr{\cdonkey \z1}\cnj2 \tr{\cown \x1 \z1}  )  \cnj2  \n2 ( \lambda e \phi. \tr{\cbeat \x1  ( \selK_{it} e )} e \phi  )) $$ which is, according to~\eqref{eq:app4}, $\beta$-equivalent to $$\ex2 (\lambda \x1. \lambda e \phi. \cfarmer \x1  \cnj1 \ex1 ( \lambda \z1. \cdonkey \z1  \cnj1 ( \cown \x1  \z1 \cnj1 ( \n1 \cbeat  \x1 ( \selK_{it} e ) \cnj1 \phi  (\z1::e) ) ))) $$ By~\eqref{def:DLCnst-FO-ex}, it is equivalent to the following term:
\begin{align}
\lambda e \phi. \ex1 (\lambda \x1.  \cfarmer \x1  \cnj1 \ex1 ( \lambda \z1. \cdonkey \z1  \cnj1 ( \cown \x1  \z1 \cnj1 ( \n1 \cbeat  \x1 ( \selK_{it} e ) \cnj1 \phi  (\z1::\x1::e) ) ))) \label{eq:app7}
\end{align}
Now we only need to apply $\n2$ to the term~\eqref{eq:app7} using Equation~\eqref{def:DLCnst-FO-neg}. This results in the final desired interpretation:
\begin{align}
& \lambda e \phi. \n1 \ex1 (\lambda \x1.  \cfarmer \x1  \cnj1 \ex1 ( \lambda \z1. \cdonkey \z1  \cnj1 ( \cown \x1  \z1 \cnj1 ( \n1 \cbeat  \x1 ( \selK_{it} e )  ) ))) \cnj1 \phi e \notag
\end{align}

\TODOK{Finish this example... Context $(\z1::\x1::e)$ was erased ...  mention DRT accessibility constraints...}

% consider commented text below!!!
\comments{

However, this is not always the case. For example, assuming accessibility constraint requirements of DRT, the individuals introduced by quantifiers in Sentence~\eqref{donkey-every-2006} should not be accessible for anaphoric triggers outside of the sentence. However, they clearly should be accessible for anaphoric pronouns within the sentence. 
\enumsentence{ \txt{Every farmer who owns a donkey beats it.} \label{donkey-every-2006}}
This accessibility constraint can also be implemented in the continuation-based dynamic approach. For example, lexical items of~\eqref{donkey-every-2006} can be assigned meanings, shown in Table~\ref{tbl:donkey-lexical-2006}, that lead to the desirable interpretation of the sentence, as demonstrated in Example~\ref{ex:donkeysentence}. %Since the lexical interpretations are dynamic, the resulting dynamic meaning of the donkey sentence does not suffer the drawbacks of the static meaning.

Note that in Equation~\eqref{eq:everyfarmerwhoownsadonkey-nf-2006} the context containing all the individuals with their properties collected during the computation is locally passed to the formula $\Q2$. The continuation $\phi$ receives only the context $e$ that is passed to the term as an argument; therefore, all individuals collected during the computation of the meaning of \txt{every farmer who owns a donkey} are not available outside the logical formula interpreting this phrase.


Note that the second argument of $\cbeat$, standing for the anaphoric pronoun, is not a free dummy variable, but a term $( \selK_{it} (\y1 :: \x1 :: e))$. This term consists of the selection function $\selK$ that takes as argument a context containing the available individuals ``collected'' during the computation. The function $\selK$, which implements an anaphora resolution algorithm, selects a required individual from the context, which, in the current case, is the individual $\y1$. This  leads to the final dynamic meaning~\eqref{eq:donkey-nf-meaning-3-2006} of Sentence~\eqref{donkey-every-2006}:
\begin{align}
\lambda e \phi. \forall ( \lambda \x1.  \cfarmer \x1  \rightarrow  \forall ( \lambda \y1.  ( \cdonkey \y1 \cnj1  \cown \x1 \y1 ) \rightarrow   \cbeat \x1 \y1  ) )  \cnj1 \phi e  \label{eq:donkey-nf-meaning-3-2006}
\end{align}

Moreover, the formula $\cbeat \x1 \y1 $ is within the scope of the quantifier binding the variable $y$, exactly as desired.  Furthermore, the quantifier binding $y$ has been changed during the computation from existential to universal. This is achieved by employing a continuation-passing technique. 

Finally, in accordance with DRT's accessibility constraint, the individuals bound by quantifiers are not accessible outside the sentence, because the continuation $\phi$ of the term simply receives non-updated context $e$ as an argument.






 Taking these dynamic interpretations to compute the meaning of Sentence~\eqref{donkey-every-FO}, term~\eqref{eq:FO-donkey-int0} $\beta$-reduces to term~\eqref{eq:FO-donkey-int1}, which normalizes to~\eqref{eq:FO-donkey-int1-equiv}:
\begin{align}
& \tr{\I{beats}}  \ \etr{\I{it}} (\tr{\I{every}}  ((\tr{\I{who}}  (\tr{\I{owns}}  (\tr{\I{a}} \ \tr{\I{donkey}} ) ) ) \tr{\I{farmer}}  ))  \label{eq:FO-donkey-int0} \\
\bred \ &  \all2 (\lambda \x1. (\tr{\cfarmer} \x1  \cnj2     \ex2 (\lambda \z1.   \tr{\cdonkey}   \z1  \cnj2      \tr{\textbf{o}}  \x1  \z1 ))    \impK2    (\lambda e \phi. \tr{\textbf{b}}  \x1 ( \selK_{it} e ) e \phi ) )  \label{eq:FO-donkey-int1} \\
\bred \ & \lambda e \phi. \forall (\lambda \x1. \cfarmer \x1  \rightarrow \forall  (\lambda \z1.  (\cdonkey  \z1 \cnj1 \cown  \x1 \z1 )   \rightarrow  \cbeat  \x1 ( \selK_{it}  (\upii{\x1}{\upii{\z1}{e}}) )))\cnj1 \phi e  \label{eq:FO-donkey-int1-equiv}
\end{align}


Resulting term~\eqref{eq:FO-donkey-int1-equiv} is equivalent to~\eqref{eq:donkey-nf-meaning-2-2006} obtained in framework {\G} interpretations. Indeed, framework {\GN} is equivalent to de Groote's~\cite{deGroote:2006:Towards-a-Montagovian-Account-of-Dynamics} framework {\G}. However, it is advantageous over {\G} due to the compact notations for dynamic terms. These notations significantly systematize the framework and make the interpretations more concise and intuitive. Moreover, the systematic translations of static terms into dynamic terms makes it possible to prove a conservation result~\ref{th:conservation:FO} for {\GN}, that guarantees that static and dynamic interpretations are satisfied in the same models. 

}



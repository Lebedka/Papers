\section{Examples} \label{sec:examples}


\subsection{Donkey Sentences}




\begin{table}
\begin{tabular}{ l l l l }
  Lexical item &  Static type & Static interpretation \\
  \hline
  \\
  \txt{farmer} &  ${\iota} \rightarrow {o}$ &  $\textbf{f}$   \\
  \txt{donkey} &  ${\iota} \rightarrow {o}$ &  $\textbf{d}$   \\
  \txt{owns} & $((\iota \rightarrow o) \rightarrow o) \rightarrow ((\iota \rightarrow o) \rightarrow o) \rightarrow {o}$  & $ \lambda \Y1 \X1. \X1 ( \lambda \x1. \Y1 (\lambda \y1.  \textbf{o}  \x1 \y1 ))$ \\
    \txt{beats} & $((\iota \rightarrow o) \rightarrow o) \rightarrow ((\iota \rightarrow o) \rightarrow o) \rightarrow {o}$  & $ \lambda \Y1 \X1. \X1 ( \lambda \x1. \Y1 (\lambda \y1.  \textbf{b}  \x1 \y1 ))$ \\
   \txt{every} & $({\iota} \rightarrow {o}) \rightarrow ( ({\iota} \rightarrow {o}) \rightarrow {o}) $ & $\lambda \P1 \Q1. \forall (\lambda x. \P1 x \rightarrow \Q1 x ) $   \\
   \txt{a} & $({\iota} \rightarrow {o}) \rightarrow ( ({\iota} \rightarrow {o}) \rightarrow {o}) $ & $ \lambda \P1 \Q1. \exists (\lambda x.  \P1 x \land \Q1 x )$ \\
  \txt{who} & $( ( ({\iota} \rightarrow {o}) \rightarrow {o} ) \rightarrow o  )  \rightarrow (\iota \rightarrow o)  \rightarrow (\iota \rightarrow o) $ & $ \lambda \R1 \Q1 \x1. \Q1 \x1 \land \R1 (\lambda \P1. \P1 \x1) $  \\
   \txt{it} & $ ({\iota} \rightarrow {o}) \rightarrow {o} $  & $\lambda \P1. \P1 ?$ \\ 
 \end{tabular}
\caption{Static lexical interpretations.} \label{tbl:stat-FO-donkey}
\end{table}
\begin{table}
\begin{tabular}{ l l l l l}
  Lexical item & Dynamic type & Dynamic interpretation in {\GN} \\
  \hline
  \\
  \txt{farmer} &  $\tr{\iota} \rightarrow \tr{o}$ &  $\tr{\textbf{f}}$  \\
    \txt{donkey} &  $\tr{\iota} \rightarrow \tr{o}$ &  $\tr{\textbf{d}}$  \\
   \txt{owns} & $((\tr{\iota} \rightarrow \tr{o}) \rightarrow \tr{o}) \rightarrow ((\tr{\iota} \rightarrow \tr{o}) \rightarrow \tr{o}) \rightarrow \tr{o}$  & $ \lambda \Y2 \X2. \X2 ( \lambda \x1. \Y2 (\lambda \y1.  \tr{\textbf{o}}  \x1 \y1 ))$ \\
      \txt{beats} & $((\tr{\iota} \rightarrow \tr{o}) \rightarrow \tr{o}) \rightarrow ((\tr{\iota} \rightarrow \tr{o}) \rightarrow \tr{o}) \rightarrow \tr{o}$  & $ \lambda \Y2 \X2. \X2 ( \lambda \x1. \Y2 (\lambda \y1.  \tr{\textbf{b}}  \x1 \y1 ))$ \\
   \txt{every} & $(\tr{\iota} \rightarrow \tr{o}) \rightarrow ( (\tr{\iota} \rightarrow \tr{o}) \rightarrow \tr{o}) $ & $\lambda \P2 \Q2. \all2 (\lambda \x1.  \P2 \x1  \impK2 \Q2 \x1 ) $ \\
    \txt{a} &  $(\tr{\iota} \rightarrow \tr{o}) \rightarrow ( (\tr{\iota} \rightarrow \tr{o}) \rightarrow \tr{o}) $  & $ \lambda \P2 \Q2. \ex2 (\lambda \x1.  \P2 \x1  \cnj2   \Q2 \x1 )$ \\
   \txt{who} & $( ( (\tr{\iota} \rightarrow \tr{o}) \rightarrow \tr{o} ) \rightarrow \tr{o}  )  \rightarrow (\tr{\iota} \rightarrow \tr{o})  \rightarrow (\tr{\iota} \rightarrow \tr{o}) $ & $\lambda \R2 \Q2 \x1. \Q2 \x1  \cnj2  \R2 (\lambda \P2. \P2 \x1 ) $\\
      \txt{it} & $ (\tr{\iota} \rightarrow \tr{o}) \rightarrow \tr{o} $ & $ \lambda \P2. \lambda e \phi. \P2 ( \selK_{it} e ) e \phi $ \\ 
   \end{tabular}
\caption{Dynamic lexical interpretations.} \label{tbl:dyn-FO-donkey}
\end{table}

Tables~\ref{tbl:stat-FO-donkey} and~\ref{tbl:dyn-FO-donkey} show respectively static and dynamic interpretations for the lexical items in the donkey sentence~\eqref{donkey-every-FO}:
\enumsentence{ \txt{Every farmer who owns a donkey beats it.} \label{donkey-every-FO}}
Note that the type of every dynamic term is analogous to its static type. The only difference is that each atomic type of a dynamic term is dynamized according to Definition~\ref{def:DTypes-FO}. All terms in Table~\ref{tbl:dyn-FO-donkey}, except the interpretation of the pronoun \txt{it}, are dynamized following the rules in Definition~\ref{def:DTerms-FO}. These rules allow the presentation of dynamic terms in a compact way. They ensure that dynamic terms are structurally analogous to their static counterparts and, therefore, are more intuitive. The dynamic interpretation of \txt{it} is constructed not by directly following the dynamization rules, because \txt{it} is a unconventional lexical item: there is an anaphor to be solved. Therefore, $\etr{\I{\txt{it}}}$\footnote{Here and further on, dynamic interpretations of unconventional lexical items are marked with tilde.} contains the selection function $\selK$ that takes a context (from which a referent has to be chosen) as an argument.



\begin{figure}[h!]
 \centering
    \includegraphics[width=0.85\textwidth]{images/Donkey.pdf}
\caption{Syntactic parse tree of sentence \txt{Every farmer who owns a donkey beats it}.} \label{fig:donkey-copy}
\end{figure}

According to the parse tree, shown in Figure~\ref{fig:donkey} and for convenience repeated in Figure~\ref{fig:donkey-copy}, the dynamic meaning can be computed by $\beta$-reducing the term~\eqref{term:donkey-dynamic}:
\begin{align}
\tr{\I{beats}}  \ \etr{\I{it}} (\tr{\I{every}} ( (\tr{\I{who}}  (\tr{\I{owns}}  (\tr{\I{a}} \ \tr{\I{donkey}} ) ) ) \tr{\I{farmer}}  )) \label{term:donkey-dynamic}
\end{align}
The dynamic lexical interpretations (in compact form) for words in Sentence~\eqref{donkey-every} are shown in Table~\ref{tbl:donkey-lexical}. These interpretations comply with dynamization principles discussed in Sections~\ref{sec:StandardInterpretations} and~\ref{sec:ExceptionTriggers}. Since the meanings of the lexical items are taken in compact form, the computation of the meaning of the sentence is free of ``bureaucracy'' and more intuitive compared to the computation using the normalized lexical interpretations. 




The meaning of the noun phrase \txt{a donkey} is computed by reducing the term $\tr{\I{a}} \ \tr{\I{donkey}} $:
\begin{align*}
 \tr{\I{a}} \ \tr{\I{donkey}}   = \ & (\lambda \P2 \Q2. \ex2 (\lambda \x2.  \P2 \x2 \cnj2 \Q2 \x2 ) ) \tr{\I{donkey}}   \\
 \bconv \ &  \lambda  \Q2. \ex2 (\lambda \x2.  \tr{\I{donkey}} \x2 \ \cnj2 \ \Q2 \x2 ) \\
= \ &  \lambda  \Q2. \ex2 (\lambda \x2.  \tr{\cdonkey} \x2 \ \cnj2 \ \Q2 \x2 )
\end{align*}

The term $ \tr{\I{owns}}  (\tr{\I{a}} \ \tr{\I{donkey}}) $ is $\beta$-reduced as follows:
\begin{align*}
\tr{\I{owns}}  (\tr{\I{a}} \ \tr{\I{donkey}} ) = \ & ( \lambda \Y2 \X2. \X2 ( \lambda \x2. \Y2 (\lambda \y2.  \tr{\cown}  \x2 \y2 )))  (\tr{\I{a}} \ \tr{\I{donkey}} ) \\
\bconv \ &   \lambda \X2. \X2 ( \lambda \x2. \tr{\I{a}} \ \tr{\I{donkey}} (\lambda \y2.  \tr{\cown}  \x2 \y2 ))   \\
= \ &  \lambda \X2. \X2 ( \lambda \x2. ( \lambda  \Q2. \ex2 (\lambda \z2.  \tr{\cdonkey} \z2 \ \cnj2 \ \Q2 \z2 ) ) (\lambda \y2.  \tr{\cown}  \x2 \y2 ))   \\
\bconv \ &  \lambda \X2. \X2 ( \lambda \x2. \ex2 (\lambda \z2.  \tr{\cdonkey} \z2 \ \cnj2 \ (\lambda \y2.  \tr{\cown}  \x2 \y2 ) \z2   ))   \\
\bconv \ & \lambda \X2. \X2 ( \lambda \x2.  \ex2 (\lambda \z2. \tr{\cdonkey}  \z2 \ \cnj2  \ \tr{\cown}  \x2 \z2 ))  
\end{align*}

The meaning of the relative clause \txt{who owns a donkey} is computed by $\beta$-reducing the term $\tr{\I{who}}  (\tr{\I{owns}}  (\tr{\I{a}} \ \tr{\I{donkey}} ))$:
\begin{align*}
 \tr{\I{who}}  (\tr{\I{owns}}  (\tr{\I{a}} \ \tr{\I{donkey}} )) = \ & (\lambda \R2 \Q2 \y2.  \Q2 \y2 \ \cnj2 \ \R2 (\lambda \P2. \P2 \y2 ) ) (\tr{\I{owns}}  (\tr{\I{a}} \ \tr{\I{donkey}} )) \\
\bconv \ &  \lambda \Q2 \y2. \Q2 \y2 \ \cnj2 \ \tr{\I{owns}}  (\tr{\I{a}} \ \tr{\I{donkey}}) (\lambda \P2. \P2 \y2)   \\
= \ & \lambda \Q2 \y2. \Q2 \y2 \ \cnj2 \ ( \lambda \X2. \X2 ( \lambda \x2.  \ex2 (\lambda \z2.  \tr{\cdonkey}  \z2 \ \cnj2  \ \tr{\cown}  \x2 \z2 ) ) )  (\lambda \P2. \P2 \y2)   \\
\bconv \ & \lambda \Q2 \y2.  \Q2 \y2 \ \cnj2 \ ( \lambda \P2. \P2 \y2) ( \lambda \x2.  \ex2 (\lambda \z2.  \tr{\cdonkey}  \z2 \ \cnj2  \ \tr{\cown}  \x2 \z2 ) )     \\
\bconv \ & \lambda \Q2 \y2. \Q2 \y2 \ \cnj2 \ (   \lambda \x2.  \ex2 (\lambda \z2.   \tr{\cdonkey}  \z2 \ \cnj2  \ \tr{\cown}  \x2 \z2)    \y2)   \\
\bconv \ & \lambda \Q2 \y2.  \Q2 \y2 \ \cnj2 \  \ex2 (\lambda \z2.  \tr{\cdonkey}  \z2 \ \cnj2  \ \tr{\cown}  \y2 \z2  ) 
\end{align*}

The meaning of \txt{farmer who owns a donkey} is computed by applying the resulting $\lambda$-term in the computation above to the interpretation of \txt{farmer}:
\begin{align*}
 (\tr{\I{who}}  (\tr{\I{owns}}  (\tr{\I{a}} \ \tr{\I{donkey}} ) ) )\tr{\I{farmer}} 
 = \ & (\lambda \Q2 \y2.  \Q2 \y2 \ \cnj2 \  \ex2 (\lambda \z2.  \tr{\cdonkey}  \z2 \ \cnj2  \ \tr{\cown}  \y2 \z2  )) \tr{\I{farmer}} \\
\bconv \ & \lambda  \y2.  \tr{\I{farmer}} \y2 \ \cnj2 \  \ex2 (\lambda \z2.  \tr{\cdonkey}  \z2 \ \cnj2  \ \tr{\cown}  \y2 \z2  )   \\ 
= \ & \lambda  \y2.  \tr{\cfarmer} \y2 \ \cnj2 \  \ex2 (\lambda \z2.  \tr{\cdonkey} \z2 \ \cnj2  \ \tr{\cown}  \y2 \z2 )   
\end{align*}

The meaning of the noun phrase \txt{every farmer who owns a donkey} is computed by applying the interpretation of \txt{every} to the interpretation of \txt{farmer who owns a donkey}: 
\begin{align}
& \tr{\I{every}}  ((\tr{\I{who}}  (\tr{\I{owns}}  (\tr{\I{a}} \ \tr{\I{donkey}} ) )) \tr{\I{farmer}}  )  \notag \\
= \ & (\lambda \P2 \Q2. \all2 (\lambda \x2.  \P2 \x2 \ \impK2 \ \Q2 \x2 ) ) ((\tr{\I{who}}  (\tr{\I{owns}}  (\tr{\I{a}} \ \tr{\I{donkey}} ) )) \tr{\I{farmer}}  )  \notag \\
\bconv \ & \lambda \Q2. \all2 (\lambda \x2.  ((\tr{\I{who}}  (\tr{\I{owns}}  (\tr{\I{a}} \ \tr{\I{donkey}} ) )) \tr{\I{farmer}}  ) \x2 \ \impK2 \ \Q2 \x2 )   \notag \\
= \ & \lambda \Q2. \all2 (\lambda \x2.  (\lambda  \y2.  \tr{\cfarmer}  \y2 \ \cnj2 \  \ex2 (\lambda \z2. \tr{\cdonkey}  \z2 \ \cnj2  \ \tr{\cown}  \y2 \z2  )   ) \x2 \ \impK2 \ \Q2 \x2)    \notag \\
\bconv \ & \lambda \Q2. \all2 (\lambda \x2.  ( \tr{\cfarmer}  \x2 \ \cnj2 \  \ex2 (\lambda \z2.  \tr{\cdonkey}  \z2 \ \cnj2  \ \tr{\cown}  \x2 \z2  ) )  \ \impK2 \ \Q2 \x2)   \label{eq:everyfarmerwhoownsadonkey}
\end{align}



\todo{Change interpretation of ``it´´ here.}
The meaning of the verb phrase \txt{beats it} is computed as follows:
\begin{align}
\tr{\I{beats}}  \ \etr{\I{it}}  = \ &  (\lambda \Y2 \X2. \X2 ( \lambda \x2. \Y2 (\lambda \y2.  \tr{\cbeat}  \x2 \y2 ))) \etr{\I{it}} \notag  \\
\bconv \ &  \lambda  \X2. \X2 ( \lambda \x2.  \etr{\I{it}}  (\lambda \y2.  \tr{\cbeat}  \x2 \y2 ))\notag  \\
= \ & \lambda  \X2. \X2 ( \lambda \x2.  (\lambda \P2. \P2 ( \sel ( \lambda \z1. \mathbf{non\_human} \z1 ) ))  (\lambda \y2.  \tr{\cbeat}  \x2 \y2 ))\notag  \\
\bconv \ &  \lambda  \X2. \X2 ( \lambda \x2.   (\lambda \y2.  \tr{\cbeat}  \x2 \y2 ) ( \sel ( \lambda \z1. \mathbf{non\_human} \z1 )) )\notag  \\
\bconv \ &  \lambda  \X2. \X2 ( \lambda \x2.   \tr{\cbeat}  \x2 ( \sel ( \lambda \z1. \mathbf{non\_human} \z1 ))  ) \label{eq:beatsit}
\end{align}

\todo{Fix computation here.}
Finally, the meaning of the sentence is computed by applying the interpretation~\eqref{eq:beatsit} to the interpretation~\eqref{eq:everyfarmerwhoownsadonkey}:
%\begin{footnotesize}
\begin{align}
& \tr{\I{beats}}  \ \etr{\I{it}} (\tr{\I{every}}  ((\tr{\I{who}}  (\tr{\I{owns}}  (\tr{\I{a}} \ \tr{\I{donkey}} ) ) ) \tr{\I{farmer}}  )) \notag \\
= \ & (\lambda  \X2. \X2 ( \lambda \x2.   \tr{\cbeat}  \x2 ( \sel ( \lambda \z1. \mathbf{non\_human} \z1 ))  )) \notag \\
& (\tr{\I{every}}  ((\tr{\I{who}}  (\tr{\I{owns}}  (\tr{\I{a}} \ \tr{\I{donkey}} ) ) ) \tr{\I{farmer}}  ))  \notag \\
\bconv \ & (\tr{\I{every}}  ((\tr{\I{who}}  (\tr{\I{owns}}  (\tr{\I{a}} \ \tr{\I{donkey}} ) ) ) \tr{\I{farmer}}  ))  \notag \\
&  ( \lambda \y2.   \tr{\cbeat}  \y2 ( \sel ( \lambda \z1. \mathbf{non\_human} \z1 ))  )  \notag \\
= \ &  (\lambda \Q2. \all2 (\lambda \x2.  ( \tr{\cfarmer} \x2 \ \cnj2 \  \ex2 (\lambda \z2.  \tr{\cdonkey}  \z2 \ \cnj2  \ \tr{\cown}  \x2 \z2 )   ) \ \impK2 \ \Q2 \x2 ) ) ( \lambda \y2.   \tr{\cbeat}  \y2 ( \sel ( \lambda \z1. \mathbf{non\_human} \z1 ))  )  \notag \\
\bconv \ &  \all2 (\lambda \x2.  (\tr{\cfarmer} \x2 \ \cnj2 \  \ex2 (\lambda \z2.  \tr{\cdonkey}  \z2 \ \cnj2  \ \tr{\cown}  \x2 \z2 )) \ \impK2 \ ( \lambda \y2.   \tr{\cbeat}  \y2 ( \sel ( \lambda \z1. \mathbf{non\_human} \z1 ))  )   \x2)   \notag \\
\bconv \ &  \all2 (\lambda \x2. (  \tr{\cfarmer} \x2 \ \cnj2 \  \ex2 (\lambda \z2.  \tr{\cdonkey}  \z2 \ \cnj2  \ \tr{\cown}  \x2 \z2  )  ) \ \impK2 \  \tr{\cbeat}  \x2 ( \sel ( \lambda \z1. \mathbf{non\_human} \z1 )) ) \label{eq:donkey-compact-meaning}
\end{align}
%\end{footnotesize}
Formula~\eqref{eq:donkey-compact-meaning} is the resulting dynamic meaning of Sentence~\eqref{donkey-every} in the compact form. Its computation is analogous to the computation of the static formula~\eqref{eq:donkey-st-meaning}, and therefore more intuitive than if normalized (not compact, unfolded) interpretations of lexical items were used. Due to the similarity with the ``static'' computation, it may seem, at the first glance, that the formula~\eqref{eq:donkey-compact-meaning} has the same problems as the static formula~\eqref{eq:donkey-st-meaning}. However, this is not the case! 

 %As shown below, formula~\eqref{eq:donkey-compact-meaning} normalizes to formula~\eqref{eq:donkey-nf-meaning}, thus overcoming the drawbacks of the static interpretation.
The ``secret'' of having the computation similar to and as intuitive as the static one but without the ``static problems'' lies in the fact that all the subterms used, particularly those standing for the logical connectives, are dynamic. Term~\eqref{eq:donkey-compact-meaning} is analysed below in three ways. Firstly, simply by applying Proposition~\ref{thrm:scopeextension} to it; secondly, by examining its structure and its dynamic subterms; and finally, by normalizing it by performing step-by-step $\beta$-reductions.

Remember that by  Remark~\ref{rem:MorganLaws-dynamic} the following equations hold: 
\begin{subequations}
\begin{align}
\all2 \ \defeq \ & \lambda \P2. \n2 \ex2 (\n2 \P2) \\
\impK2 \ \defeq \ & \lambda \A2 \B2. \n2 ( \A2 \cnj2 \n2 \B2)
\end{align} \label{eq:logsymbabbrev}
\end{subequations}

Therefore, term~\eqref{eq:donkey-compact-meaning} is equivalent to~\eqref{eq:donkey-compact-meaning-2}:
\begin{align}
%\n2 \ex2 (\lambda \x2.  (\tr{\cfarmer}  \x2 \ \cnj2 \  \ex2 (\lambda \z2.  \tr{\cdonkey}  \z2 \ \cnj2  \ \tr{\cown}  \x2 \z2  ) ) \ \cnj2 \ \n2 \tr{\cbeat}  \x2 ( \sel ( \lambda \z1. \mathbf{non\_human} \z1 ))  )   \label{eq:donkey-compact-meaning-2}
\n2 \ex2 (\lambda \x2.  \tr{\cfarmer}  \x2 \ \cnj2 \  \ex2 (\lambda \z2.  \tr{\cdonkey}  \z2 \ \cnj2  \ \tr{\cown}  \x2 \z2  )  \ \cnj2 \ \n2 \tr{\cbeat}  \x2 ( \sel ( \lambda \z1. \mathbf{non\_human} \z1 ))  )   \label{eq:donkey-compact-meaning-2}
\end{align}

\subsubsection{Analysis 1}
\TODO{See sec-DS-DonkeySentences.tex in the thesis.}

\subsubsection{Analysis 2}
\TODO{See sec-DS-DonkeySentences.tex in the thesis.}

\subsubsection{Analysis 3}
\TODO{See sec-DS-DonkeySentences.tex in the thesis.}


% commented text below!!!
\comments{

However, this is not always the case. For example, assuming accessibility constraint requirements of DRT, the individuals introduced by quantifiers in Sentence~\eqref{donkey-every-2006} should not be accessible for anaphoric triggers outside of the sentence. However, they clearly should be accessible for anaphoric pronouns within the sentence. 
\enumsentence{ \txt{Every farmer who owns a donkey beats it.} \label{donkey-every-2006}}
This accessibility constraint can also be implemented in the continuation-based dynamic approach. For example, lexical items of~\eqref{donkey-every-2006} can be assigned meanings, shown in Table~\ref{tbl:donkey-lexical-2006}, that lead to the desirable interpretation of the sentence, as demonstrated in Example~\ref{ex:donkeysentence}. %Since the lexical interpretations are dynamic, the resulting dynamic meaning of the donkey sentence does not suffer the drawbacks of the static meaning.

Note that in Equation~\eqref{eq:everyfarmerwhoownsadonkey-nf-2006} the context containing all the individuals with their properties collected during the computation is locally passed to the formula $\Q2$. The continuation $\phi$ receives only the context $e$ that is passed to the term as an argument; therefore, all individuals collected during the computation of the meaning of \txt{every farmer who owns a donkey} are not available outside the logical formula interpreting this phrase.


Note that the second argument of $\cbeat$, standing for the anaphoric pronoun, is not a free dummy variable, but a term $( \selK_{it} (\y1 :: \x1 :: e))$. This term consists of the selection function $\selK$ that takes as argument a context containing the available individuals ``collected'' during the computation. The function $\selK$, which implements an anaphora resolution algorithm, selects a required individual from the context, which, in the current case, is the individual $\y1$. This  leads to the final dynamic meaning~\eqref{eq:donkey-nf-meaning-3-2006} of Sentence~\eqref{donkey-every-2006}:
\begin{align}
\lambda e \phi. \forall ( \lambda \x1.  \cfarmer \x1  \rightarrow  \forall ( \lambda \y1.  ( \cdonkey \y1 \land  \cown \x1 \y1 ) \rightarrow   \cbeat \x1 \y1  ) )  \land \phi e  \label{eq:donkey-nf-meaning-3-2006}
\end{align}

Moreover, the formula $\cbeat \x1 \y1 $ is within the scope of the quantifier binding the variable $y$, exactly as desired.  Furthermore, the quantifier binding $y$ has been changed during the computation from existential to universal. This is achieved by employing a continuation-passing technique. 

Finally, in accordance with DRT's accessibility constraint, the individuals bound by quantifiers are not accessible outside the sentence, because the continuation $\phi$ of the term simply receives non-updated context $e$ as an argument.






 Taking these dynamic interpretations to compute the meaning of Sentence~\eqref{donkey-every-FO}, term~\eqref{eq:FO-donkey-int0} $\beta$-reduces to term~\eqref{eq:FO-donkey-int1}, which normalizes to~\eqref{eq:FO-donkey-int1-equiv}:
\begin{align}
& \tr{\I{beats}}  \ \etr{\I{it}} (\tr{\I{every}}  ((\tr{\I{who}}  (\tr{\I{owns}}  (\tr{\I{a}} \ \tr{\I{donkey}} ) ) ) \tr{\I{farmer}}  ))  \label{eq:FO-donkey-int0} \\
\bred \ &  \all2 (\lambda \x1. (\tr{\textbf{f}} \x1  \cnj2     \ex2 (\lambda \z1.   \tr{\textbf{d}}   \z1  \cnj2      \tr{\textbf{o}}  \x1  \z1 ))    \impK2    (\lambda e \phi. \tr{\textbf{b}}  \x1 ( \selK_{it} e ) e \phi ) )  \label{eq:FO-donkey-int1} \\
\bred \ & \lambda e \phi. \forall (\lambda \x1. \cfarmer \x1  \rightarrow \forall  (\lambda \z1.  (\cdonkey  \z1 \land \cown  \x1 \z1 )   \rightarrow  \cbeat  \x1 ( \selK_{it}  (\upii{\x1}{\upii{\z1}{e}}) )))\land \phi e  \label{eq:FO-donkey-int1-equiv}
\end{align}


Resulting term~\eqref{eq:FO-donkey-int1-equiv} is equivalent to~\eqref{eq:donkey-nf-meaning-2-2006} obtained in framework {\G} interpretations. Indeed, framework {\GN} is equivalent to de Groote's~\cite{deGroote:2006:Towards-a-Montagovian-Account-of-Dynamics} framework {\G}. However, it is advantageous over {\G} due to the compact notations for dynamic terms. These notations significantly systematize the framework and make the interpretations more concise and intuitive. Moreover, the systematic translations of static terms into dynamic terms makes it possible to prove a conservation result~\ref{th:conservation:FO} for {\GN}, that guarantees that static and dynamic interpretations are satisfied in the same models. 

}



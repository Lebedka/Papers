%%%%% Dynamic connectives and quantifiers

\newcommand{\dynexists}{\mathord{\reflectbox{$\mathds{E}$}}}

\newcommand{\dynall}{\mathord{%
\raisebox{\depth}{\rotatebox{180}{$\mathds{A}$}}}}

\newcommand{\dynor}{\mathbin{\mathds{V}}}

\newcommand{\dynand}{\mathbin{%
\reflectbox{\raisebox{\depth}{\rotatebox{180}{${\dynor}$}}}}}

\newcommand{\dynimplies}{\mathbin{%
\mbox{\raisebox{.28ex}{\rule{1.7ex}{.15ex}}%
\hspace{-1.7ex}\raisebox{.73ex}{\rule{1.7ex}{.15ex}}%
\hspace{-.4ex}\raisebox{-.01ex}{\rotatebox{45}{\rule{.83ex}{.15ex}}}%
\hspace{-.69ex}\raisebox{.58ex}{\rotatebox{135}{\rule{.83ex}{.15ex}}}}
}}

\newcommand{\dynneg}{\mathop{%
\mbox{\raisebox{.68ex}{\rule{1.2ex}{.15ex}}%
\hspace{-.15ex}\raisebox{.2ex}{\rule{.15ex}{.63ex}}%
\hspace{.25ex}\raisebox{.2ex}{\rule{.15ex}{.63ex}}\hspace{.15ex}}
}}

%%%%% lambda-terms

\newif\ifatomic
\newif\ifoperator
\newif\ifconj
\newif\ifdisj
\newif\ifdconj
\newif\ifddisj
\newif\ifcons

\def\reset{%
\atomicfalse\operatorfalse\conjfalse\disjfalse\dconjfalse\ddisjfalse\consfalse}

% abstraction

\def\abs#1#2%
{\ifatomic%
(\lambda #1 . \, \reset #2)%
\else%
\lambda #1 . \, \reset #2%
\fi\reset}

% application

\def\app#1#2%
{\ifoperator%
\reset\atomictrue\operatortrue #1 \, \reset\atomictrue #2%
\else%
\ifatomic%
(\reset\atomictrue\operatortrue #1 \, \reset\atomictrue #2)%
\else%
\reset\atomictrue\operatortrue #1 \, \reset\atomictrue #2%
\fi%
\fi\reset}

% universal quantification

\def\uquant#1#2%
{\ifatomic%
(\forall #1 . \, \reset #2)%
\else%
\forall #1 . \, \reset #2%
\fi\reset}

% existential quantification

\def\equant#1#2%
{\ifatomic%
(\exists #1 . \, \reset #2)%
\else%
\exists #1 . \, \reset #2%
\fi\reset}

% implication

\def\imp#1#2%
{\ifatomic%
(\reset\atomictrue #1 \rightarrow \reset\atomictrue #2)%
\else%
\reset\atomictrue #1 \rightarrow \reset\atomictrue #2%
\fi\reset}

% conjunction

\def\conj#1#2%
{\ifconj%
\reset\atomictrue\conjtrue #1 \wedge \reset\atomictrue\conjtrue #2%
\else%
\ifatomic%
(\reset\atomictrue\conjtrue #1 \wedge \reset\atomictrue\conjtrue #2)%
\else%
\reset\atomictrue\conjtrue #1 \wedge \reset\atomictrue\conjtrue #2%
\fi%
\fi\reset}

% disjunction

\def\disj#1#2%
{\ifdisj%
\reset\atomictrue\disjtrue #1 \vee \reset\atomictrue\disjtrue #2%
\else%
\ifatomic%
(\reset\atomictrue\disjtrue #1 \vee \reset\atomictrue\disjtrue #2)%
\else%
\reset\atomictrue\disjtrue #1 \vee \reset\atomictrue\disjtrue #2%
\fi%
\fi\reset}

% negation

\def\nega#1%
{\neg \reset\atomictrue #1\reset}

% dynamic universal quantification

\def\duquant#1#2%
{\ifatomic%
(\dynall #1 . \, \reset #2)%
\else%
\dynall #1 . \, \reset #2%
\fi\reset}

% dynamic existential quantification

\def\dequant#1#2%
{\ifatomic%
(\dynexists #1 . \, \reset #2)%
\else%
\dynexists #1 . \, \reset #2%
\fi\reset}

% dynamic implication

\def\dimp#1#2%
{\ifatomic%
(\reset\atomictrue #1 \dynimplies \reset\atomictrue #2)%
\else%
\reset\atomictrue #1 \dynimplies \reset\atomictrue #2%
\fi\reset}

% dynamic conjunction

\def\dconj#1#2%
{\ifdconj%
\reset\atomictrue\dconjtrue #1 \dynand \reset\atomictrue\dconjtrue #2%
\else%
\ifatomic%
(\reset\atomictrue\dconjtrue #1 \dynand \reset\atomictrue\dconjtrue #2)%
\else%
\reset\atomictrue\dconjtrue #1 \dynand \reset\atomictrue\dconjtrue #2%
\fi%
\fi\reset}

% dynamic disjunction

\def\ddisj#1#2%
{\ifddisj%
\reset\atomictrue\ddisjtrue #1 \dynor \reset\atomictrue\ddisjtrue #2%
\else%
\ifatomic%
(\reset\atomictrue\ddisjtrue #1 \dynor \reset\atomictrue\ddisjtrue #2)%
\else%
\reset\atomictrue\ddisjtrue #1 \dynor \reset\atomictrue\ddisjtrue #2%
\fi%
\fi\reset}

% dynamic negation

\def\dnega#1%
{\dynneg \reset\atomictrue #1\reset}

% closure
\def\close#1%
{{\diamond} \reset\atomictrue #1\reset}

% environment

\def\nil{\mathsf{nil}}

\def\sel#1{\app{\mathsf{sel}}{#1}}

\def\cons#1#2%
{\ifcons%
\reset\atomictrue #1 \mathbin{{::}} \reset\atomictrue\construe #2%
\else%
\ifatomic%
(\reset\atomictrue #1 \mathbin{{::}} \reset\atomictrue\construe #2)%
\else%
\reset\atomictrue #1 \mathbin{{::}} \reset\atomictrue\construe #2%
\fi%
\fi\reset}

%%%%% definitional equality


\newcommand{\defeq}{\stackrel{{\scriptstyle\triangle}}{=}}


%%%%% Interpreatations and translations

\def\dplsem#1{\llbracket #1 \rrbracket^{\mbox{\scriptsize\textsc{dpl}}}}

\def\tdlsem#1#2{\llbracket #2 \rrbracket^{\mbox{\scriptsize\textsc{tdl}}}_{#1}}

\def\synlift#1{\lceil#1\rceil^{\mbox{\scriptsize\textsc{syn}}}}

\def\semlift#1{\lceil#1\rceil^{\mbox{\scriptsize\textsc{sem}}}}

\def\dplvalid#1#2{#1 \models_{\mbox{\scriptsize\textsc{dpl}}} #2}

\def\tdlvalid#1#2{#1 \models_{\mbox{\scriptsize\textsc{tdl}}} #2}

%%%%% Sets of environments and continuations

\def\env{\mathbb{E}}
\def\cont{\mathbb{C}}

%%%%% dynamic embedding & static projection 

\def\dyn#1#2{\app{\mathbb{D_{#1}}}{#2}}

\def\stat#1#2{\app{\mathbb{S_{#1}}}{#2}}

%%%%%%%%%%%%%%%%%%%%%%%%%%%%%%%%%%%%%%%%%%%%%%%%%%%%%%%%%%%%%%%%%%%%%%%%


%------------------------------------------------------------------------------------
% New Commands: Miscelaneous abbreviations
%------------------------------------------------------------------------------------

\newcommand{\TDL}{TDL} %name of the framework
\newcommand{\FullName}{Type-theoretic Dynamic Logic} %name of the framework, Typed Dynamic Logic ???
\newcommand{\comments}[1]{} % comments

\newcommand{\qex}{\qed}

\newcommand{\I}[1]{[\![#1]\!]}  % interpretation

\newcommand{\txt}[1]{\textit{#1}} % italic
%\newcommand{\defeq}{\doteq} % equal by definition
\newcommand{\logeq}{\equiv} % logical equivalence
\newcommand{\seq}{=_s} % syntactic equality
\newcommand{\nseq}{\neq_s} % syntactic non-equality

\newcommand{\congr}{=_{\alpha}} % alpha-congruent
\newcommand{\ncongr}{\neq_{\alpha}} % alpha non-congruent

\newcommand{\trr}[1]{#1} % this one does not do anything

\newcommand{\Types}{\text{T}} % Types

\newcommand{\upii}[2]{#1 ::  #2}  % list constructor
\newcommand{\selK}{\mathsf{sel} } % selection function
\newcommand{\updt}{\mathsf{upd}}  % discourse update
\newcommand{\D}{\mathbf{D}} % discourse
\renewcommand{\S}{\mathbf{S}} % sentence

\newcommand{\tr}[1]{\overline{#1}} % dynamizing
\newcommand{\ttr}[1]{ \overline{#1}^*} 
\newcommand{\etr}[1]{ \widetilde{#1}} % non-conventional dynamic term


\newcommand{\dnm}[2]{\mathbb{D}_{#1} [ #2] } % dynamization function with arguments
\newcommand{\std}[2]{\mathbb{S}_{#1} [ #2] } % reading function with arguments


\newcommand{\state}{\mathsf{state}}
\newcommand{\expr}{\mathsf{expr}} 
\newcommand{\Boolexpr}{\mathsf{Bool\_expr}} 
\newcommand{\abort}{\mathsf{abort}} 
\newcommand{\vvar}{\mathsf{var}} 
\newcommand{\Print}{\mathsf{Print}} 



%------------------------------------------------------------------------------------
% New Commands: Lambda Calculus
%------------------------------------------------------------------------------------
\newcommand{\bconv}{\rightarrow_\beta}  %one-step beta reduction
\newcommand{\bred}{\rightarrow^{*}_\beta} %beta reduction
\newcommand{\econv}{\rightarrow_\eta} %eta conversion 


%------------------------------------------------------------------------------------
% New Commands: 
% Abbreviations for distinguishing variables for static and dynamic cases.
% E.g.: \x1 - static, \x2 - dynamic
%------------------------------------------------------------------------------------
\newcommand{\A}[1]{\ifcase #1 \or \AAA\or\AB\fi}
\newcommand{\B}[1]{\ifcase #1 \or \BA\or\BB\fi}
\newcommand{\R}[1]{\ifcase #1 \or \RA\or\RB\fi}
\newcommand{\Q}[1]{\ifcase #1 \or \QA\or\QB\fi}
\renewcommand{\P}[1]{\ifcase #1 \or \PA\or\PB\fi}
\newcommand{\X}[1]{\ifcase #1 \or \XA\or\XB\fi}
\newcommand{\Y}[1]{\ifcase #1 \or \YA\or\YB\fi}
\newcommand{\x}[1]{\ifcase #1 \or \xA\or\xB\fi}
\newcommand{\y}[1]{\ifcase #1 \or \yA\or\yB\fi}
\newcommand{\z}[1]{\ifcase #1 \or \zA\or\zB\fi}
\renewcommand{\t}[1]{\ifcase #1 \or \tA\or\tB\fi}
\renewcommand{\k}[1]{\ifcase #1 \or \kA\or\kB\fi}
\newcommand{\h}[1]{\ifcase #1 \or \hA\or\hB\fi}
\renewcommand{\v}[1]{\ifcase #1 \or \vA\or\vB\fi}
\renewcommand{\u}[1]{\ifcase #1 \or \uA\or\uB\fi}
\newcommand{\w}[1]{\ifcase #1 \or \wA\or\wB\fi}
\renewcommand{\a}[1]{\ifcase #1 \or \aA\or\aB\fi}


%------------------------------------------------------------------------------------

\newcommand{\AAA}{\mathrm{A}}
\newcommand{\AB}{\mathbf{A}}

\newcommand{\BA}{\mathrm{B}}
\newcommand{\BB}{\mathbf{B}}

\newcommand{\RA}{\mathrm{R}}
\newcommand{\RB}{\mathbf{R}}

\newcommand{\QA}{\mathrm{Q}}
\newcommand{\QB}{\mathbf{Q}}

\newcommand{\PA}{\mathrm{P}}
\newcommand{\PB}{\mathbf{P}}

\newcommand{\XA}{\mathrm{X}}
\newcommand{\XB}{\mathbf{X}}

\newcommand{\YA}{\mathrm{Y}}
\newcommand{\YB}{\mathbf{Y}}

\newcommand{\xA}{x}
\newcommand{\xB}{\mathbf{x}}

\newcommand{\yA}{y}
\newcommand{\yB}{\mathbf{y}}

\newcommand{\zA}{z}
\newcommand{\zB}{\mathbf{z}}

\newcommand{\tA}{t}
\newcommand{\tB}{\mathbf{t}}

\newcommand{\kA}{k}
\newcommand{\kB}{\mathbf{k}}

\newcommand{\hA}{h}
\newcommand{\hB}{\mathbf{h}}

\newcommand{\vA}{v}
\newcommand{\vB}{\mathbf{v}}

\newcommand{\uA}{u}
\newcommand{\uB}{\mathbf{u}}

\newcommand{\wA}{w}
\newcommand{\wB}{\mathbf{w}}

\newcommand{\aA}{\mathtt{a}}
\newcommand{\aB}{\mathbf{a}}


%------------------------------------------------------------------------------------
% New Commands: Non-logical constants
%------------------------------------------------------------------------------------

\newcommand{\cfarmer}{\textbf{f} }
\newcommand{\cdonkey}{\textbf{d} }
\newcommand{\cown}{\textbf{o} }
\newcommand{\cbeat}{\textbf{b} }

%------------------------------------------------------------------------------------
% New Commands: Frameworks
%------------------------------------------------------------------------------------

\newcommand{\expn}{\chi}  % exception

\newcommand{\G}{G$_0$} % framework introduced in SALT 2006 paper
\newcommand{\GN}{G} % first order framework
\newcommand{\GL}{GL} % higher order framework
\newcommand{\GLex}{GL$\expn$} % dynamic logic with exceptions

\renewcommand{\ex}[1]{\ifcase #1 \or \existsA\or\existsB\fi}
\newcommand{\all}[1]{\ifcase #1 \or \forallA\or\forallB\fi}
\newcommand{\impK}[1]{\ifcase #1 \or \impA\or\impB\fi}
\newcommand{\dsj}[1]{\ifcase #1 \or \lorA\or\lorB\fi}
\newcommand{\n}[1]{\ifcase #1 \or \negA\or\negB\fi}
\newcommand{\cnj}[1]{\ifcase #1 \or \landA\or\landB\fi}

\newcommand{\existsA}{\exists}
\newcommand{\existsB}{\dynexists} 

\newcommand{\forallA}{\exists}
\newcommand{\forallB}{\dynall}  

\newcommand{\impA}{\rightarrow}
\newcommand{\impB}{\dynimplies}

\newcommand{\negA}{\neg}
\newcommand{\negB}{\dynneg}

\newcommand{\landA}{\land}
\newcommand{\landB}{\dynand}

\newcommand{\lorA}{\lor}
\newcommand{\lorB}{\dynor} 

%------------------------------------------------------------------------------------
% New Commands: Colorful symbols for the comparison with DPL
%------------------------------------------------------------------------------------

\newcommand{\ntodl}{h_G}
%\newcommand{trntog}{$h_g$}

\newcommand{\ntodpl}{h_{\textit{DPL}}}

\newcommand{\dpltos}{s_{DPL}}
\newcommand{\mdpltos}[1]{ ( #1 )_{s_\textit{DPL}}}

\newcommand{\dltos}{\textcolor{red}{s_{G}}}
\newcommand{\mdltos}[1]{ \textcolor{red}{(} #1 \textcolor{red}{)_{s_{G}}}}

\newcommand{\dplstodls}{\textcolor{blue}{\lceil \cdot \rceil_S}}
\newcommand{\mdplstodls}[1]{\textcolor{blue}{\lceil}  #1 \textcolor{blue}{\rceil_S}}

\newcommand{\dplstodpll}{\textcolor{cyan}{t^{\lambda}_\textit{DPL}}}
\newcommand{\mdplstodpll}[1]{\textcolor{cyan}{(} #1 \textcolor{cyan}{ )_{t^{\lambda}_\textit{DPL}}}}

\newcommand{\dpltodl}{\textcolor{cyan}{t^{G}_\textit{DPL}}}
\newcommand{\mdpltodl}[1]{\textcolor{cyan}{(} #1 \textcolor{cyan}{ )_{t^{G}_\textit{DPL}}}}


\newcommand{\dpllstodl}{\textcolor{blue}{\lceil \cdot \rceil_L}}
\newcommand{\mdpllstodl}[1]{\textcolor{blue}{\lceil} #1 \textcolor{blue}{\rceil_L}}

\newcommand{\newDPL}{\textit{DPL}^\lambda_L}

\newcommand{\statesDPL}{\textit{DPL}_{S}}
\newcommand{\statesDL}{\GN_S}

\newcommand{\langDPL}{\textit{DPL}_L}
\newcommand{\langDL}{\GN_L}


%------------------------------------------------------------------------------------
% TODO Comments
% \TODO{comment} appears within the text
% \todo{comment} appears on the margin side of a page
%------------------------------------------------------------------------------------
\newif\ifcomments
\commentstrue
%\commentsfalse   % gobble comments, set for final version

\ifcomments
 \newcommand{\TODO}[1]{\par\noindent{ \color{red} \sf  TODO: #1}\par\medskip}
 \newcommand{\todo}[1]{\marginpar{ \color{red} \footnotesize{\sf TODO: #1}}}
\else
 \newcommand{\todo}[1]{}
 \newcommand{\TODO}[1]{}
\fi

\ifcomments
 \newcommand{\TODOK}[1]{\par\noindent{ \color{red} \sf  Katya: #1}\par\medskip}
 \newcommand{\todoK}[1]{\marginpar{ \color{red} \footnotesize{\sf Katya: #1}}}
\else
 \newcommand{\todoK}[1]{}
 \newcommand{\TODOK}[1]{}
\fi
